\documentclass[mathserif]{beamer}

\usepackage[italian]{babel}
\usepackage{euscript}
\usepackage[utf8]{inputenc}
\usepackage{amsmath, amsfonts, epsfig, xspace}
%\usepackage{algorithm,algorithmic}
\usepackage{pstricks,pst-node}
\usepackage{multimedia}
\usepackage[normal,tight,center]{subfigure}
\setlength{\subfigcapskip}{-.5em}
\usepackage{beamerthemesplit}
\usetheme{lankton-keynote}

%pacchetti come sulla tesi pdf
\usepackage{eurosym}
\usepackage{amsfonts}
\usepackage{latexsym}
%\usepackage{makeidx}
\usepackage{amsmath}
\usepackage{amsthm}
\usepackage{comment}
\usepackage{enumerate}

\usepackage{mathrsfs}




%%% per i diagrammi commutativi:
\usepackage{tikz}

\usetikzlibrary{matrix, arrows, scopes}
%\usepackage{tikzcd}


%Ridefinizione per il quoziente!
\def\quotient#1#2{%
   \raise1ex\hbox{$#1$}\Big/\lower1ex\hbox{$#2$}%
}

%%%%%%%%%%%%%%%%%%%%%%%%%%% Definizione teoremi: $%%%%%

\newtheorem{definizione}{Definizione}
\newtheorem{esempio}{Esempio}
\newtheorem{osservazione}{Osservazione}
\newtheorem{lemmax}{Lemma}[section]
\newtheorem{teorema}{Teorema}
\newtheorem{prop}{Proprietà}
\newtheorem{corollario}{Corollario}


\title[nella teoria dei codici correttori]{Tesi di Laurea }
\author[La trasformata di Winograd]{Sebastiano Ferraris}
\subtitle{LA TRASFORMATA DI WINOGRAD \\ NELLA TEORIA DEI CODICI CORRETTORI}
\date{}
\institute{Università degli studi di Torino \\Facoltà di Scienze Matematiche Fisiche Naturali \\ Corso di laurea Magistrale in Matematica}

\begin{document}

%%%%%%%%%%%%%%%%%%%%%%%%%%%%%%%%%%%%%%%%%%5
%%%%%%%% inizio

%%%%%%%%%%%%%%%%%%%%%%%%%%%%%%%%%%
%%%TITOLO! %%%%%%%%%%%%%%%%%%%%%%%%%%
\thispagestyle{empty}
\begin{frame}
    \pagestyle{empty}
    \vspace{-1.5cm}
    \maketitle
    \vspace{-1cm}
    \begin{figure}[!h]
    \begin{center}
    \resizebox{2.0 cm}{2.0 cm}{\includegraphics{unito.jpg}}
    \end{center}
    \end{figure}
\end{frame}

%%%%%%%%%%%%%%%%%%%%%%%%%%%%%%%%%%%%%%%%%%%%%%%%%%%%%%%%%%%%%%%
%%% SECTION FATTORIZZAZIONE %%%%%%%%%%%%%%%%%%%%%%%%%%
\section{Fattorizzazione di $x^r-1$}

%%%%%%%%%%%%%%%%%%%%%%%%%%%%%%%%%%
%% DIAGRAMMA QUATTRO ALGEBRE ISOMORFE %%%%%%%%%%%%%%%%%%%%%%%%%%
\thispagestyle{empty}
\begin{frame}
   \begin{center}
   \begin{tikzpicture}[xscale=1.5,yscale=1.5]

\path node   (V) at (0,0) {$\mathcal{V}_{r, \mathbb{F}}^{c}$}         %(-1,0) node[anchor=east] {$e_1$ \ :}
      node   (M) at (4,0) {$ \mathcal{M}_{r,\mathbb{F} }^{c} $}
      node   (R) at (0,3) {$ \quotient{ \mathbb{F}[x] }{ (x^r -1 )}$} %(-0.9,3) node[anchor=east] {$\mathcal{R}_{r, \mathbb{F}}$ \ =}
      node   (A) at (4,3) {$\mathbb{F}C_{r}$} %(-1,1) node[anchor=east] {$e_1\times A$ \ :}
      ;
     { %[ thick]
      \draw[->]   (R) -- node[above]{$\psi_4$}     (A);
      \draw[->]   (V) -- node[above]{$\psi_1$}     (M);
      \draw[->]   (R) -- node[right]{$\psi_2$}     (V);
      \draw[->]   (A) -- node[right]{$\psi_3$}     (M);
      }
\end{tikzpicture}
   \end{center}
\end{frame}

%%%%%%%%%%%%%%%%%%%%%%%%%%%%%%%%%%%%%%%%%%%%%%%%%%%%%%%%%%%%%%%

%%% SHIFTERS DELLE QUATTRO ALGEBRE ISOMORFE 3step %%%%%%%%%%%%%%%%%%%%%%%%%%

% \thispagestyle{empty}
% \begin{frame}
%       \vspace{-1.5cm}
%       Prodotto di convoluzione dei vettori circolanti $\mathbf{a}, \mathbf{b}$:
%       \begin{align*}
%       (\mathbf{a} \star \mathbf{b})_{i} = \sum_{j \in \mathbb{Z}_{r} } a_{j} b_{i-j}
%       \qquad \forall i \in \mathbb{Z}_{r}
%       \end{align*}
%       \thispagestyle{empty}
%       \pause
%       Per $r=3$ equivale a
%       \begin{align*}
%       \mathbf{a} \star \mathbf{b} =
%       \left(
%       \begin{array} {c c c }
%       a_0 & a_1 & a_2
%       \end{array}
%       \right)
%       \left(
%       \begin{array} {c c c}
%       b_0 & b_1 & b_2   \\
%       b_2 & b_0 & b_1   \\
%       b_1 & b_2 & b_0
%       \end{array}
%       \right)
%       \end{align*}
%       \thispagestyle{empty}
%       \pause
%       Tutte le strutture possiedono uno Shifter:
%       \begin{center}
%       \begin{tabular}{ c | c c c c }
% 	  Strutture
% 	  &
% 	  $\mathcal{V}_{r, \mathbb{F}}^{c}$
% 	  &
% 	  $\mathcal{M}_{r,\mathbb{F} }^{c} $
% 	  &
% 	  $\mathbb{F}C_{r} $
% 	  &
% 	  $\quotient{ \mathbb{F}[x] }{ (x^r -1 )} $
% 	  \\
% 	  \hline
% 	  Shifter & $(0,1,0,\dots,0)$ & $s_n$ & $g$ & $x$
%       \end{tabular}
%       \end{center}
%       Generano il gruppo ciclico su cui le algebre sono definite e definiscono le trasformazioni.
% \end{frame}

%%%%%%%%%%%%%%%%%%%%%%%%%%%%%%%%%%%%%%%%%%%%%%%%%%%%%%%%%%%%%%%
%%%%%%%%%%%%%%%%%%%%%%%%%%%%%
\subsection{IL GRUPPO $Gal(\mathbb{F}(\xi), \mathbb{F})$}
\begin{frame}
\begin{itemize}
   \item Per ogni $\varphi \in Gal(\mathbb{F}(\xi), \mathbb{F})$, $\xi$ e $\varphi(\xi)$ hanno lo stesso
      polinomio minimo.
\end{itemize}

\vspace{0.5cm}

   $E^{(r)} = \lbrace \xi^{j}\rbrace_{j=0}^{r-1}\cong \mathbb{Z}_{r}
   \qquad \quad
   G \trianglelefteq \mathbb{Z}_{r}^{\star}
   \qquad  \quad
   G \cong Gal(\mathbb{F}(\xi), \mathbb{F})$
   \vspace{0.5cm}
   \begin{align*}
	Gal(\mathbb{F}(\xi), \mathbb{F}) \times E^{(r)}  & \longrightarrow  E^{(r)}   \\
		  (\varphi_{k},\xi^{l}) & \longmapsto \varphi_{k}(\xi^{l}) = \xi^{lk}
    \end{align*}
    \begin{align*}
	G \times \mathbb{Z}_{r} \longrightarrow  \mathbb{Z}_{r}   \\
		  (g,l) \longmapsto gl
    \end{align*}
\end{frame}

%%%%%%%%%%%%%%%%%%%%%%%%%%%%%%%%%%%%%%%%%%%%%%%%%%%%%%%%%%%%%%%
%%% ORBITE ED ETICHETTE  1 %%%%%%%%%%%%%%%%%%%%%%%%%%
\subsection{ORBITE ED ETICHETTE}
\begin{frame}
       Due elementi $l_{1}$ ed $l_{2}$ di $\mathbb{Z}_{r}$ sono nella stessa orbita della azione
	\begin{align*}
	G \times \mathbb{Z}_{r} &\longrightarrow  \mathbb{Z}_{r}   \\
		  (g,l) &\longmapsto gl
	\end{align*}
	se e solo se $\xi^{l_{1}}$ e $\xi^{l_{2}}$ hanno lo stesso polinomio minimo su $\mathbb{F}$.\\
        La {\bf classe ciclotomica} o {\bf $(r,\mathbb{F})$-orbita} di $t$ è definita come l'insieme
	\begin{align*}
	  O_{r,\mathbb{F}}(t) = O(t) = \lbrace gt \mod{r} \mid g \in G \rbrace \subseteq  \mathbb{Z}_{r}
	\end{align*}
	Diciamo {\bf etichetta} il più piccolo elemento di ogni orbita, e indichiamo l'{\bf insieme delle etichette} con
	\begin{align*}
	   \mathscr{L}_{r,\mathbb{F}} = \mathscr{L}
	\end{align*}
\end{frame}


%%%%%%%%%%%%%%%%%%%%%%%%%%%%%%%%%%%%%%%%%%%%%%%%%%%%%%%%%%%%%%%
%%% ORBITE ED ETICHETTE 2 %%%%%%%%%%%%%%%%%%%%%%%%%%
\begin{frame}
   \begin{center}
      La cardinalità dell'orbita di $t$
        \begin{align*}
	  m_{r,\mathbb{F}}(t) = m(t) = \arrowvert O(t) \arrowvert
	\end{align*}
      coincide con il grado del polinomio minimo di \\
     $\xi^{t}$
     \vspace{0.5cm}
     \\
      La cardinalità dell'insieme delle etichette
 	\begin{align*}
	   l_{r,\mathbb{F}} = l = \arrowvert \mathscr{L} \arrowvert
	\end{align*}
     coincide con il numero di fattori irriducibili di \\
     $x^r-1$
    \end{center}
\end{frame}

%%%%%%%%%%%%%%%%%%%%%%%%%%%%%%%%%%%%%%%%%%%%%%%%%%%%%%%%%%%%%%%
%%% TEOREMA DI FATTORIZZAZIONE %%%%%%%%%%%%%%%%%%%%%%%%%%
\subsection{TRASFORMATA DI WINOGRAD}
\begin{frame}
    \begin{teorema}
      Sia $r$ intero positivo ed $\mathbb{F}$ campo perfetto.
      \begin{enumerate}
	  \item Ad ogni orbita $O(v)$ corrisponde un polinomio irriducibile in $\mathbb{F}[x]$ definito da
		\begin{align*}
		  M^{(v)}(x) =  \prod_{t \in O(v)} (x- \xi^t)
		\end{align*}
	  \item La decomposizione in $\mathbb{F}$ di $x^r - 1$ in fattori irriducibili è data da
		\begin{align*}
		  x^r - 1 = \prod_{v \in \mathscr{L} } M^{(v)}(x)
		\end{align*}
      \end{enumerate}
    \end{teorema}
\end{frame}


% %%%%%%%%%%%%%%%%%%%%%%%%%%%%%%%%%%%%%%%%%%%%%%%%%%%%%%%%%%%%%%%
% %%% FAttorizzazione di R come prodotto di campi %%%%%%%%%%%%%%%%%%%%%%%%%%
% \subsection{CARDINALITA' DI  $\mathscr{L}$ }
% \begin{frame}
%    \begin{teorema}
% 	$\mathbb{F}$ campo perfetto, $r$ intero positivo coprimo con $p$,
% 	\begin{align*}
% 	  l_{r,\mathbb{F}} = \frac{1}{\arrowvert G \arrowvert }\sum_{g \in G} (g-1, r)
% 	\end{align*}
%     \end{teorema}
%     \begin{proof}
% 	\begin{align*}
% 	  l = \frac{1}{\arrowvert G \arrowvert }\sum_{g \in G}\arrowvert X_{g} \arrowvert \qquad \quad
%  	   X_{g} = \lbrace t \in \mathbb{Z}_{r} \mid gt = t \mod r \rbrace
%  	   \\
%  	   \arrowvert X_{g} \arrowvert = \arrowvert \lbrace t \mid  gt \equiv t \mod{r} \rbrace = (g-1, r)
% 	\end{align*}
% 	%$\arrowvert X_{g} \arrowvert = \arrowvert \lbrace t \mid  gt \equiv t \mod{r} \rbrace = (g-1, r) $
%     \end{proof}
% \end{frame}


%%%%%%%%%%%%%%%%%%%%%%%%%%%%%%%%%%%%%%%%%%%%%%%%%%%%%%%%%%%%%%%
%%% FAttorizzazione di R 1 %%%%%%%%%%%%%%%%%%%%%%%%%%
%\subsection{FATTORIZZAZIONE DI $\mathcal{R}_{r,\mathbb{F}}$ }
\begin{frame}
      \begin{teorema}
        $\mathbb{F}$ campo perfetto, $r$ intero positivo ed
	\begin{align*}
	      x^r-1 = \prod_{v\in \mathscr{L}} M^{(v)}(x)
	\end{align*}
       Allora $\gamma$, {\bf trasformata di Winograd},  è un isomorfismo di algebre
	\begin{align*}
	\gamma :  \quotient{\mathbb{F} \lbrack x \rbrack  }{x^{r} - 1}
		  & \longrightarrow
		  \prod_{v\in \mathscr{L}} \quotient{\mathbb{F} \lbrack x \rbrack  }{M^{(v)}(x)}   \\
		  %%
		  a(x) &\longmapsto  ( a(x)\mod{M^{(v)}(x)})_{v\in \mathscr{L}}
	\end{align*}
      \end{teorema}
\end{frame}

%%%%%%%%%%%%%%%%%%%%%%%%%%%%%%%%%%%%%%%%%%%%%%%%%%%%%%%%%%%%%%%
%%% FAttorizzazione di R 2 %%%%%%%%%%%%%%%%%%%%%%%%%%
\thispagestyle{empty}
\begin{frame}
   \begin{proof}
	\begin{center}
	 \begin{tikzpicture}[xscale=1,yscale=1]
	  \path node   (Dom) at (0,2.8) {$\quotient{\mathbb{F} \lbrack x \rbrack  }{ (x^r - 1)}$}
		node   (Im) at (4,0) {$  \prod_{v\in \mathscr{L}} \quotient{\mathbb{F} \lbrack x \rbrack  }{M^{(v)}(x)} $}
		node   (Quoz) at (6,2.8) {$ \quotient{\quotient{\mathbb{F} \lbrack x \rbrack  }{(x^r-1)} }{ \prod_{v\in \mathscr{L}} M^{(v)}(x)}$}
		;
	      { %[ thick]
		\draw[->]   (Dom)  -- node[above]{$\gamma$}     (Im);
		\draw[->]   (Dom)  -- node[above]{$\pi$}        (Quoz);
		\draw[->]   (Quoz) -- node[right]{$\sim$}       (Im);
		}
	 \end{tikzpicture}
         \end{center}
   \end{proof}
\end{frame}

%%%%%%%%%%%%%%%%%%%%%%%%%%%%%%%%%%%%%%%%%%%%%%%%%%%%%%%%%%%%%%%
%%% Esempio %
\thispagestyle{empty}
\begin{frame}
   \vspace{-1.2cm}
   Ad esempio:
   \begin{itemize}
      \item   $r = 7 , \quad \mathbb{F} = \mathbb{Q}, \quad G = \mathbb{Z}_{7}^{\star}$
	  \begin{align*}
	      O(0)&= \lbrace 0 \rbrace \\
	      O(1)&= \lbrace 1,2,3,4,5,6 \rbrace
	  \end{align*}
	    \begin{align*}
	      x^{7} - 1 &= M^{(0)}(x) M^{(1)}(x) =  \Phi_{1}(x) \Phi_{7}(x)  \\
			&= (x-1)(x^6 +x^5 + x^4 + x^3 +x^2 + x +1)
	    \end{align*}
      \item  $r = 7 , \quad \mathbb{F} = GF(2)  \quad G =\lbrace 1,2,4 \rbrace \triangleleft \mathbb{Z}_{7}^{\star}$
	  \begin{align*}
	      O(0) &= \lbrace 0 \rbrace \\
	      O(1) &= \lbrace 1,2,4 \rbrace \\
	      O(3) &= \lbrace 3,5,6 \rbrace
	  \end{align*}
	  \begin{align*}
	      x^{7} - 1 &= M^{(0)}(x) M^{(1)}(x) M^{(3)}(x) \\
			&= (x-1)(x^3 + x + 1)(x^3 + x^2 + 1)
	    \end{align*}
   \end{itemize}
\end{frame}

%%%%%%%%%%%%%%%%%%%%%%%%%%%%%%%%%%%%%%%%%%%%%%%%%%%%%%%%%%%%%%%
%%% Esempio %
\thispagestyle{empty}
\begin{frame}
   \vspace{-1.5cm}
        \begin{center}
         $x^7 - 1$ si scompone in $ \mathbb{Q} $ in due fattori irriducibili
	\begin{align*}
	  \quotient{\mathbb{Q} \lbrack x \rbrack  }{(x^{7} - 1)}
	  \cong
	  \quotient{\mathbb{Q} \lbrack x \rbrack  }{\Phi_{1}(x)}
	  \times
	  \quotient{\mathbb{Q} \lbrack x \rbrack  }{\Phi_{7}(x)}
	\end{align*}
	\vspace{1cm}

	 $x^7 - 1$ si scompone in $ \mathbb{Z}_{2}$ in tre fattori irriducibili
	\begin{align*}
	  \quotient{\mathbb{Z}_{2} \lbrack x \rbrack  }{(x^{7} - 1)}
	  \cong
	  \quotient{\mathbb{Z}_{2} \lbrack x \rbrack  }{M^{(0)}(x)}
	  \times
	  \quotient{\mathbb{Z}_{2} \lbrack x \rbrack  }{M^{(1)}(x)}
	  \times
	  \quotient{\mathbb{Z}_{2} \lbrack x \rbrack  }{M^{(3)}(x)}
	\end{align*}
	\end{center}
\end{frame}

%%%%%%%%%%%%%%%%%%%%%%%%%%%%%%%%%%%%%%%%%%%%%%%%%%%%%%%%%%%%%%%
%%% slide presentazione Q  1 %%%%%%%%%%%%%%%%%%%%%%%%%%
\begin{frame}
   \begin{lemma}
   \begin{align*}
      \forall v \in \mathscr{L}
      \qquad
      \mathbb{F}(\xi^{v})
      \cong
      \quotient{\mathbb{F} \lbrack x \rbrack  }{M^{(v)}(x)}
   \end{align*}
  \end{lemma}
  \begin{corollario}
     \begin{align*}
        \quotient{\mathbb{F} \lbrack x \rbrack  }{ (x^{r} - 1)}
         \cong
         \prod_{v\in \mathscr{L}} \quotient{\mathbb{F} \lbrack x \rbrack  }{M^{(v)}(x)}
         \cong
         \prod_{v\in \mathscr{L}} \mathbb{F}(\xi^{v})
     \end{align*}
  \end{corollario}

\end{frame}

%%%%%%%%%%%%%%%%%%%%%%%%%%%%%%
%diagramma finale capitolo 1
%%%%%%%%%%%%%%%%%%%%%%%%%%%%%%
\thispagestyle{empty}
\begin{frame}
   \vspace{-1.5cm}
    \begin{center}
    \begin{tikzpicture}[xscale=1,yscale=1]
    \path node   (Q) at (-5,3) {}
	  node   (P) at (0,6.5) {}
	  node   (R) at (0,3) {$\quotient{\mathbb{F} \lbrack x \rbrack  }{ (x^{r} - 1)} $}
	  node   (V) at (0,0) {$\mathcal{V}_{r, \mathbb{F}}^{c}$}         %(-1,0) node[anchor=east] {$e_1$ \ :}
          node   (M) at (4,0) {$ \mathcal{M}_{r,\mathbb{F} }^{c} $}
          node   (A) at (4,3) {$\mathbb{F}C_{r}$}
	  ;
	{
	  \draw[->]   (R) -- node[above]{$\psi_4$}     (A);
	  \draw[->]   (V) -- node[above]{$\psi_1$}     (M);
	  \draw[->]   (R) -- node[right]{$\psi_2$}     (V);
	  \draw[->]   (A) -- node[right]{$\psi_3$}     (M);
        }
    \end{tikzpicture}
  \end{center}
\end{frame}


%%%%%%%%%%%%%%%%%%%%%%%%%%%%%%
%diagramma finale capitolo 1 seconda parte
%%%%%%%%%%%%%%%%%%%%%%%%%%%%%%
\thispagestyle{empty}
\begin{frame}
   \vspace{-1.5cm}
    \begin{center}
    \begin{tikzpicture}[xscale=1,yscale=1]
    \path node   (Q) at (-4.5,3) {$  \prod_{v\in \mathscr{L}} \mathbb{F}(\xi^{v})$}
	  node   (P) at (0,6) {$ \prod_{v\in \mathscr{L}} \quotient{\mathbb{F} \lbrack x \rbrack  }{M^{(v)}(x)}$}
	  node   (R) at (0,3) {$\quotient{\mathbb{F} \lbrack x \rbrack  }{ (x^{r} - 1)} $}
	  node   (V) at (0,0) {$\mathcal{V}_{r, \mathbb{F}}^{c}$}         %(-1,0) node[anchor=east] {$e_1$ \ :}
          node   (M) at (4,0) {$ \mathcal{M}_{r,\mathbb{F} }^{c} $}
          node   (A) at (4,3) {$\mathbb{F}C_{r}$}
	  ;
	{ [ thick]
	  \draw[->]          (R)  -- node[right]{$\gamma$}     (P);
	  \draw[->]          (R)  -- node[above]{$\eta$}       (Q);
	  \draw[->]          (Q)  -- node[above]{$\mu$}        (P);
	}
	{
	  \draw[->]   (R) -- node[above]{$\psi_4$}     (A);
	  \draw[->]   (V) -- node[above]{$\psi_1$}     (M);
	  \draw[->]   (R) -- node[right]{$\psi_2$}     (V);
	  \draw[->]   (A) -- node[right]{$\psi_3$}     (M);
        }
    \end{tikzpicture}
  \end{center}
\end{frame}



%%%%%%%%%%%%%%%%%%%%%%%%%%%%%%%%%%%%%%%%%%%%%%%%%%%%%%%%%%%%%%%
%%%%%%%%%%%%%%%%%%%%%%%%%%%%%%%%%%%%%%%%%%%%%%%%%%%%%%%
%%%%%%%%%%%%%%%%%%%%%%%%%%%%%%%%%%%%%%%%%%%%%%%%%%%%%%%%%%%%%%%
%%% SECTION IDEALI E IDEMPOTENTI %%%%%%%%%%%%%%%%%%%%%%%%%%
\section{Ideali e Idempotenti}
\subsection{COSTRUZIONE DEGLI IDEALI}
%%%%%%%%%%%%%%%%%%%%%%%%%%%%%%%%%%%%%%%%%%%%%%%%%%%%%%%%%%%%%%%
%%% IDEALI DI R %%%%%%%%%%%%%%%%%%%%%%%%%%
\begin{frame}
  \frametitle{Ideali}
   \begin{teorema}
   \begin{center}
      \begin{align*}
	  \mathfrak{a}\phantom{a} \unlhd \phantom{a} \quotient{\mathbb{F} \lbrack x \rbrack  }{ (x^{r} - 1)} \\
	  \end{align*}
	  se e solo se
	  \begin{align*}
	      \exists ! \phantom{a}  a(x),   \phantom{aa}
	      a(x) \mid x^r - 1 \phantom{aa} monico, \phantom{aa} \mathfrak{a} = (a(x))
	\end{align*}
   \end{center}
   \end{teorema}
\end{frame}

%%%%%%%%%%%%%%%%%%%%%%%%%%%%%%%%%%%%%%%%%%%%%%%%%%%%%%%%%%%%%%%
%%% IDEALI DI R %%%%%%%%%%%%%%%%%%%%%%%%%%
\begin{frame}
   \begin{teorema}
   \begin{center}
   \begin{tikzpicture}[xscale=1,yscale=0.7]
      \path node   (A) at (0,2) {$ \lbrace A \mid A \subseteq \mathscr{L} \rbrace $}
	    node   (O) at (4,4) {$ \lbrace \cup_{t \in A} O(t)  \mid A \subseteq \mathscr{L} \rbrace $}
	    node   (R) at (4,0) {$ \lbrace \mathfrak{a} \mid \mathfrak{a} \trianglelefteq  \quotient{\mathbb{F} \lbrack x \rbrack  }{ (x^{r} - 1)}  \rbrace $}
	    ;
	  { %[ thick]
	    \draw[->]   (A) -- node{}     (O);
	    \draw[->]   (A) -- node{}     (R);
	    \draw[->]   (O) -- node{}     (R);
	    }
    \end{tikzpicture}
   \end{center}
   sono biiezioni.
   \end{teorema}
\end{frame}

%%%%%%%%%%%%%%%%%%%%%%%%%%%%%%%%%%%%%%%%%%%%%%%%%%%%%%%%%%%%%%%
%%% ESEMPIO IDEALI %%%%%%%%%%%%%%%%%%%%%%%%%%
\thispagestyle{empty}
\begin{frame}
   \vspace{-1.3cm}
     Ad esempio $r = 9$, $q = 2$: $ G = \mathbb{Z}_{9}^{\star} $, $\mathscr{L} = \lbrace 0,1,3 \rbrace$:
      \begin{align*}
	  M^{(0)}(x) &= x+1 \\
	  M^{(1)}(x) &= x^6+x^3+1 \\
	  M^{(3)}(x) &= x^2+x+1
      \end{align*}
      \begin{align*}
	  &\mathfrak{a}_{\lbrace 0 \rbrace} = (M^{(0)}(x)) = (x-1) \\
	  &\mathfrak{a}_{\lbrace 1 \rbrace} = (M^{(1)}(x)) = (x^6+x^3+1) \\
	  &\mathfrak{a}_{\lbrace 3 \rbrace} = (M^{(3)}(x)) = (x^2+x+1) \\
	  &\mathfrak{a}_{\lbrace 0,1 \rbrace} = (M^{(0)}(x)M^{(1)}(x)) = (x^3 -1) \\
	  &\mathfrak{a}_{\lbrace 0,3 \rbrace} = (M^{(0)}(x)M^{(3)}(x)) = (x^7 + x^6 + x^4 + x^3 + x + 1) \\
	  &\mathfrak{a}_{\lbrace 1,3 \rbrace}
			= (M^{(1)}(x)M^{(3)}(x)) = (x^8 + x^7 + \dots + x + 1) \\
	  &\mathfrak{a}_{\lbrace 0,1,3 \rbrace} = (M^{(0)}(x)M^{(1)}(x)M^{(3)}(x)) = (x^9 -1) = (0) \\
	  &\mathfrak{a}_{\emptyset} = (1)
      \end{align*}
\end{frame}


\subsection{COSTRUZIONE DEGLI IDEMPOTENTI}
%%%%%%%%%%%%%%%%%%%%%%%%%%%%%%%%%%%%%%%%%%%%%%%%%%%%%%%%%%%%%%%
%%% IDEMPOTENTI %%%%%%%%%%%%%%%%%%%%%%%%%%
\begin{frame}
  \frametitle{Idempotenti}
  \begin{definizione}
     $ $  %{\bf idempotente} se
     \begin{align*}
        a(x) \in \quotient{\mathbb{F} \lbrack x \rbrack  }{ (x^{r} - 1)} \qquad \qquad a(x)^2 = a(x)
     \end{align*}
  \end{definizione}
  \pause
    \begin{prop}
      In $\prod_{v\in \mathscr{L}} \quotient{\mathbb{F} \lbrack x \rbrack  }{M^{(v)}(x)}$
      %$  \prod_{v\in \mathscr{L}} \mathbb{F}(\xi^{v})$
      sono tutti e soli i vettori $l$-dimensionali di polinomi costituiti da $1$ e da $0$.
    \end{prop}
    {\bf Idempotente minimale}: $  \mathbf{e}_{j} = (0, \dots, 0,1,0, \dots, 0)$ \\
    {\bf Idempotente massimale}: $  \mathbf{\hat{e}}_{j} = (1, \dots, 1,0,1, \dots, 1)$
\end{frame}

%%%%%%%%%%%%%%%%%%%%%%%%%%%%%%%%%%%%%%%%%%%%%%%%%%%%%%%%%%%%%%%
%%% TEROREMI %%%%%%%%%%%%%%%%%%%%%%%%%%
\begin{frame}
\begin{teorema}
   \begin{enumerate}
      \item L'intero $l$, cardinalità dei fattori di $x^r-1$ in $\mathbb{F}_{q}$ coincide con il numero degli idempotenti minimali e con il numero degli idempotenti massimali.
      \item Gli idempotenti sono ortogonali: $\mathbf{e}_{i}\mathbf{e}_{j}  = \mathfrak{0}$ per $i \neq j$.
      \item Gli idempotenti decompongono l'unità: $\sum_{i \in \mathscr{L}}\mathbf{e}_{i}  = \mathfrak{1}$.
      \item Le combinazioni di idempotenti generano tutti gli ideali.
      \item Ogni elemento di $ \prod_{v\in \mathscr{L}} \mathbb{F}(\xi^{v})$ si decompone come combinazione lineare a coefficienti in $\mathbb{F}_{q}$ degli idempotenti minimali.
      \item $(\mathbf{e}_{j})$ è un ideale minimale, $(\mathbf{\hat{e}}_{j})$ un ideale massimale.
   \end{enumerate}
\end{teorema}
\end{frame}


%%%%%%%%%%%%%%%%%%%%%%%%%%%%%%%%%%%%%%%%%%%%%%%%%%%%%%%%%%%%%%%
%%% SECTION TRASFORMATA DI WINOGRAD %%%%%%%%%%%%%%%%%%%%%%%%%%
%%%%%%%%%%%%%%%%%%%%%%%%%%%%%%%%%%%%%%%%%%%%%%%%%%%%%%%%%%%%%%%
%%%%%%%%%%%%%%%%%%%%%%%%%%%%%%%%%%%%%%%%%%%%%%%%%%%%%%%%%%%%%%%
\section{Trasformata di Winograd}

%%%%%%%%%%%%%%%%%%%%%%%%%%%%%%
%% esempio 1
\thispagestyle{empty}
\begin{frame}
   \vspace{-1.2cm}
    Le immagini tramite $\gamma$ degli elementi della base di $ \quotient{\mathbb{F}_{2} \lbrack x \rbrack  }{ (x^{7} - 1)}$ sono
    \begin{align*}
      \gamma(x^0) &= (1,1,1)  \\
      \gamma(x^1) &= (1,x,x) \\
      \gamma(x^2) &= (1,x^2,x^2)  \\
      \gamma(x^3) &= (1,1+x^2,1+x+x^2)  \\
      \gamma(x^4) &= (1,1+x,x+x^2)  \\
      \gamma(x^5) &= (1,1+x+x^2,1+x^2)  \\
      \gamma(x^6) &= (1,x+x^2,1+x)
    \end{align*}
    per
    \begin{align*}
      x^{7} + 1 &= M^{(0)}(x) M^{(1)}(x) M^{(3)}(x) \\
		&= (x+1)(x^3 + x + 1)(x^3 + x^2 + 1)
    \end{align*}
\end{frame}

%%%%%%%%%%%%%%%%%%%%%%%%%%%%%%
%% esempio 2
\thispagestyle{empty}
\begin{frame}
   \vspace{-1.2cm}
    Le immagini tramite $\gamma$ degli elementi della base di $ \quotient{\mathbb{F}_{2} \lbrack x \rbrack  }{ (x^{7} - 1)}$ sono
    \begin{align*}
	\gamma(1,0,0,0,0,0,0)  = \gamma(x^0) &= (1,1,1) =  (1|1,0,0|1,0,0)          \\
	\gamma(0,1,0,0,0,0,0)  = \gamma(x^1) &= (1,x,x) = (1|0,1,0|0,1,0)            \\
	\gamma(0,0,1,0,0,0,0)  = \gamma(x^2) &= (1,x^2,x^2) = (1|0,0,1|0,0,1)        \\
	\gamma(0,0,0,1,0,0,0)  = \gamma(x^3) &= (1,1+x^2,1+x+x^2) = (1|1,0,1|1,1,1)  \\
	\gamma(0,0,0,0,1,0,0)  = \gamma(x^4) &= (1,1+x,x+x^2) = (1|1,1,0|0,1,1)      \\
	\gamma(0,0,0,0,0,1,0)  = \gamma(x^5) &= (1,1+x+x^2,1+x^2) = (1|1,1,1|1,0,1)  \\
	\gamma(0,0,0,0,0,0,1)  = \gamma(x^6) &= (1,x+x^2,1+x) = (1|0,1,1|1,1,0)
    \end{align*}
    per
    \begin{align*}
      x^{7} + 1 &= M^{(0)}(x) M^{(1)}(x) M^{(3)}(x) \\
		&= (x+1)(x^3 + x + 1)(x^3 + x^2 + 1)
    \end{align*}
\end{frame}


%%%%%%%%%%%%%%%%%%%%%%%%%%%%%%%%%%%%%%%%%%%%%%%%%%%%%%%%%%%%%%%
%%% VDEFINIZIONE FONDAMENTALE %%%%%%%%%%%%%%%%%%%%%%%%%%
\begin{frame}
    \begin{definizione}
      La matrice della trasformazione $\gamma$ fra le algebre
      \begin{align*}
         \quotient{\mathbb{F} \lbrack x \rbrack  }{ (x^{r} - 1)}
         \quad
         e
         \quad
         \prod_{v\in \mathscr{L}} \quotient{\mathbb{F} \lbrack x \rbrack  }{M^{(v)}(x)}
      \end{align*}
      nelle rispettive rappresentazioni vettoriali
      \begin{align*}
         \mathcal{V}_{r, q}^{c} \quad e \quad \mathcal{V}_{r,q}^{\mathscr{L}}
      \end{align*}
      è detta {\bf trasformata di Winograd} \\
      È indicata con $\Gamma$, e la sua inversa è indicata con $\Delta$.
    \end{definizione}
\end{frame}



%%%%%%%%%%%%%%%%%%%%%%%%%%%%%%
%% esempio 3
\thispagestyle{empty}
\begin{frame}
   \vspace{-1.2cm}
      \begin{align*}
	  \Gamma =
	  \left(
	    \begin{array} { c }
	    \Gamma^{(0)}  \\ \\
	    \Gamma^{(1)} \\ \\
	    \Gamma^{(3)}
	    \end{array}
	  \right)
	  =
	  \left(
	    \begin{array} { c c c c c c c c}
	    1 & 1 & 1 & 1 & 1 & 1 & 1  \\
	    \hline
	    1 & 0 & 0 & 1 & 1 & 1 & 0  \\
	    0 & 1 & 0 & 0 & 1 & 1 & 1  \\
	    0 & 0 & 1 & 1 & 0 & 1 & 1  \\
	    \hline
	    1 & 0 & 0 & 1 & 0 & 1 & 1  \\
	    0 & 1 & 0 & 1 & 1 & 0 & 1  \\
	    0 & 0 & 1 & 1 & 1 & 1 & 0
	    \end{array}
	  \right)
      \end{align*}
      \begin{align*}
	  \Delta =
	  \left(
	  \begin{array} { c | c | c }
	  \Delta^{(0)} & \Delta^{(1)} & \Delta^{(3)}
	  \end{array}
	  \right)
	  =
	  \left(
	  \begin{array} { c | c c c | c c c c}
	  1 & 1 & 1 & 1 & 1 & 0 & 0  \\
	  1 & 0 & 1 & 1 & 1 & 1 & 0  \\
	  1 & 0 & 0 & 1 & 1 & 1 & 1  \\
	  1 & 1 & 0 & 0 & 0 & 1 & 1  \\
	  1 & 0 & 1 & 0 & 1 & 0 & 1  \\
	  1 & 1 & 0 & 1 & 0 & 1 & 0  \\
	  1 & 1 & 1 & 0 & 0 & 0 & 1
	  \end{array}
	  \right)
      \end{align*}
\end{frame}

%%%%%%%%%%%%%%%%%%%%%%%%%%%%%%%%%%%%%%%%%%%%%%%%%%%%%%%%%%%%%%%
%%% PROPRIETà GAMMA %%%%%%%%%%%%%%%%%%%%%%%%%%
%
% \begin{frame}
%     \begin{lemmax}
%       Il divisore $M^{(v)}(x)$ soddisfa l'eqazione
%       \begin{align*}
% 	  M^{(v)}(x) = x^{m(v)} - \sum_{j=1}^{m(v)} \Gamma_{m(v) - j, m(v)}^{(v)}  x^{m(v) - j}
%       \end{align*}
%     \end{lemmax}
%     \begin{teorema} \label{teo:winogradRicorrenza}
% 	Sia $v$ appartenente all'insieme delle etichette $\mathscr{L}_{r,q}$, allora per ogni $i \in \lbrace  0,1, \dots, m(v) - 1 \rbrace$ segue che
% 	\begin{align*} \label{eq:winogradRicorrenza}
% 	    \Gamma_{i,j}^{(v)} &= \delta_{i,j}  \qquad j \in \lbrace  0,1, \dots, m(v) - 1 \rbrace \\
% 	    \Gamma_{i,n}^{(v)} &= \sum_{k=1}^{m(v)} \Gamma_{m(v) - k, m(v)}^{(v)} \Gamma_{i,n-k}^{(v)} \qquad \forall n \in \mathbb{Z}
% 	  \end{align*}
%       \end{teorema}
% \end{frame}

%%%%%%%%%%%%%%%%%%%%%%%%%%%%%%%%%%%%%%%%%%%%%%%%%%%%%%%%%%%%%%%
%%% PROPRIETà GAMMA %%%%%%%%%%%%%%%%%%%%%%%%%%
\subsection{PROPRIETA' DI $\Gamma$}
\begin{frame}
    \begin{lemmax}
    \begin{center}
      $c(x) \in (M^{(v)}(x))\quad $ se e solo se $ \quad \Gamma^{(v)} \mathbf{c}^{t} = 0$.
    \end{center}
    \end{lemmax}
    \begin{proof}
      Dato che
      \begin{enumerate}
	  \item $\eta^{(v)}(x^{j}) = \xi^{vj} = \sum_{i=0}^{m(v)-1}\Gamma_{i,j}^{(v)}\xi^{iv}$ per $j \in \mathbb{Z}_{r}$.
	  \item Per $H^{(v)}$ matrice dell'epimorfismo $\eta^{(v)}$ si ha $\eta^{(v)}m(x) = H^{(v)}\mathbf{m}^{t}$.
	  \item $H=\Gamma$ per $H$ e $\Gamma$ matrici di trasformazione di $\eta$ e $\gamma$.
      \end{enumerate}
      allora vale la catena di biimplicazioni:\\
      $m(x)\in (M^{(v)}(x))$ $\iff$ $M^{(v)}(x)$ divide $m(x)$ $\iff$ $m(\xi^{v}) = 0$ $\iff$ $\eta^{(v)}(m(x)) = H^{(v)}\mathbf{m}^{t} = 0$ $\iff$ $ \Gamma^{(v)}\mathbf{m}^{t} = 0$.
    \end{proof}

\end{frame}


%%%%%%%%%%%%%%%%%%%%%%%%%%%%%%%%%%%%%%%%%%%%%%%%%%%%%%%%%%%%%%%
%%% PROPRIETà DELTA %%%%%%%%%%%%%%%%%%%%%%%%%%

% \begin{frame}
%     Come determinare l'inversa di $\gamma$?
%     \begin{align*}
%       \gamma^{-1} :  \prod_{v\in \mathscr{L}} \quotient{\mathbb{F} \lbrack x \rbrack  }{M^{(v)}(x)}
%       & \longrightarrow
%       \quotient{\mathbb{F} \lbrack x \rbrack  }{ (x^{r} - 1)}
%       \\
%       %%
%       (a_{v}(x) \mod{M^{(v)}(x)})_{v \in \mathscr{L}  }
%       &\longmapsto
%       a(x)
%     \end{align*}
%     $a(x)$ è il risultato del sistema di congruenze $a(x) \equiv a_{v}(x) \mod{M^{(v)} (x) }$ ottenuto con il teorema cinese dei resti.
% \end{frame}

%%%%%%%%%%%%%%%%%%%%%%%%%%%%%%%%%%%%%%%%%%%%%%%%%%%%%%%%%%%%%%%
%%% PROPRIETà DELTA %%%%%%%%%%%%%%%%%%%%%%%%%%
\subsection{PROPRIETA' DI $\Delta$}
\begin{frame}
    \begin{prop}
    \begin{enumerate}
      \item  La prima colonna di ogni blocco $\Delta^{(v)}$ di $\Delta$ è l'idempotente miminale $\mathbf{e}_{v}$ che genera l'ideale minimale
      \begin{align*}
	  (\hat{M}^{(v)}(x)) = \Big(\frac{1-x^r}{M^{(v)}(x)} \Big)
      \end{align*}
      \item L'insieme $\lbrace \Delta_{\sim, 0}^{(v)} \rbrace_{v\in \mathscr{L}}$ delle prime colonne di tutti i blocchi di $\Delta$ costituisce l'insieme di tutti gli idempotenti minimali che generano tutti gli ideali minimali.
      \item  $v \in \mathscr{L}$, $j \in \lbrace 0, 1 \dots , m(v)-1 \rbrace$:
      \begin{align*}
      \big( \Delta_{\sim, j}^{(v)} \big)^{t} \in \mathcal{V}_{m(v), q}^{c}
      \qquad \qquad
      \big( \Delta_{\sim, j}^{(v)} \big)^{t}  =  (0,1,0,\dots,0)^{j} \star \big( \Delta_{\sim, 0}^{(v)} \big)^{t}
      \end{align*}
    \end{enumerate}
    \end{prop}
\end{frame}

%%%%%%%%%%%%%%%%%%%%%%%%%%%%%%%%%%%%%%%%%%%%%%%%%%%%%%%%%%%%%%%
%%% PROPRIETà DELTA %%%%%%%%%%%%%%%%%%%%%%%%%%
% \begin{frame}
%     \begin{prop}
%       L'insieme dei vettori colonna $\lbrace \Delta_{\sim, j}^{(v)} \rbrace_{j=0}^{m(v)-1}$ determina una base dell'ideale $(M^{(v)}(x))$.
%     \end{prop}
%     \begin{prop}
% 	Esiste un altro modo per ricavare $\Delta$, senza utilizzare il teorema cinese dei resti:
% 	\begin{align*}
% 	  \Delta_{i,j} = \frac{1}{r} \sum_{k=1}^{m(v)-1} \Gamma_{k,j-i+k}^{(v)}
% 	\end{align*}
%     \end{prop}
% \end{frame}


%%%%%%%%%%%%%%%%%%%%%%%%%%%%%%%%%%%%%%%%%%%%%%%%%%%%%%%%%%%%%%%
%%% SECTION APPLICAZIONI CODICI CORRETTORI %%%%%%%%%%%%%%%%%%%%%%%%%%
%%%%%%%%%%%%%%%%%%%%%%%%%%%%%%%%%%%%%%%%%%%%%%%%%%%%%%%%%%%%%%%
%%%%%%%%%%%%%%%%%%%%%%%%%%%%%%%%%%%%%%%%%%%%%%%%%%%%%%%%%%%%%%%

\thispagestyle{empty}
\begin{frame}
   \begin{center}
      Applicazioni della trasformata di Winograd \\
      nella\\
      teoria dei codici correttori
   \end{center}
\end{frame}

%%%%%%%%%%%%%%%%%%%%%%%%%%%%%%%%%%%%%%%5
%%%%%%%%%%%%%%%%%%%%%%%%%%%%%%%%%%%%%%%5
\section{Codici correttori}
\begin{frame}
  \frametitle{Coici Ciclici}
    \begin{definizione}
    Un codice lineare $C$ di lunghezza $r$ (cioè sottospazio vettoriale di $\mathbb{F}^{r}$) si dice {\bf ciclico} se è chiuso rispetto alla permutazione ciclica dei suoi elementi verso destra:
    \begin{align*}
	\mathbf{c} = (c_{0},c_{1}, \dots , c_{r-1}) \in C \Longrightarrow (c_{r-1},c_{0}, \dots , c_{r-2}) \in C
    \end{align*}
    \end{definizione}
    \begin{teorema}
      Un codice lineare $C$ di lunghezza $r$ sull'alfabeto $\mathbb{F}_{q}$ è ciclico se e solo se è un ideale di $\quotient{\mathbb{F} \lbrack x \rbrack  }{ (x^{r} - 1)}$.
    \end{teorema}
\end{frame}

%a voce cosa è il polinomio generatore

%%%%%%%%%%%%%%%%%%%%%%%%%%%%%%%%%%%%%%%%%%%%%%%%%%%%%%%%%%%%%%%
%%%%%%%%%%%%%%%%%%%%%%%%%%%%%%%%%%%%%%%%%%%%%%%%%%%%%%%%%%%%%%%
\begin{frame}
  La matrice di trasformazione fra lo spazio $\mathbb{F}^{r}$ ed il codice ciclico $(a(x))$ è chiamata {\bf matrice generatrice} ed è indicata con $G$.
  La matrice di trasformazione fra lo spazio $\mathbb{F}^{r}$ ed il codice duale $(a(x))^{\perp} = (\hat{a}(x)^{\perp})$ è chiamata {\bf matrice di controllo} ed è indicata con $H$.
  \begin{teorema}
     Sia $\mathfrak{a} $ codice ciclico
     \begin{align*}
	c(x) \in \mathfrak{a} &\iff H \mathbf{c}^{t} = \mathbf{0}^{t}  \\
	c(x) \in \mathfrak{a}^{\perp} &\iff G \mathbf{c}^{t} = \mathbf{0}^{t}
      \end{align*}
  \end{teorema}
\end{frame}


%%%%%%%%%%%%%%%%%%%%%%%%%%%%%%%%%%%%%%%5
%%%%%%%%%%%%% APPLICAZIONI
\section{Applicazioni}
\subsection{$\Gamma$ COME MATRICE GENERATRICE E DI CONTROLLO}
\begin{frame}
  Consideriamo i risultati:
      \begin{align*}
      c(x) \in \mathfrak{a} &\iff H \mathbf{c}^{t} = \mathbf{0}^{t}  \\
      c(x) \in \mathfrak{a}^{\perp} &\iff G \mathbf{c}^{t} = \mathbf{0}^{t} \\
      c(x) \in (M^{(v)}(x)) &\iff \Gamma^{(v)} \mathbf{c}^{t} = \mathbf{0}^{t}
    \end{align*}
   \begin{teorema}
      Sia $v \in \mathscr{L}$, allora il codice ciclico massimale $\mathfrak{a} = (M^{(v)}(x))$ ha come matrice di controllo $\Gamma^{(v)} $, $v$-esimo blocco della trasformata di Winograd.
    \end{teorema}
    \begin{teorema}
      Sia $v \in \mathscr{L}$, allora il codice ciclico minimale $(M^{(-v)}(x))^{\perp}$ ha come matrice generatrice $\Gamma^{(v)} $ $v$-esimo blocco della trasformata di Winograd.
    \end{teorema}
\end{frame}

%%%%%%%%%%%%%%%%%%%%%%%%%%%%%%%%%%%%%%%5
%%%%%%%%%%%%%%%%%%%%%%%%%%%%%%%%%%%%%%%5
%%%%%%%%%%%%% APPLICAZIONI
\begin{frame}
  Generalizzazione: sia $a(x) = M^{(v_{1})}(x) \cdots M^{(v_{k})}(x)$.

    \begin{corollario}
      $A=(v_{1}, \dots, v_{k}) \subseteq \mathscr{L}$, allora
      \begin{align*}
	  \Gamma^{(A)}
	  =
	  \left(
	  \begin{array} { c }
	  \Gamma^{(v_{1})}  \\ \\
	  \vdots \\ \\
	  \Gamma^{(v_{k})}
	  \end{array}
	  \right)
      \end{align*}
      è matrice di controllo del codice $\mathfrak{a} = (  M^{(v_{1})}(x) \cdots M^{(v_{k})}(x) )$
      ed è matrice generatrice del codice $(  M^{(-v_{1})}(x)\cdots M^{(-v_{k})}(x) )^{\perp}$.
    \end{corollario}
\end{frame}




%%%%%%%%%%%%%%%%%%%%%%%%%%%%%%%%%%%%%%%5
%%%%%%%%%%%%% APPLICAZIONI
\subsection{$\Delta$ NELLA DECODIFICA}
\begin{frame}
   $A=(v_{1}, \dots, v_{k}) \subseteq \mathscr{L}$, $\mathfrak{a} = (  M^{(v_{1})}(x) \cdots  M^{(v_{k})}(x) )$.
   L'immagine della parola $c(x)$ di $\mathfrak{a}$ tramite $\gamma$ è costituita da sottovettori circolanti nulli nei posti $v_{1}, \dots, v_{k}$. Definiamo questi sottovettori {\bf privi di informazione}.
    \begin{align*}
      \mathbf{c} \in \mathcal{V}_{r, q}^{\mathscr{L}}
      \mapsto
      \mathbf{c} ~ \Bigg|~ \coprod_{v\in \mathscr{L} \setminus A} \mathcal{V}_{m(v), q}^{c}
    \end{align*}
\end{frame}



%%%%%%%%%%%%%%%%%%%%%%%%%%%%%%%%%%%%%%%5
%%%%%%%%%%%%% APPLICAZIONI
\begin{frame}
    \begin{esempio}
      $c(x) = 1 + x + x^2 + x^5 = (1,1,1,0,0,1,0)\in (M^{(0)}(x)M^{(1)}(x))$.\\ Tramite $\gamma$ diventa:
      \begin{align*}
	  \gamma(c(x)) = (0,0,x^2) = (0|0,0,0|0,0,1)
      \end{align*}
      Sarà sufficiente inviare il terzo blocco $(0,0,1)$ invece della parola $(1,1,1,0,0,1,0)$. In fase di decodifica si aggiungeranno i sottovettori privi di informazione e si applicherà $\Delta$ per ottenere $(1,1,1,0,0,1,0)$.
    \end{esempio}
\end{frame}

%%%%%%%%%%%%%%%%%%%%%%%%%%%%%%%%%%%%%%%5
%%%%%%%%%%%%% BIBLIOGRAFIA
%%%%%%%%%%%%%%%%%%%%%%%%%%%%%%%%%%%%%%%%%%%%%%%%%%%%%%%%%%%%%%%
\thispagestyle{empty}
%%%%%%%%%%%%%%%%%%%%%%%%%%%%%%%%%%%%%%%5
\begin{frame}
Fonti principali:
    \begin{itemize}
      \item U. Cerruti, F. Vaccarino \emph{From Cyclotomic Extensions to Generalized
	    Ramanuyan's Sum through the Winograd Transform}, pre-print.
      \item Luigia Berardi, \emph{Algebra e teoria dei codici correttori}, Franco angeli
	    Editore 1994.
      \item Richard E. Blahut, \emph{Theory and Practice of Error Control Codes},
	    Addison Wesley publishing Company, 1984.
      \item Ian F. Blake, Ronald C. Mullin, \emph{The Mathematical Theory of Coding},
	    Academic Press 1975.

    \end{itemize}
\end{frame}




\end{document}