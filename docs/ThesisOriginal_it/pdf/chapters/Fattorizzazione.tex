%%%%%%%%%%%%%%%%%%%%%%%%%%%%%%%%%%%%%%%%%%%%%%%%
%%%%%%%%%%%%%%%%%%%%%% CAPITOLo 2   %%%%%%%%%%%%%%%%%%%%
%%%%%%%%%%%%%%%%%%%%%%%%%%%%%%%%%%%%%%%%%%%%%%%%
\chapter{Fattorizzazione di $x^r - 1$} \label{cap:fattorizzazione}

Possiamo ampliare il diagramma del capitolo precedente con altre due
strutture algebriche definite dalla fattorizzazione dell'algebra
$\mathcal{R}_{r, \mathbb{F}} $ in un prodotto di campi:

\[
\begindc{\commdiag}[30]
%sotto
\obj(0,5)[V]{$ \mathcal{V}_{r, \mathbb{F}}^{c} $}
\obj(40,5)[M]{$\mathcal{M}_{r,\mathbb{F} }^{c}$}


%metà
\obj(0,30)[R]{$ \quotient{ \mathbb{F}[x] }{ (x^r -1 )} $}
\obj(40,30)[A]{$ \mathbb{F}C_{r} $}
\obj(-40,30)[P]{$ \prod_{v} \mathbb{F} (\xi^{v}) $}

%sopra
\obj(0,60)[Q]{$ \prod_{v} \quotient{ \mathbb{F}[x] }{ (M^{(v)}(x) )} $}

%frecce orizzontali
\mor{V}{M}{$\psi_{1}$}
\mor{R}{A}{$\psi_{4}$}
\mor{R}{P}{$\eta$}

%frecce verticali
\mor{R}{V}{$\psi_{2}$}
\mor{A}{M}{$\psi_{3}$}
\mor{R}{Q}{$\gamma$}

%frecce oblique
\mor{P}{Q}{$\mu$}
%\mor{A}{Q}{$\mu_{2}$}

\enddc
\]


\noindent
Per questo scopo è necessario fattorizzare $x^{r} - 1$ nel prodotto
di polinomi irriducibili su $\mathbb{F}$. Proponiamo inizialmente tre esempi sui
quali sarà articolata la teoria:

\begin{esempio}
   Per $\mathbb{F} = \mathbb{Q}$ ed $r = 6$ allora
   \begin{align*}
      \mathcal{R}_{6, \mathbb{Q}} 
       := \quotient{\mathbb{Q} \lbrack x \rbrack  }{(x^{6} - 1)} 
   \end{align*}
  ed $x^{6}-1$ si scompone\footnote{Vedere \emph{ricerca dei polinomi minimi sui campi finiti} in appendice.} nel prodotto dei polinomi ciclotomici irriducibili:
\begin{align*}
  x^{6} - 1 =   \prod_{d \mid r} \Phi_{d}(x) =
  \Phi_{1}(x)\Phi_{2}(x)\Phi_{3}(x)\Phi_{6}(x) \\
  x^{6} - 1 = (x-1)(x+1)(x^2+x+1)(x^2-x-1)
\end{align*}
In conseguenza del fatto che ci sono radici seste dell'unità
distinte aventi lo stesso polinomio minimo,  il polinomio $x^{6} - 1$ non si
scompone esclusivamente in polinomi di primo grado. Sia $\xi_{6}$ radice
primitiva sesta:\\
$\xi_{6}^{0} = 1$ è radice di $x-1$.\\
$\xi_{6}^{3} = -1$ è radice di $x + 1$.\\
$\xi_{6}^{2}, \xi_{6}^{4} $ sono radici di $x^2+x+1$.\\
$\xi_{6}^{1}, \xi_{6}^{5}$ sono radici di $x^2-x-1$.\\
Osserivamo che le radici di uno stesso polinomio ciclotomico espresse come potenze della radice primitiva sesta $\xi_{6}$ hanno esponenti raggruppati nelle orbite dell'azione di
$\mathbb{Z}_{6}^{\star}$ su $\mathbb{Z}_{6}$: \\
$O(0)= \lbrace 0 \rbrace$.\\
$O(1)= \lbrace 1,5 \rbrace$.\\
$O(2)= \lbrace 2,4 \rbrace$.\\
$O(3)= \lbrace 3 \rbrace$.\\
Questo fatto sarà dimostrato in generale nel corso del capitolo, così come si
dimostrerà che esiste un isomorfismo fra
$\mathcal{R}_{6, \mathbb{Q}} $ ed il prodotto dei quozienti sui singoli
polinomi:
\begin{align*}
\quotient{\mathbb{Q} \lbrack x \rbrack  }{(x^{6} - 1)} =
\quotient{\mathbb{Q} \lbrack x \rbrack  }{ \prod_{d\mid r} \Phi_{d}(x)} \cong
\prod_{d\mid r} \quotient{\mathbb{Q} \lbrack x \rbrack  }{\Phi_{d}(x)}
\end{align*}
Dato che i polinomi ciclotomici sono irriducibili, i fattori in cui si scompone
$ \mathcal{R}_{6, \mathbb{Q}} $ sono campi.
\end{esempio}

\begin{esempio} \label{ese:fattor2_9}
Se scegliamo invece il campo finito,
$\mathbb{F} = \mathbb{Z}_{2} = GF(2)$ campo di Galois di ordine $2$, ed
$r = 9$ allora
\begin{align*}
  \mathcal{R}_{9, \mathbb{Z}_{2}} 
       := \quotient{\mathbb{Z}_{2} \lbrack x \rbrack  }{(x^{9} - 1)} 
\end{align*}
ed $x^{9}-1$ si scompone nel prodotto di polinomi irriducibili:
\begin{align*}
  x^9 - 1 &= M^{(0)}(x) M^{(1)}(x) M^{(3)}(x) \\
          &= (x-1)(x^6+x^3+1)(x^2+x+1)
\end{align*}
Sia $\xi_{9}$ radice primitiva nona:\\
$\xi_{9}^{0} = 1$ è radice di $M^{(0)}(x)= x-1$.\\
$\xi_{9}^{1},\xi_{9}^{2}, \xi_{6}^{4}, \xi_{9}^{8}, \xi_{9}^{7},\xi_{9}^{5} $
sono radici di $M^{(1)}(x) = x^6+x^3+1$.\\
$\xi_{9}^{3}, \xi_{9}^{6}$ sono radici di $M^{(3)}(x) = x^2+x+1$.\\
In questo caso, gli esponenti di $\xi_{9}$ che soddisfano lo
stesso polinomio ciclotomico sono raggruppate nelle orbite dell'azione di
$\mathbb{Z}_{9}^{\star}$ su $\mathbb{Z}_{9}$: \\
$O(0)= \lbrace 0 \rbrace$.\\
$O(1)= \lbrace 1,2,4,8,7,5 \rbrace$.\\
$O(3)= \lbrace 3,6 \rbrace$.\\
e vale l'isomorfismo
\begin{align*}
\quotient{\mathbb{Z}_{2} \lbrack x \rbrack  }{(x^{9} - 1)}
\cong
\prod_{v = 0,1,3} \quotient{\mathbb{Q} \lbrack x \rbrack  }{M^{(v)}(x)}
\end{align*}
\end{esempio}

I polinomi $M^{(v)}(x)$ della fattorizzazione dell'esempio precedente coincidono
con i polinomi ciclotomici definiti dalla fattorizzazione di $x^9 - 1$ nel
campo dei razionali. Possiamo scrivere impropriamente
\begin{align*}
   M^{(0)}(x) = \Phi_{1}(x) \qquad M^{(1)}(x) = \Phi_{9}(x) 
                            \qquad M^{(3)}(x) = \Phi_{3}(x)
\end{align*}
quindi abbiamo la stessa
fattorizzazione che si avrebbe avuto
scegliendo $\mathbb{F} = \mathbb{Q}$ anziché $\mathbb{F} = \mathbb{Z}_{2}$. \\
Questo non accade in generale; ad esempio per la fattorizzazione $x^7 - 1$,
dove i polinomi della fattorizzazione sul campo finito sono di più di quelli
nel caso della fattorizzazione sui razionali.

\begin{esempio} \label{ese:fattorQ_7}
Sia $\mathbb{F} = \mathbb{Q}$ ed $r = 7$ allora
\begin{align*}
  x^{7} - 1 &= \Phi_{1}(x) \Phi_{7}(x) \\
            &= (x-1)(x^6 +x^5 + x^4 + x^3 +x^2 + x +1)
\end{align*}
Sia $\xi_{7}$ radice primitiva settima:\\
$\xi_{7}^{0} = 1$ è radice di $x-1$.\\
$\xi_{7}^{j}$ è radice di $\Phi_{7}(x)$ per $j = 1, \dots ,6 $.\\
Come prima gli esponenti delle radici primitive settime che soddisfano lo
stesso polinomio ciclotomico sono raggruppate nelle orbite dell'azione di
$\mathbb{Z}_{7}^{\star}$ su $\mathbb{Z}_{7}$: \\
$O(0)= \lbrace 0 \rbrace$.\\
$O(1)= \lbrace 1,2,3,4,5,6 \rbrace$.\\
Mentre se fattorizziamo $ x^{7} - 1$ sul campo $\mathbb{F} = \mathbb{Z}_{2}$,
otteniamo il prodotto
\begin{align*}
  x^{7} - 1 &= M^{(0)}(x) M^{(1)}(x) M^{(3)}(x) \\
            &= (x-1)(x^3 + x + 1)(x^3 + x^2 + 1) 
\end{align*}
Le radici primitive settime si distribuiscono nel modo seguente: \\
$\xi_{7}^{0} = 1$ è radice di $M^{(0)}(x) = x-1$.\\
$\xi_{7}^{1}, \xi_{7}^{2}, \xi_{7}^{4}$ sono radici di $M^{(1)}(x) = x^3 + x +
1$.\\
$\xi_{7}^{3}, \xi_{7}^{5}, \xi_{6}^{1}$ sono radici di $M^{(3)}(x)= x^3 + x^2 +
1$.\\
A differenza del caso sui razionali gli esponenti delle radici primitive settime
che soddisfano lo
stesso polinomio irriducibile della fattorizzazione sono raggruppate nelle
orbite dell'azione di un {\bf sottogruppo} di
$\mathbb{Z}_{7}^{\star}$ (precisamente quello costituito dagli elementi che
coincidono con gli elementi dell'orbita O(1)), su $\mathbb{Z}_{7}$: \\
$O(0)= \lbrace 0 \rbrace$.\\
$O(1)= \lbrace 1,2,4 \rbrace$.\\
$O(3)= \lbrace 3,5,6 \rbrace$.\\
Quindi $ \mathcal{R}_{7, \mathbb{Q}} $ si scompone nel prodotto di due
campi, 
\begin{align*}
\quotient{\mathbb{Q} \lbrack x \rbrack  }{(x^{7} - 1)}
\cong
\quotient{\mathbb{Q} \lbrack x \rbrack  }{\Phi_{1}(x)}
\times
\quotient{\mathbb{Q} \lbrack x \rbrack  }{\Phi_{7}(x)}
\end{align*}
mentre $\mathcal{R}_{7, \mathbb{Z}_{2}}$ si scompone in tre campi.
\begin{align*}
\quotient{\mathbb{Z}_{2} \lbrack x \rbrack  }{(x^{7} - 1)}
\cong
\quotient{\mathbb{Z}_{2} \lbrack x \rbrack  }{M^{(0)}(x)}
\times
\quotient{\mathbb{Z}_{2} \lbrack x \rbrack  }{M^{(1)}(x)}
\times
\quotient{\mathbb{Z}_{2} \lbrack x \rbrack  }{M^{(3)}(x)}
\end{align*}
\end{esempio}

\noindent
È interessante notare come le algebre $\mathcal{R}_{r, \mathbb{F}}$ si
decompongono in campi, in modo simile a come gli interi si fattorizzano in
numeri primi. A determinare la decomposizione in campi è la fattorizzazione del
polinomio $x^r - 1$ alla quale è dedicato questo capitolo.

%%%%%%%%%%%%%%%%%%%%%%%%%%%%%%%%%%%%%%%%%%%%%%%%
%%%%%%%%%%%%%%%%%%%%%% SEZIONI    %%%%%%%%%%%%%%%%%%%%
\section{Classi ciclotomiche}

Scopo del paragrafo è scomporre il polinomio $x^r-1$ in un prodotto di polinomi minimi. Esaminiamo il caso generale per poi proseguire nel caso in cui $x^r-1$ è definito sul campo dei razionali e sui campi finiti. Ricordiamo inizialmente la definizione di polinomio minimo e dimostriamo la sua irriducibilità mediante il seguente lemma\footnote{Variante del teorema $6.1.16$ pag. $314$ \cite{cattaneo}.}.
\begin{lemmax}\label{le:poliMinimo}
   Sia $\mathbb{F}$ campo perfetto, $\xi$ radice (non necessariamente primitiva!) $r$-esima dell'unità,  $\mathbb{F}(\xi)$ estensione di  $\mathbb{F}$ e sia $v_{\xi}$ l'omomorfismo di valutazione:
   \begin{align*}
      v_{\xi}: \mathbb{F}[x]  &\longrightarrow  \mathbb{F}[\xi]  \\
		    f(x) &\longmapsto f(\xi)
   \end{align*}
   Allora valgono le seguenti proprietà: 
   \begin{enumerate}
      \item $ker(v_{\xi}) = (p(x))$ dove $p(x)$ è un polinomio irriducibile monico detto {\bf polinomio minimo} di $\xi $ su $\mathbb{F}$.
      \item Sia $f(x) \in \mathbb{F}[x]$ allora $f(\xi) = 0 \iff p(x) \mid f(x)$.
      \item $\mathbb{F}[\xi]=\mathbb{F}(\xi)$ ed inoltre vale l'isomorfismo
        \begin{align*}
           \mathbb{F}(\xi) \cong \quotient{\mathbb{F}[x]}{(p(x))}
        \end{align*}
       \item Se $deg(p(x)) = m$ allora $[\mathbb{F}(\xi): \mathbb{F}] = m$ ed $\mathbb{F}(\xi)$ ha come base $\lbrace1, \xi, \xi^2, \dots, \xi^{m-1} \rbrace$.
   \end{enumerate}
\end{lemmax}
\begin{proof}
   Dimostriamo i $4$ punti separatamente:
   \begin{enumerate}
      \item Dal teorema fondamentale degli isomorfismi di anelli segue che 
      
      \vspace{0.2cm}

      \[
      \begindc{\commdiag}[30]
      %insiemi
      \obj(0,25)[Fx]{$ \mathbb{F}[x] $}
      \obj(40,25)[Quoz]{$ \quotient{\mathbb{F}[x]}{ker(v_{\xi})} $}
      \obj(35,0)[Fxi]{$ \mathbb{F}[\xi] $}

      %frecce
      \mor{Fx}{Quoz}{$\pi$}
      \mor{Fx}{Fxi}{$ v_{\xi}$}
      \mor{Quoz}{Fxi}{$ \cong $}

      \enddc
      \]

      \vspace{0.2cm}

      Quindi 
      \begin{align*}
         \mathbb{F}[\xi] \cong \quotient{\mathbb{F}[x]}{ker(v_{\xi})}
      \end{align*}
      Ma $\mathbb{F}[\xi]$ è sottoanello del campo $\mathbb{F}(\xi)$ ed  è quindi dominio di integrità. Quindi lo è $\quotient{\mathbb{F}[x]}{ker(v_{\xi})}$ e da una nota proprietà $ker(v_{\xi})$ è un ideale primo. Quindi abbiamo che è generato da un polinomio irriducibile. Esiste allora un polinomio monico $p(x)$ (o riconducibile ad un polinomio monico dividendo per il coefficiente direttivo) tale che $ker(v_{\xi}) = (p(x))$.
      
      \item Sia $f(\xi) = 0$ allora $f(x) \in ker(v_{\xi})$ e viceversa. Essendo $ker(v_{\xi}) = (p(x))$ dal punto precedente, allora segue la tesi.
      
      \item Dall'irriducibilità di $p(x)$ si ha che $\quotient{\mathbb{F}[x]}{(p(x))}$ è un campo e dal teorema fondamentale degli isomorfismi di anelli segue che anche $\mathbb{F}[\xi]$ è un campo. Quindi $\mathbb{F}[\xi]= \mathbb{F}(\xi)$ ed 
      \begin{align*}
           \mathbb{F}(\xi) \cong \quotient{\mathbb{F}[x]}{(p(x))}
      \end{align*}
      
      \item Sia $m$ il grado di $p(x)$, allora si dimostra che $\Xi = \lbrace1, \xi, \xi^2, \dots, \xi^{m-1} \rbrace$ è una base di $\mathbb{F}[\xi] = \mathbb{F}(\xi)$. \\
      \emph{Linearmente indipendenti}: sia 
      \begin{align*}
         \sum_{j=0}^{m-1} a_{j}\xi^{j} = 0 \qquad \qquad a_{j} \in \mathbb{F}
      \end{align*}
      e sia $f(x) = \sum_{j=0}^{m-1} a_{j}\xi^{j}$ polinomio associato alla combinazione lineare precedente. Dato che $f(x)$ si annulla in $\xi$ allora appartiene al nucleo dell'omomorfismo di valutazione $ker(v_{\xi})$ ed è un quindi un multiplo di $p(x)$. Dato che la differenza fra il grado di $p(x)$ ed il grado di $f(x)$ è $1$ allora $f(x)$ deve essere necessariamente il polinomio nullo. \\
      \emph{Generatori}: sia $a \in \mathbb{F}(\xi)$ elemento generico allora esiste un polinomio $f(x)$ che calcolato in $\xi$ risulta valere $a$:
      \begin{align*}
         a = f(\xi) = v_{\xi}(f(x))
      \end{align*}
      Considerando la divisione di $f(x)$ per $p(x)$ si ottiene
      \begin{align*}
         f(x) = p(x)q(x) + r(x) \qquad \qquad 0 \leq deg(r(x)) \le deg(p(x))
      \end{align*}
      da cui, calcolando $f(x)$ in $\xi$:
      \begin{align*}
         f(\xi) = 0 + r(\xi) = a
      \end{align*}
      e quindi $r(\xi) = a$ è una combinazione lineare ad elementi in $\mathbb{F}$ di $\Xi$. 
   \end{enumerate}
\end{proof}

\begin{osservazione}
   Se $\xi$ è una radice \emph{primitiva} $r$-esima dell'unità, allora possiamo applicare il lemma \ref{le:poliMinimo} precedente ad $\xi^{t}$ con $t \in \mathbb{Z}_{r}^{\star}$ ed ottenere così l'estensione $\mathbb{F}(\xi^{t})$ del campo $\mathbb{F}$ con base su $\mathbb{F}$ data da $\lbrace 1, \xi^{t}, \xi^{2t}, \dots, \xi^{m_{t} - 1} \rbrace$ per $m_{t}$ grado del polinomio minimo di $\xi^{t}$.
\end{osservazione}

Nella fattorizzazione di $x^r - 1$ possono esistere delle radici distinte
aventi lo stesso polinomio minimo, come affermato nel caso dei campi finiti dal
seguente lemma:
\begin{lemmax} \label{le:lemmaGalois}
   Sia $\beta$ elemento del campo finito $\mathbb{F}_{q}$ di
caratteristica $p$ e di ordine $p^n=q$, allora $\beta$ e $\beta^{p}$ hanno lo
stesso polinomio minimo su $\mathbb{F}$.
\end{lemmax}
\begin{proof}
   Sia $a(x) = \sum_{j=0}^{m} a_{j}x^j$ il polinomio minimo di $\beta$
   allora 
   \begin{align*}
      a(\beta^p ) = \sum_{j=0}^{m} a_{j} (\beta^p)^j 
                  = \sum_{j=0}^{m} a_{j}^p (\beta^j)^p 
                  = (\sum_{j=0}^{m} a_{j} \beta^j )^p 
                  = 0
   \end{align*}
\end{proof}
\noindent
Anche per i campi a caratteristica zero accade che radici distinte
dell'unità abbiano lo stesso polinomio minimo. Per esempio in $\mathbb{Q}$ tutte
le radici primitive $r$-esime di ordine $d$, dove $d \mid r$, appartengono allo stesso polinomio ciclotomico $\Phi_{d}(x)$ irriducibile\footnote{\cite{cattaneo}
pag. 133 per una presentazione generale. \cite{milne} pag. 40 per 
l'irriducibilità. Per ricavare il $d$-esimo polinomio ciclotomico usando la formula di inversione di Moebius \cite{sivarama} pag. 198.}.
Le radici distinte di un polinomio che hanno lo
stesso polinomio minimo nella sua fattorizzazione sono dette {\bf radici coniugate}.
Esiste un modo per determinare quali sono le radici coniugate di un polinomio
se osserviamo che sono legate fra di loro dagli automorfismi del gruppo di Galois
dell'estensione che le contiene\footnote{Variante della
proposizione $7.2.3$ \cite{cattaneo} pag. 349.}.
\begin{teorema}\label{teo:stesseRadici}
Sia $\mathbb{F}$ campo perfetto, $\xi$ radice primitiva $r$-esima
dell'unità, $Gal(\mathbb{F}(\xi), \mathbb{F})$, gruppo di Galois
dell'estensione $\mathbb{F}(\xi) $ su $ \mathbb{F}$, allora
\begin{enumerate}
   \item Per ogni $\varphi \in Gal(\mathbb{F}(\xi), \mathbb{F})$, $\xi$ e $\varphi(\xi)$ hanno lo stesso
polinomio minimo.
   \item Per ogni $t \in \mathbb{Z}_{r}$ e per ogni $\varphi \in Gal(\mathbb{F}(\xi), \mathbb{F})$, $\xi^{t}$ e
$\varphi(\xi^{t})$ hanno lo stesso polinomio minimo.
\end{enumerate}
\end{teorema}
\begin{proof}
È sufficiente dimostrare il secondo punto, essendo il primo un suo caso
particolare per $t = 1$. \\
Sia $a(x) = \sum_{j=0}^{m} a_{j}x^j$ polinomio minimo di $\xi^{t}$
allora 
\begin{align*}
      a(\varphi(\xi^t) ) &= \sum_{j=0}^{m} a_{j} (\varphi(\xi^t) )^j
                  = \sum_{j=0}^{m} \varphi(a_{j}) \varphi(\xi^t)^j  
                  = \sum_{j=0}^{m} \varphi(a_{j}) \varphi((\xi^t )^j) \\ 
                  &= \sum_{j=0}^{m} \varphi(a_{j}(\xi^t)^j)
                  = \varphi( \sum_{j=0}^{m} a_{j} (\xi^t )^j ) 
                  = \varphi(0) = 0
\end{align*}
\end{proof}
Il teorema precedente ci fornisce tutte le informazioni utili per calcolare i
polinomi della fattorizzazione $x^r - 1$, svelando il rapporto che intercorre
fra due radici coniugate e due radici non coniugate.\\
Indicando con $E^{(r)} = \lbrace \xi^{j}\rbrace_{j=0}^{r-1}$ il gruppo delle
radici $r$-esime dell'unità, riformuliamo nei prossimi due paragrafi, sul campo dei
razionali e sui campi finiti, il teorema precedente
identificando il generico elemento $\xi^{j}$ di $E^{(r)}$ con il suo esponente
$j$ e ridefinendo $Gal(\mathbb{F}(\xi), \mathbb{F})$ come gruppo che agisce su $\mathbb{Z}_{r}$ invece che su
$E^{(r)}$. 

%%%%%%%%%%%%%%%%%%%%%%%%%%%%%%%%%%%%%%%%%%%%%%%%
%%%%%%%%%%%%%%%%%%%%%% subSEZIONI    %%%%%%%%%%%%%%%%%%%%
\subsection{Il gruppo $Gal(\mathbb{Q}(\xi), \mathbb{Q}))$} \label{cap2:Grazionali}

Sia $Gal(\mathbb{Q}(\xi), \mathbb{Q}))$ per $\xi$ radice primitiva $r$-esima
dell'unità. L'estensione $\lbrack \mathbb{Q}(\xi), \mathbb{Q} \rbrack$ ha grado
$\varphi(r)$ in conseguenza del fatto\footnote{\cite{cattaneo} pag. 133.} che il grado dell'$r$-esimo polinomio
ciclotomico ha come radici tutte le potenze $\xi^{k}$ per $(r,k) = 1$.
Dal teorema \ref{teo:stesseRadici} ogni elemento di $Gal(\mathbb{Q}(\xi), \mathbb{Q}))$ manda $\xi$ in un'altra radice
di $\Phi_{r}(x)$:  \\
per ogni $\varphi \in Gal(\mathbb{Q}(\xi), \mathbb{Q}))$ esiste $k$ intero, $(k,r) = 1$ tale che 
$\varphi(\xi) = \xi^k $.  \\
Viceversa per ogni $k$ intero, $(k,r) = 1$ allora l'applicazione $\varphi_{k}$
che manda $\xi$ in $\xi^{k}$ è un automorfismo. \\
Esiste quindi una corrispondenza fra il gruppo degli elementi invertibili di
$\mathbb{Z}_{n}$ e gli elementi di $Gal(\mathbb{Q}(\xi), \mathbb{Q}))$.

\begin{teorema}
Sia $\xi$ radice primitiva $r$-esima dell'unità, allora la corrispondenza
\begin{align*}
\psi: \mathbb{Z}_{r}^{\star} & \longrightarrow  Gal(\mathbb{Q}(\xi), \mathbb{Q})) &   \\
                           k &\longmapsto  \varphi_{k} : \mathbb{Q}(\xi)  \longrightarrow  \mathbb{Q}(\xi) \\
                                              & \qquad \qquad \qquad \xi \longmapsto \varphi_{k}(\xi) = \xi^{k}
\end{align*}
è un isomorfismo di gruppi.
\end{teorema}
\begin{proof}
$\psi$ è biunivoca considerando che la cardinalità dei due insiemi è $\varphi(r)$ e dato che tutte le $\xi^k$ sono distinte per $(r,k)=1$.\\
Si verifica che $\psi$ è un omomorfismo di gruppi:
siano $i,j$ elementi di $\mathbb{Z}_{r}^{\star}$ tali che $ij \equiv k \mod{r}$ allora $\varphi_{i}\varphi_{j} = \varphi_{k}$, infatti
\begin{align*}
   \varphi_{i}\varphi_{j}(\xi) = \varphi_{i}(\xi^{j}) = \xi^{ji} = \xi^{k} = \varphi_{k}(\xi)
\end{align*}
segue quindi
\begin{align*}
   \mathbb{Z}_{r}^{\star} \cong  Gal(\mathbb{Q}(\xi), \mathbb{Q})) 
\end{align*}
\end{proof}

Osserviamo che gli elementi $\varphi_{k}$ del gruppo di Galois analizzati nel precedente teorema, oltre ad agire sul sottogruppo moltiplicativo delle radici primitive $r$-esime dell'unità, possono anche agire su ogni altra radice $r$-esima. Sia $\xi^{l}$ radice di ordine $l$, allora $\varphi_{k}(\xi^{l})$ manda $\xi^{l}$ in un'altra radice dello stesso polinomio minimo di $\xi^{l}$.\\
È dunque possibile concentrare l'attenzione solo sugli esponenti degli elementi di $E^{(r)} = \lbrace \xi^{j}\rbrace_{j=0}^{r-1} \cong \mathbb{Z}_{r}$, rappresentando l'azione delle $\varphi_{k}$ di $Gal(\mathbb{Q}(\xi), \mathbb{Q}))$ come l'azione del gruppo $\mathbb{Z}_{r}^{\star}$ su $\mathbb{Z}_{r}$ in virtù del teorema precedente.\\
Riassumendo, l'automorfismo
\begin{align*}
\varphi_{k}: E^{(r)}  &\longrightarrow  E^{(r)}   \\
               \xi^{l} &\longmapsto \varphi_{k}(\xi^{l}) = \xi^{lk}
\end{align*}
definisce l'azione di gruppi
\begin{align*}
 Gal(\mathbb{Q}(\xi), \mathbb{Q})) \times E^{(r)}  &\longrightarrow  E^{(r)}   \\
           (\varphi_{k},\xi^{l}) &\longmapsto \varphi_{k}(\xi^{l}) = \xi^{lk}
\end{align*}
che grazie all'isomorfismo di gruppi appena ricavato ed alla corrispondenza fra gli elementi di $E^{(r)}$ e gli esponenti di $\xi$ che lo definiscono, mantiene le stesse caratteristiche di 
\begin{align*}
 \mathbb{Z}_{r}^{\star} \times \mathbb{Z}_{r} \longrightarrow  \mathbb{Z}_{r}   \\
           (k,l) \longmapsto kl
\end{align*}
Risulta quindi essere dimostrato il teorema:
\begin{teorema} \label{teo:iffPoliMinimoQ}
Due elementi $l_{1}$ ed $l_{2}$ di $\mathbb{Z}_{r}$ sono nella stessa orbita della azione 
\begin{align*}
 \mathbb{Z}_{r}^{\star} \times \mathbb{Z}_{r} &\longrightarrow  \mathbb{Z}_{r}   \\
           (k,l) &\longmapsto kl
\end{align*}
se e solo se $\xi^{l_{1}}$ e $\xi^{l_{2}}$ hanno lo stesso polinomio minimo su $\mathbb{Q}$.
\end{teorema}

\begin{definizione}\label{cap2:orbiteq}
   Le orbite dell'azione definita nel teorema precedente, ciascuna delle quali definisce l'insieme di potenze di $\xi$ che soddisfano lo stesso polinomio minimo, sono dette {\bf classi ciclotomiche}. Due elementi appartenenti ad una stessa classe ciclotomica sono detti {\bf coniugati}. 
\end{definizione}

Negli esempi proposti all'inizio del capitolo, abbiamo osservato che il teorema \ref{teo:iffPoliMinimoQ} non può essere generalizzato per un campo qualsiasi, mantenendo l'isomorfismo $Gal(\mathbb{F}(\xi), \mathbb{F}) \cong  \mathbb{Z}_{r}^{\star}$ il quale si verifica nel caso specifico dei razionali. Infatti passando da $\mathbb{Q}$ al campo $\mathbb{Z}_{2}$ nell'esempio \ref{ese:fattorQ_7}, il gruppo $Gal(\mathbb{F}(\xi), \mathbb{F})$, da essere isomorfo a $\mathbb{Z}_{r}^{\star}$ si restringe ad un suo sottogruppo. \\
Purtroppo neppure tutti i campi a caratteristica zero hanno corrispondente gruppo di Galois isomorfo a $\mathbb{Z}_{r}^{\star}$, come si verifica esplorando un caso sul campo dei reali.
\begin{esempio}
Sia $\mathbb{F}=\mathbb{R}$, ed $r$ generico: 
\begin{align*}
  \mathcal{R}_{r, \mathbb{R}} 
       := \quotient{\mathbb{R} \lbrack x \rbrack  }{(x^{r} - 1)} 
\end{align*}
Sia $\xi$ radice primitiva di $x^r-1$, allora $Gal(\mathbb{R}(\xi), \mathbb{R})) = \lbrace \varphi_{1}, \varphi_{-1} \rbrace $ e quindi $Gal(\mathbb{R}(\xi), \mathbb{R})) \cong \mathbb{Z}_{2}$.
\end{esempio}

%%%%%%%%%%%%%%%%%%%%%%%%%%%%%%%%%%%%%%%%%%%%%%%%
%%%%%%%%%%%%%%%%%%%%%% subSEZIONI    %%%%%%%%%%%%%%%%%%%%
\subsection{Il gruppo $Gal(\mathbb{F}_{q}(\xi), \mathbb{F}_{q})$} \label{cap2:Gfiniti}

Nel caso finito, la prima informazione utile che si può ottenere è il campo di spezzamento del polinomio $x^r-1$ che sarà estensione di $\mathbb{F}_{q}$, ma prima di cominciare la sua ricerca dobbiamo fare una
\begin{osservazione}\label{cap2:osservaziorp}
Sia $\mathbb{F}_{q} = GF(q)$ campo di Galois di caratteristica $p$ e di ordine $q=p^n$. Se $r$ e $q$ non sono primi fra loro, cioè se esiste un $d = (p,r) \neq 1$ allora $r = \rho p^n$ con $(\rho,p) = 1$. Il polinomio $x^r-1$ può essere scomposto come
\begin{align*}
x^r-1 =  x^{\rho p^n} - 1^{p^{n}} = (x^{\rho}-1)^{p^n} 
\end{align*}
quindi le radici di $x^r-1$ coincidono con le radici di $x^{\rho}-1$ con molteplicità $p^n$.
Quindi per evitare casi ridondanti, da questo punto in poi della tesi $p$ ed $r$ saranno sempre considerati coprimi. 
\end{osservazione}
Il prossimo lemma determina il campo di spezzamento di $x^r-1$ ricordando che il {\bf periodo} di un elemento $x$ di un gruppo $X$ è il più piccolo intero positivo $t$ tale che $x^t = 1_{X}$.
\begin{lemmax} \label{le:campoDiSpezzamento}
Sia $\mathbb{K}$ campo di spezzamento di $x^r-1$ in $\mathbb{F}_{q}$, allora il grado dell'estensione di $\mathbb{K}$ su $\mathbb{F}_{q}$ coincide con il periodo di $q$ in $\mathbb{Z}_{r}^{\star}$
\end{lemmax}
\begin{proof}
Dato che $\mathbb{K}$ estende $\mathbb{F}_{q} \supseteq \mathbb{F}_{p}$ la sua caratteristica è $p$.
Sia $m = \lbrack \mathbb{K} : \mathbb{F}_{q} \rbrack$, cioè
\begin{align*}
   \mathbb{F}_{p} \subseteq \mathbb{F}_{p^n} =  \mathbb{F}_{q}\subseteq \mathbb{F}_{q^m} = \mathbb{K}
\end{align*}
Allora, indicando con $\arrowvert \cdot \arrowvert$ la cardinalità di un insieme, si osserva che 
\begin{enumerate}
   \item $\arrowvert \mathbb{K}^{\star} \arrowvert = q^m -1$, dato che $\arrowvert \mathbb{K} \arrowvert = q^m$ e $\mathbb{K}$ è un campo.
   \item $E^{(r)} := \lbrace \xi \in \mathbb{K}^{\star} \mid \xi^{r} = 1 \rbrace$ gruppo delle radici $r$-esime dell'unità è sottogruppo moltiplicativo di $\mathbb{K}^{\star}$ ha cardinalità $r$. 
\end{enumerate}
Dato che l'ordine dei sottogruppi deve dividere l'ordine del gruppo che lo contiene
\footnote{Ad esempio \cite{herstein} pag 43, \cite{cattaneo} pag. 240.}
allora 
\begin{align*}
   \arrowvert E^{(r)} \arrowvert \mid \arrowvert \mathbb{K}^{\star} \arrowvert \\
   r \mid q^m - 1
\end{align*}
Quindi $q^m \equiv 1 \mod{r}$: il periodo di $q$ in $\mathbb{Z}_{r}^{\star}$, indicato con 
$per_{\mathbb{Z}_{r}^{\star}}(q)$ è uguale ad $m$, grado dell'estensione $\lbrack \mathbb{K} : \mathbb{F}_{q} \rbrack$.
\end{proof}

\begin{esempio}
   Riprendendo l'esempio \ref{ese:fattorQ_7} , dove $r=7$ e $\mathbb{F} = \mathbb{Z}_{2}$, il campo di spezzamento di $x^7 -1$ è $\mathbb{F}_{2^3}$, essendo $3$ il periodo di $2$ in $\mathbb{Z}_{7}^{\star}$. Notiamo che il grado dell'estensione è pari alla cardinalità dell'orbita dell'azione che contiene $1$.
\end{esempio}

Sia $\xi$ radice primitiva $r$-esima dell'unità sul campo finito $\mathbb{F}_{q}$, allora la corrispondenza
\begin{align*}
\psi: \mathbb{Z}_{r}^{\star} & \longrightarrow  Gal(\mathbb{F}_{q}(\xi), \mathbb{F}_{q}) &   \\
                         k &\longmapsto  \varphi_{k} : \mathbb{F}_{q}(\xi)\longrightarrow  \mathbb{F}_{q}(\xi)\\
                                    & \quad \qquad \qquad \quad \xi^{l} \longmapsto \varphi_{k}(\xi^{l}) = \xi^{kl}
\end{align*}
come si può verificare, non è una corrispondenza biunivoca per ogni valore di $r$ e di $q$
\footnote{Si consideri $q=2$ ed $r= 7,15$.}
. Però è possibile trovare un sottogruppo di $\mathbb{Z}_{r}^{\star}$ da sostituire al dominio in modo che $\psi$ diventi un isomorfismo.
\begin{teorema} \label{teo:HsuGF}
Sia $\xi$ radice primitiva $r$-esima dell'unità sul campo finito $\mathbb{F}_{q}$, allora esiste un sottogruppo di $\mathbb{Z}_{r}^{\star}$ indicato con $G$ tale che la corrispondenza
\begin{align*}
\psi: G & \longrightarrow  Gal(\mathbb{F}_{q}(\xi), \mathbb{F}_{q}) &   \\
      k &\longmapsto \varphi_{k}
\end{align*}
è un isomorfismo di gruppi.
\end{teorema}
\begin{proof}
   \emph{Parte 1: analisi di $Gal(\mathbb{F}_{q}(\xi), \mathbb{F}_{q})$}. Sia $\varphi \in  Gal(\mathbb{F}_{q}(\xi), \mathbb{F}_{q})$,
   \begin{align*}
      \varphi: \mathbb{F}_{q}(\xi) & \longrightarrow  \mathbb{F}_{q}(\xi)&   \\
	    \xi^l &\longmapsto \varphi(\xi^l)
    \end{align*}
   Dal teorema \ref{teo:stesseRadici} $\xi^l$ e $\varphi(\xi^l)$ devono avere lo stesso polinomio minimo, quindi ogni elemento di $Gal(\mathbb{F}_{q}(\xi), \mathbb{F}_{q})$ permuta le radici dei polinomi minimi. Dal lemma \ref{le:campoDiSpezzamento} precedente, il campo di spezzamento $\mathbb{F}_{q}(\xi)$ di $x^r - 1$ su $\mathbb{F}_{q}$ possiede $q^{m}$ elementi per $m = per_{\mathbb{Z}_{r}^{\star}}(q)$. Inoltre\footnote{\cite{cattaneo} teorema $7.3.5$ pag. 361.}
   \begin{align*}
       \arrowvert Gal(\mathbb{F}_{q}(\xi), \mathbb{F}_{q}) \arrowvert = [\mathbb{F}_{q}(\xi): \mathbb{F}_{q}] = per_{\mathbb{Z}_{r}^{\star}}(q) = m
   \end{align*}
   Quindi il polinomio minimo di $\xi$ è di grado $m-1$ ed il sottogruppo di $\mathbb{Z}_{r}^{\star}$ cercato, da mettere in corrispondenza con un sottogruppo di $Gal(\mathbb{F}_{q}(\xi), \mathbb{F}_{q})$ è di ordine $m$.\\
   \emph{Parte 2: esistenza ed unicità di $G$}. Dato che il periodo di un elemento di un gruppo deve dividere l'ordine del gruppo in cui è contenuto, esiste (ed è unico) un sottogruppo di $\mathbb{Z}_{r}^{\star}$ di ordine $per_{\mathbb{Z}_{r}^{\star}}(q) = m$. \\
   \emph{Parte 3: $\phi$ è un isomorfismo}. Le informazioni note a questo punto della dimostrazione sono:
   \begin{itemize}
      \item Esiste $G$ sottogruppo di $\mathbb{Z}_{r}^{\star}$.
      \item $\arrowvert G \arrowvert = m $ periodo di $q$ in $\mathbb{Z}_{r}^{\star}$.
      \item Se $g \in G$ allora $\varphi_{g}(\xi^t) = \xi^{gt}$ è ancora una radice del polinomio minimo di $\xi^{t}$ per $t \in \mathbb{Z}_{r}$.
   \end{itemize}
   Si verifica che
   \begin{align*}
      \psi: G & \longrightarrow  Gal(\mathbb{F}_{q}(\xi), \mathbb{F}_{q}) &   \\
			      g &\longmapsto  \varphi_{g} : \mathbb{F}_{q}(\xi)\longrightarrow  \mathbb{F}_{q}(\xi)\\
					  & \quad \qquad \qquad \quad \xi^{t} \longmapsto \varphi_{g}(\xi^{t}) = \xi^{gt}
      \end{align*}
   è un isomorfismo:\\
   L'iniettività segue dal fatto che se $\psi(g_{1})=\psi(g_{2})$ allora $\xi^{g_{1}} = \xi^{g_{2}}$, cioè $g_{1} = g_{2}$ modulo $r$ e quindi $g_{1} = g_{2}$.\\
   La suriettività è conseguenza del fatto che la cardinalità del dominio è uguale a quella del codominio e dell'iniettività.\\
   È inoltre un omomorfismo di gruppi: siano $g_{1}, g_{2} \in G$ tali che 
   \begin{align*}
      g_{1} g_{2} \equiv k \mod{r}, \quad (k \in G)
   \end{align*}
   Allora 
   \begin{align*}
      \varphi_{g_{1}} \varphi_{g_{2}}(\eta) = \varphi_{g_{1}}(\eta^{g_{2}}) = \eta^{g_{1}g_{2}} = \eta^{k} = \varphi_{k}(\eta) 
   \end{align*}
\end{proof}
\noindent
Riportiamo in un corollario una delle conseguenze della dimostrazione precedente:
\begin{corollario}
   La cardinalità di $G$ coincide con $m = per_{\mathbb{Z}_{r}^{\star}}(q)$.
\end{corollario}

Analogamente a quanto visto nel caso razionale, segue che l'azione definita da
\begin{align*}
 Gal(\mathbb{F}_{q}(\xi), \mathbb{F}_{q})) \times E^{(r)}  & \longrightarrow  E^{(r)}   \\
           (\varphi_{k},\xi^{l}) & \longmapsto \varphi_{k}(\xi^{l}) = \xi^{lk}
\end{align*}
può essere vista come l'azione
\begin{align*}
 G \times \mathbb{Z}_{r} \longrightarrow  \mathbb{Z}_{r}   \\
           (g,l) \longmapsto gl
\end{align*}
semplicemente concentrandosi sugli esponenti delle radici $r$-esime dell'unità che determinano univocamente gli elementi di $E^{(r)}$.

Si può quindi formulare un analogo del teorema \ref{teo:iffPoliMinimoQ} sui campi finiti, la cui dimostrazione è conseguenza di quanto detto precedentemente: 
\begin{teorema} \label{teo:iffPoliMinimoGF}
Sia $G$ definito come sopra. Due elementi $l_{1}$ ed $l_{2}$ di $\mathbb{Z}_{r}$ sono nella stessa orbita della azione 
\begin{align*}
 G \times \mathbb{Z}_{r} &\longrightarrow  \mathbb{Z}_{r}   \\
           (g,l) &\longmapsto gl
\end{align*}
se e solo se $\xi^{l_{1}}$ e $\xi^{l_{2}}$ hanno lo stesso polinomio minimo su $\mathbb{F}$.
\end{teorema}

In generale non è banale trovare il gruppo $G \cong Gal(\mathbb{F}(\xi), \mathbb{F})$ come sottogruppo di $\mathbb{Z}_{r}^{\star}$, ma da quanto visto in precedenza, sappiamo che 
\begin{align*}
   \mathbb{F} = \mathbb{Q} \qquad &\Longrightarrow \qquad G = \mathbb{Z}_{r}^{\star} \\
   \mathbb{F} = \mathbb{F}_{q} \qquad  &\Longrightarrow \qquad G \trianglelefteq \mathbb{Z}_{r}^{\star} 
      \qquad \arrowvert G \arrowvert = m = per_{\mathbb{Z}_{r}^{\star}}(q)
\end{align*}

%%%%%%%%%%%%%%%%%%%%%%%%%%%%%%%%%%%%%%%%%%%%%%%%
%%%%%%%%%%%%%%%%%%%%%% subSEZIONI    %%%%%%%%%%%%%%%%%%%%
\subsection{La fattorizzazione di $x^r - 1$} \label{cap2:fattorizzazione}
Questo paragrafo utilizza quanto visto fino ad ora con lo scopo di fattorizzare
%%%
\footnote{Nel 1967 è stato pubblicato un algoritmo per fattorizzare un polinomio sui campi finti detto algoritmo di Berkelamp\cite{berkelamp}. Questo è ancora utilizzato nell'implementazione di alcuni software, come ad esempio PARI/Gp, sebbene sia stato rimpiazzato nel 1981 dall'algoritmo di Cantor-Zassenhaus \cite{cantor}.}
%%%
il polinomio $x^r - 1$, noto $G \trianglelefteq \mathbb{Z}_{r}^{\star}$ isomorfo a $Gal(\mathbb{F}(\xi), \mathbb{F})$. 
I fattori di $x^r - 1$ sono polinomi minimi di radici $r$-esime coniugate i cui esponenti sono quindi nella stessa orbita dell'azione di $G$ su $\mathbb{Z}_{r}$. Fattorizzare $x^r-1$ equivale a stabilire tali orbite.\\
Completiamo la definizione di orbite già data in \ref{cap2:orbiteq}, nel caso dei razionali, poi dimostriamo il teorema di fattorizzazione di $x^r - 1$. 
\begin{definizione}
Sia $t\in \mathbb{Z}$, $Gal(\mathbb{F}(\xi), \mathbb{F}) \cong G \trianglelefteq \mathbb{Z}_{r}^{\star}$, allora si definisce {\bf $(r,\mathbb{F})$-orbita} di $t$ l'insieme
\begin{align*}
   O_{r,\mathbb{F}}(t) = O(t) = \lbrace gt \mod{r} \mid g \in G \rbrace \subseteq  \mathbb{Z}_{r}
\end{align*}
Le $(r,\mathbb{F})$-orbite sono talvolta dette classi ciclotomiche o impropriamente laterali ciclotomici. Il più piccolo elemento di ogni orbita è detto {\bf etichetta} dell'orbita. \\
Indichiamo con $\mathscr{L}_{r,\mathbb{F}} = \mathscr{L}$ l'insieme delle etichette, e indichiamo con $l_{r,\mathbb{F}} = l = \arrowvert \mathscr{L} \arrowvert$ la sua cardinalità.
Definiamo
\begin{align*}
   m_{r,\mathbb{F}}(t) = m(t) = \arrowvert O(t) \arrowvert
\end{align*}
la cardinalità dell'orbita di $t$, che coincide con il grado del polinomio minimo di $\xi^{t}$.
\end{definizione}

Nei prossimi paragrafi capiterà di considerare vettori, o prodotti di polinomi, o di strutture algebriche i cui elementi (ordinati) hanno una corrispondenza con le etichette contenute nell'insieme $\mathscr{L}$ (non ordinato). Per evitare confusione sull'ordine degli elementi di tali vettori considereremo quando necessario gli elementi di $\mathscr{L}$ ordinati in modo crescente.\\
La notazione
\begin{align*}
   (f_{v})_{v \in \mathscr{L}}
\end{align*}
rappresenta quindi il vettore
\begin{align*}
   (f_{0},f_{1}, \dots , f_{t}, \dots)
\end{align*}
nel quale $f_{t}$ non è necessariamente al $t$-esimo posto per $t$ diverso da $0$ e da $1$, ma corrisponde alla posizione di $t$ in $\mathscr{L}$ ordinato in modo crescente.

\begin{teorema}
   Sia $r$ intero positivo ed $\mathbb{F}$ campo perfetto. 
   \begin{enumerate}
      \item Ad ogni orbita $O(v)$ corrisponde un polinomio irriducibile in $\mathbb{F}[x]$ definito da 
            \begin{align*}
               M^{(v)}(x) =  \prod_{t \in O(v)} (x- \xi^t)
            \end{align*}
      \item La decomposizione in $\mathbb{F}$ di $x^r - 1$ in fattori irriducibili è data da 
            \begin{align*}
               x^r - 1 = \prod_{v \in \mathscr{L} } M^{(v)}(x)
            \end{align*}
   \end{enumerate}
\end{teorema}
\begin{proof}
Il secondo punto è conseguenza del primo, considerando la definizione di fattorizzazione; è allora sufficiente dimostrare il secondo punto.\\
Dato che le orbite sono definite dall'azione del gruppo $G$ su $\mathbb{Z}_{r}$ è noto che sono una partizione di $\mathbb{Z}_{r}$.
Quindi è possibile scrivere
\begin{align*}
   \prod_{t= 0}^{r-1} (x - \xi^{t})  
   = \prod_{v \in \mathscr{L} } \prod_{t \in O(v)} (x - \xi^{t}) 
   = \prod_{v \in \mathscr{L} } M^{(v)}(x)
   =  x^r - 1
\end{align*}
Dal teorema \ref{teo:iffPoliMinimoGF} abbiamo che $M^{(v)}(x)$ è il polinomio minimo che contiene tutti i coniugati di $\xi^{v}$. Quindi è il polinomio minimo di $\xi^{v}$ ed è irriducibile dal lemma \ref{le:poliMinimo}. 
\end{proof}

\begin{osservazione}
   Nel caso particolare di $\mathbb{F} = \mathbb{Q}$, segue che $M^{(0)}(x) = \Phi_{1}(x) = x-1$, ed $M^{(t)}(x) = \Phi_{r/t}(x)$ per ogni $t\mid r$. Infatti $\Phi_{r/t}(x)$ è definito come il polinomio minimo che contiene tutte le radici $r$-esime dell'unità di periodo $t$. 
\end{osservazione}
\begin{esempio}
   Il polinomio $x^8 - 1$ si scompone in $\mathbb{Q}$ come
   \begin{align*}
      x^8 - 1 &= (x-1)(x+1)(x^2 + 1) (x^4+1)\\  
      \Phi_{1}(x) &= (x-1)  = M^{(0)}(x) \\
      \Phi_{2}(x) &= (x+1)= M^{(4)}(x)  \\
      \Phi_{4}(x) &= (x^2 + 1) = M^{(2)}(x) \\
      \Phi_{8}(x) &= (x^4 + 1)  = M^{(1)}(x)
   \end{align*}
\end{esempio}
\noindent
Noti i fattori irriducibili di $x^r-1$ possiamo costruire i suoi divisori come prodotto dei polinomi $M^{(v)}(x)$ per qualche $v \in \mathscr{L}$.
Prima di approfondire questo tema vogliamo rispondere ad una domanda: \\
dati l'intero positivo $r$ ed il campo perfetto $\mathbb{F}$, quante sono le orbite $O_{r,\mathbb{F}}(t)$?


%%%%%%%%%%%%%%%%%%%%%%%%%%%%%%%%%%%%%%%%%%%%%%%%
%%%%%%%%%%%%%%%%%%%%%% SEZIONI    %%%%%%%%%%%%%%%%%%%%
\section{Cardinalità dell'insieme di orbite}

Presentiamo alcune proprietà\footnote{\cite{cerruti} pag. 4 e seguenti. }
sulle orbite dell'azione di $G$ su $\mathbb{Z}_{r}$ (che valgono anche per $Gal(\mathbb{F}(\xi), \mathbb{F})$ su $E^{(r)}$ grazie agli isomorfismi introdotti).
\begin{definizione}
Dato $v$ intero, l'orbita $O(-v)$ è detta {\bf coniugata} ad $O(v)$. Se $O(v) = O(-v)$, allora l'orbita è detta {\bf autoconiugata}.
\end{definizione}
\begin{prop}\label{cap2:5proporbite}
Valgono le seguenti proprietà:
\begin{enumerate}
   \item $O(1) = G$ come insieme.
   \item $m(t) \mid m(1)$ per ogni $t$ in $\mathbb{Z}_{r}$.
   \item $m(t) = m(-t)$ per ogni $t$ in $\mathbb{Z}_{r}$.
   \item $\sum_{t\in \mathscr{L} } m(t) = r$.
\end{enumerate}
\end{prop}
\begin{proof}
Dimostriamo i vari punti dell'asserto separatamente.
\begin{enumerate}
   \item Segue dalla definizione: 
         \begin{align*}
            O(1) = \lbrace g \mod{r} \mid g \in G \rbrace = G \subseteq  \mathbb{Z}_{r}
         \end{align*}
    \item Ricordando la definizione di stabilizzatore dell'elemento $t$
          \begin{align*}
            Stab(t) = \lbrace g \in G \mid gt = t \mod r \rbrace
          \end{align*}
          ed il teorema sulla cardinalità dell'orbita di $t$ rispetto allo stabilizzatore dello stesso elemento
          \footnote{ \cite{cattaneo} Corollario $5.1.17$ pag. 269.}
          \begin{align*}
            \arrowvert O(t)\arrowvert \cdot \arrowvert Stab(t) \arrowvert = \arrowvert G \arrowvert 
          \end{align*}
          segue che 
          \begin{align*}
            \arrowvert O(t)\arrowvert \mid \arrowvert G \arrowvert 
          \end{align*}
          quindi $m(1)$ è multiplo di $m(t)$.
    \item La cardinalità dello stabilizzatore di $t$ è uguale alla cardinalità dello stabilizzatore
          di $-t$, dal fatto che risultano essere lo stesso insieme: 
          \begin{align*}
            Stab(t) &= \lbrace g \in G \mid gt = t \mod r \rbrace   \\
                    &= \lbrace g \in G \mid -gt = -t \mod r \rbrace \\
                    &= \lbrace g \in G \mid g(-t) = -t \mod r \rbrace \\ 
                    &= Stab(-t)
          \end{align*}
          allora dal punto precedente segue che
          \begin{align*}
            \arrowvert Stab(t)\arrowvert \cdot m(t) = m(1) 
            \qquad \qquad
            \arrowvert Stab(-t)\arrowvert \cdot m(-t) = m(1)
          \end{align*}
          Quindi la cardinalità di due orbite coniugate è la stessa:
          \begin{align*}
            m(t) = \frac{m(1)}{\arrowvert Stab(t)\arrowvert} 
            = \frac{m(1)}{\arrowvert Stab(-t)\arrowvert} = m(-t)
          \end{align*}
      \item Le orbite sono a due a due disgiunte e la loro unione costituisce $\mathbb{Z}_{r}$. Formano quindi una partizione di $\mathbb{Z}_{r}$.
\end{enumerate}
\end{proof}

La seguente proprietà caratterizza le orbite autoconiugate:
\begin{prop}
Con le notazioni precedentemente definite, sono equivalenti le seguenti condizioni:
\begin{enumerate}[(a)]
   \item \label{cap2:puntoa} $-1 \in G$.
   \item \label{cap2:puntob} $ O(1)$ è autoconiugata.
   \item \label{cap2:puntoc} Ogni $O(t)$ è autoconiugata.
\end{enumerate}
\end{prop}
\begin{proof}
\ref{cap2:puntoc} $\Rightarrow$ \ref{cap2:puntob}) Se ogni orbita è autoconiugata, allora a fortiori lo è anche $O(1)$. \\
\ref{cap2:puntob} $\Rightarrow$ \ref{cap2:puntoa}) Sia $O(1)$ autoconiugata, allora 
          \begin{align*}
            -1 \in O(-1) = O(1)
          \end{align*}
          Allora dalla proprietà \ref{cap2:5proporbite} $-1 \in O(1) = G$.\\
\ref{cap2:puntoa} $\Rightarrow$ \ref{cap2:puntoc}) Se $-1$ appartiene ad $G$, allora per ogni $t$ in $\mathbb{Z}_{r}$ 
          \begin{align*}
            -1t \in O(t) = \lbrace ht \mod{r} \mid h \in G \rbrace
          \end{align*}
          Ma questo implica che $-t \in O(t)$ e quindi $O(t) = O(-t)$. 
\end{proof}
\begin{osservazione}
   Dal paragrafo \ref{cap2:Grazionali} segue che per $\mathbb{F}=\mathbb{Q}$, allora $G=\mathbb{Z}_{r}^{\star}$, quindi $-1 \in G$ e quindi ogni orbita è autoconiugata. Inoltre l'ordine di $O(t)$ definito come $m(t)$ è dato dal grado dell'$r/t$-esimo polinomio ciclotomico, per $t$ divisore di $r$ e coincide con $\varphi(r/t)$ per $\varphi$ funzione di Eulero. \\
   Quindi la proprietà \ref{cap2:5proporbite} verifica la nota formula $\sum_{t \mid r} \varphi(t) = r$.
\end{osservazione}

Un'interessante applicazione del lemma di Burnside
\footnote{\cite{artin} pag. 196, \cite{cattaneo} pag. 271. }
si esprime nel seguente teorema sulla cardinalità dell'insieme delle orbite
\footnote{\cite{cerruti} pag. 4. }:
\begin{teorema}
Sia $\mathbb{F}$ campo perfetto ed $r$ intero positivo coprimo con la caratteristica di $\mathbb{F}$,
\begin{align*}
   l_{r,\mathbb{F}} = \frac{1}{\arrowvert G \arrowvert }\sum_{g \in G} (g-1, r) 
\end{align*}
\end{teorema}
\begin{proof}
Il lemma di Burnside applicato a questo caso particolare afferma che il numero delle orbite dell'azione del gruppo $G$ sull'insieme $ \mathbb{Z}_{r}$ è dato da 
\begin{align*}
   l = \frac{1}{\arrowvert G \arrowvert }\sum_{g \in G}\arrowvert X_{g} \arrowvert
\end{align*}
dove con $X_{g}$ si indica l'insieme
\begin{align*}
   X_{g} = \lbrace t \in \mathbb{Z}_{r} \mid gt = t \mod r \rbrace 
\end{align*}
La cardinalità di tale insieme equivale al numero di soluzioni della congruenza lineare $ gt \equiv t \mod{r}$, che è proprio il massimo comune divisore fra $g-1$ e $r$.
\end{proof}

\begin{corollario}
Sia $\mathbb{F}= \mathbb{Q}$ ed $r$ intero positivo. Se $d(r)$ rappresenta il numero dei divisori di $r$, allora
\begin{align*}
   \frac{1}{\varphi(r) }\sum_{(g,r) = 1} (g-1, r) = d(r) 
\end{align*}
\end{corollario}
\begin{proof}
Per $\mathbb{F}= \mathbb{Q}$ abbiamo che $G =\mathbb{Z}_{r}^{\star}$, quindi si ha un'orbita per ogni divisore di $r$: $\arrowvert G \arrowvert = \varphi(r)$ e $g \in G$ se e solo se $(g,r) = 1$. Inoltre 
\begin{align*}
   \mathscr{L} = \lbrace g \in \mathbb{Z}_{r}^{\star} : g\mid r \rbrace
   \qquad 
   \arrowvert  \mathscr{L} \arrowvert = d(r)
\end{align*}
da cui segue l'implicazione
\begin{align*}
   l = \frac{1}{\arrowvert G \arrowvert }\sum_{g \in G} (g-1, r) 
   \quad
   \Longrightarrow
   \quad
   d(r) = \frac{1}{\varphi(r) }\sum_{(g,r) = 1} (g-1, r) 
\end{align*}
\end{proof}

Terminiamo il paragrafo, con la dimostrazione di un lemma che sarà utile nel prossimo capitolo.
\begin{lemmax} \label{cap2:lemma110}
Sia $\mathbb{F}$ campo perfetto, $r$ intero positivo e sia $O_{r,\mathbb{F}}(1)$ orbita autoconiugata, allora 
\begin{enumerate}
   \item Se r è dispari
      \begin{align*}
         m(v) = 1 \iff v = 0
      \end{align*}
      e per $v \neq 0$ allora $m(v)$ è pari.
   \item Se r è pari
      \begin{align*}
         m(v) = 1 \iff v = 0 \lor v = r/2
      \end{align*}
      e per $v \neq 0, r/2$ allora $m(v)$ è pari.
\end{enumerate}
\end{lemmax}

\begin{proof}
Dall'ipotesi $O(1)$ autoconiugato segue che ogni orbita è autoconiugata (proprietà \ref{cap2:5proporbite}), quindi per ogni $t$ appartenente a $O(v)$ anche $-t$ appartiene ad $O(v)$. \\ 
Sia $r$ dispari:
\begin{itemize}
   \item[$\Rightarrow$)] Per ipotesi $m(v)=1$. Sia per assurdo $v\neq 0$ e $t\in O(v)$ per $t \in \lbrace 1,2,\dots, r-1\rbrace$, allora 
   \begin{align*}
     t \equiv -t \mod{r} \qquad \qquad 2t \equiv 0 \mod{r}   
   \end{align*}
   in contraddizione con $r$ dispari.\\
   Inoltre $m(v)$ è pari, dato che per ogni $t\neq 0$, $O(v)$ se contiene $t$ contiene anche $-t $ che non è congruo a  $t$ modulo $r$.
   \item[$\Leftarrow$)] Viceversa se $v=0$ allora $O(v) = \lbrace 0 \rbrace$ per definizione.
\end{itemize}
Sia $r$ pari:
\begin{itemize} 
   \item[$\Rightarrow$)] Per ipotesi $m(v)=1$. Sia per assurdo $v\neq 0$ e $t\in O(v)$ per $t \in \lbrace 1,2,\dots, r-1\rbrace$, allora come prima $r\mid 2t$, che accade solo per $t=0$ o per $t=t/2$. Analogamente a prima, per ogni altro valore di $t$, $O(t)$ contiene un numero pari di elementi.
   \item[$\Leftarrow$)] Se $v=0$ allora $m(v)=1$ come già dimostrato, mentre per $v=r/2$ $O(v)$ contiene solo $r/2$.
\end{itemize}
\end{proof}
Conoscere la fattorizzazione di $x^r - 1$ può fornire delle informazioni sull'algebra $\mathcal{R}_{r,\mathbb{F}} = \quotient{\mathbb{F} \lbrack x \rbrack  }{(x^{r} - 1)}$ ? 

%%%%%%%%%%%%%%%%%%%%%%%%%%%%%%%%%%%%%%%%%%%%%%%%
%%%%%%%%%%%%%%%%%%%%%% SEZIONI    %%%%%%%%%%%%%%%%%%%%
\section{Fattorizzazione di $\mathcal{R}$ come prodotto di campi}

Il paragrafo precedente è stato dedicato alla fattorizzazione di $x^r - 1$. Ora analizziamo una delle conseguenze di tale fattorizzazione: la possibilità di esprimere l'algebra $\mathcal{R}_{r,\mathbb{F}}$ come prodotto di campi.
\begin{teorema}\label{teo:teoremaGamma}
   Sia $\mathbb{F}$ campo perfetto ed $r$ intero positivo. Sia 
   \begin{align*}
        x^r-1 = \prod_{v\in \mathscr{L}} M^{(v)}(x) 
   \end{align*}
   per $M^{(v)}(x)$ polinomi irriducibili in $\mathbb{F}$. Allora vale l'isomorfismo di algebre
   \begin{align*}
      \quotient{\mathbb{F} \lbrack x \rbrack  }{(x^{r} - 1)}
      \cong
      \prod_{v\in \mathscr{L}} \quotient{\mathbb{F} \lbrack x \rbrack  }{M^{(v)}(x)}
   \end{align*}
\end{teorema}
\begin{proof}
  La chiave della dimostrazione consiste nel considerare la funzione $\gamma$, che definiamo come {\bf trasformata di Winograd} e che sarà approfondita nei capitoli successivi:
  \begin{align*}
  \gamma :  \quotient{\mathbb{F} \lbrack x \rbrack  }{x^{r} - 1}  
	    & \longrightarrow  
	    \prod_{v\in \mathscr{L}} \quotient{\mathbb{F} \lbrack x \rbrack  }{M^{(v)}(x)}   \\
	    %%
	    a(x) &\longmapsto  ( a(x)\mod{M^{(v)}(x)})_{v\in \mathscr{L}}
  \end{align*}
  È una funzione suriettiva: per il teorema cinese dei resti il sistema $a_{v}(x)\mod{M^{(v)}(x)}$ per $v\in \mathscr{L}$  ammette un'unica soluzione modulo $x^r - 1$ che è la controimmagine cercata.\\
  Si verifica facilmente che è un isomorfismo di spazi vettoriali considerando il codominio come spazio $l$-dimensionale di vettori di polinomi, con il prodotto scalare e la somma usuali.
  Rimane da verificare che è un isomorfismo di anelli.\\
  L'ideale costituito dal nucleo di tale funzione è costituito da tutte le combinazioni di polinomi i cui addendi appartengono a $\lbrace M^{(v)}(x) \rbrace_{v\in \mathscr{L}}$. Quindi è definito da
  \begin{align*}
    ker(\gamma) = \lbrace a(x) \mid a(x) \equiv 0 \mod M^{(v)}(x) \quad \forall v \in \mathscr{L} \rbrace 
    = (\prod_{v\in \mathscr{L}}M^{(v)}(x) )
  \end{align*}
  Osservato questo, la tesi è conseguenza del teorema fondamentale degli isomorfismi di anelli:
  
      \vspace{0.2cm}

      \[
      \begindc{\commdiag}[30]
      %insiemi
      \obj(0,25)[Dom]{$ \quotient{\mathbb{F} \lbrack x \rbrack  }{ (x^r - 1)} $}
      \obj(50,25)[Quoz]{$ \quotient{\quotient{\mathbb{F} \lbrack x \rbrack  }{(x^r-1)} }{ \prod_{v\in \mathscr{L}} M^{(v)}(x)}  $}
      \obj(35,0)[Im]{$ \prod_{v\in \mathscr{L}} \quotient{\mathbb{F} \lbrack x \rbrack  }{M^{(v)}(x)} $}

      %frecce
      \mor{Dom}{Quoz}{$\pi$}
      \mor{Dom}{Im}{$ \gamma$}
      \mor{Quoz}{Im}{$ \cong $}

      \enddc
      \]

      \vspace{0.2cm}
      
    Da otteniamo la tesi osservando che 
    \begin{align*}
       \quotient{\quotient{\mathbb{F} \lbrack x \rbrack  }{(x^r-1)} }{ \prod_{v\in \mathscr{L}} M^{(v)}(x)} =  \quotient{\mathbb{F} \lbrack x \rbrack  }{ (x^r - 1)}
    \end{align*}
    e che $\pi$ è la funzione identità.
\end{proof}
\begin{osservazione}
   Grazie all'aiuto fornito dal teorema cinese dei resti è possibile determinare un'inversa della trasformata di Winograd utilizzando la dimostrazione della suriettività nel teorema precedente. Questo tema sarà approfondita nel capitolo \ref{cap:trasformataWinograd}.
\end{osservazione}
\begin{esempio}
   Consideriamo il polinomio $x^7+x^2+1$ in $\mathcal{R}_{r, \mathbb{F}_q }$, allora
   \begin{align*}
      \gamma(x^7+x^2+1) = (1, x^4 + x^2 + x + 1, 0)
   \end{align*}
   dal fatto che 
   \begin{align*}
      x^9 - 1 &= (x-1)(x^6 + x^3 + 1)(x^2 + x + 1) \\  
       M^{(0)}(x) &= x+1  \\
       M^{(1)}(x) &= x^6 + x^3 + 1 \\
       M^{(3)}(x) &= x^2 + x + 1
   \end{align*}
   e da
   \begin{align*}
    x^7+x^2+1 &\equiv 1 \mod{x+1}  \\
    x^7+x^2+1 &\equiv x^4 + x^2 + x + 1 \mod{x^6 + x^3 + 1} \\
    x^7+x^2+1 &\equiv 0 \mod{x^2 + x + 1}     
   \end{align*}
   La controimmagine del vettore di polinomi $(1, x^4 + x^2 + x + 1, 0)$ si calcola applicando il teorema cinese dei resti al sistema
   \begin{align*}
      a(x) &\equiv 1 \mod{x+1} \\
      a(x) &\equiv x^4 + x^2 + x + 1 \mod{x^6 + x^3 + 1} \\
      a(x) &\equiv 0 \mod{x^2 + x + 1}
   \end{align*}
   da cui si ottiene $a(x) = x^7+x^2+1 $.
\end{esempio}

Il prodotto di campi definito nel teorema precedente sarà indicato, per non appesantire le notazioni con $\mathcal{Q}_{r,\mathbb{F}}$, quindi 
   \begin{align*}
      \mathcal{Q}_{r,\mathbb{F}}
      :=
      \prod_{v\in \mathscr{L}} \quotient{\mathbb{F} \lbrack x \rbrack  }{M^{(v)}(x)}
   \end{align*}
o semplicemente con $\mathcal{Q}$ quando non ci sia bisogno di specificare $r$ ed $\mathbb{F}$. I singoli campi che definiscono il prodotto saranno indicati con $\mathcal{Q}_{r,\mathbb{F}}^{(v)}$ per $v\in \mathscr{L}$:
   \begin{align*}
      \mathcal{Q}_{r,\mathbb{F}}
      =
      \prod_{v\in \mathscr{L}} \mathcal{Q}_{r,\mathbb{F}}^{(v)}
   \end{align*}
\begin{lemmax} \label{le:lemmaIsoMu}
   Sia $\mathbb{F}$ campo perfetto, $r$ intero positivo e $\xi$ radice primitiva $r$-esima dell'unità. Allora per $v \in \mathscr{L}$, l'estensione semplice $\mathbb{F}(\xi^{v})$ è un campo che soddisfa l'isomorfismo 
      \begin{align*}
      \mathbb{F}(\xi^{v})
      \cong
      \quotient{\mathbb{F} \lbrack x \rbrack  }{M^{(v)}(x)}
   \end{align*}
\end{lemmax}
\begin{proof}
   Come già osservato, il quoziente di $\mathbb{F} \lbrack x \rbrack $ su $M^{(v)}(x)$ è un campo dal fatto che $M^{(v)}(x)$ è un polinomio irriducibile. L'isomorfismo è conseguenza immediata del lemma \ref{le:poliMinimo} e del fatto che $M^{(v)}(x)$ è il polinomio minimo di $\xi^{v}$ (e dei suoi coniugati).
\end{proof}
Il lemma appena concluso può rendere interessante l'oggetto definito dal prodotto dei campi $\mathbb{F}(\xi^{v})$ per $v \in \mathscr{L}$. Sarà indicato, per non appesantire la notazione, con $\mathcal{P}_{r,\mathbb{F}}$ o semplicemente con $\mathcal{P}$ se non ci sono ambiguità su $r$ e sul campo $\mathbb{F}$ sui quali è definito:
\begin{align*}
   \mathcal{P}_{r,\mathbb{F}} :=  \prod_{v\in \mathscr{L}} \mathbb{F}(\xi^{v})
\end{align*}
I singoli campi che definiscono il prodotto saranno indicati con $\mathcal{P}_{r,\mathbb{F}}^{(v)}$ per $v\in \mathscr{L}$:
   \begin{align*}
      \mathcal{P}_{r,\mathbb{F}}
      =
      \prod_{v\in \mathscr{L}} \mathcal{P}_{r,\mathbb{F}}^{(v)}
   \end{align*}
   
Proseguiamo il capitolo con un corollario del teorema \ref{teo:teoremaGamma} e del lemma \ref{le:lemmaIsoMu} che specifica la funzione $\mu$ indicata nel diagramma presentato all'inizio del capitolo.
\begin{corollario}
  Con le notazioni precedenti, vale la seguente catena di isomorfismi:
  \begin{align*}
      \mathcal{R}_{r, \mathbb{F}} & := \quotient{\mathbb{F} \lbrack x \rbrack  }{ (x^{r} - 1)} \\
            & = \quotient{\mathbb{F} \lbrack x \rbrack  }{ \prod_{v\in \mathscr{L}} M^{(v)}(x)} \\
            &\cong \prod_{v\in \mathscr{L}} \quotient{\mathbb{F} \lbrack x \rbrack  }{M^{(v)}(x)}  \\ 
            &\cong \prod_{v\in \mathscr{L}} \mathbb{F}(\xi^{v}) =: \mathcal{P}_{r,\mathbb{F}}
   \end{align*}
\end{corollario}
\begin{proof}
   Il primo isomorfismo è quello indotto da $\gamma$ nel diagramma costruito dal teorema fondamentale degli isomorfismi nella dimostrazione del teorema \ref{teo:teoremaGamma}. Il secondo isomorfismo è conseguenza del lemma \ref{le:lemmaIsoMu}: la mappa 
   \begin{align*}
   \mu : \prod_{v\in \mathscr{L}} \mathbb{F}(\xi^{v})   
	    & \longrightarrow  
	    \prod_{v\in \mathscr{L}} \quotient{\mathbb{F} \lbrack x \rbrack  }{M^{(v)}(x)}   \\
	    %%
	    (a(\xi^{v}))_{v \in \mathscr{L}} &\longmapsto  ( a(x)\mod{M^{(v)}(x)})_{v \in \mathscr{L}}
   \end{align*}
   è un isomorfismo di spazi vettoriali ed è un isomorfismo di algebre dato che ogni sua componente 
   \begin{align*}
   \mu^{(v)} :  \mathbb{F}(\xi^{v})   
	    & \longrightarrow  
	    \quotient{\mathbb{F} \lbrack x \rbrack  }{M^{(v)}(x)}   \\
	    %%
	    a(\xi^{v}) &\longmapsto  a(x)\mod{M^{(v)}(x)}
   \end{align*}   
   è indipendente dalle altre ed è un isomorfismo di campi.
\end{proof}

Tornando al diagramma presentato all'inizio del capitolo, anche $\eta$ è un isomorfismo di algebre perché composizione di isomorfismi:
\begin{align*}
\eta : \quotient{\mathbb{F} \lbrack x \rbrack  }{(x^r - 1)}   
	& \longrightarrow  
	 \prod_{v\in \mathscr{L}} \mathbb{F}(\xi^{v})   \\
	%%
	a(x) &\longmapsto  (a(\xi^{v}))_{v\in \mathscr{L}}
\end{align*}
Esiste una struttura algebrica soggiacente a $\mathcal{P}_{r,\mathbb{F}}$ e $\mathcal{Q}_{r,\mathbb{F}}$, sempre isomorfa ad $\mathcal{R}_{r,\mathbb{F}}$ che chiamiamo algebra dei vettori circolanti concatenati. Sarà presentata nel capitolo \ref{cap:trasformataWinograd} con lo scopo di analizzare $\gamma$ come trasformazione lineare nella rappresentazione vettoriale e quindi come matrice di trasformazione.


