
%%%%%%%%%%%%%%%%%%%%%%%%%%%%%%%%%%%%%%%%%%%%%%%%
%%%%%%%%%%%%%%%%%%%%%% CAPITOLo 1    %%%%%%%%%%%%%%%%%%%%
%%%%%%%%%%%%%%%%%%%%%%%%%%%%%%%%%%%%%%%%%%%%%%%%
\chapter{Quattro algebre isomorfe} \label{cap:algebreisomorfe}

Il tema principale della prima parte della tesi riguarda lo studio di alcune strutture algebriche e delle trasformazioni che si possono definire fra loro. In questo capitolo vediamo la struttura algebrica dei polinomi modulo $x^r -1$, delle matrici circolanti, dei vettori circolanti  e delle combinazioni lineari di elementi nel gruppo
ciclico $C_{r}$ definite sul campo $\mathbb{F}$. Lo scopo di questo capitolo è dimostrare che le algebre sopra elencate sono fra di loro isomorfe.

\[
\begindc{\commdiag}[30]
%sotto
\obj(-20,5)[V]{$\mathcal{V}_{r, \mathbb{F}}^{c}$}
\obj(20,5)[M]{$ \mathcal{M}_{r,\mathbb{F} }^{c} $}

%sopra
\obj(-20,30)[R]{$ \quotient{ \mathbb{F}[x] }{ (x^r -1 )} $}
\obj(20,30)[A]{$ \mathbb{F}C_{r} $}


%frecce orizzontali
\mor{V}{M}{$\psi_{1}$}
\mor{R}{A}{$\psi_{4}$}
%frecce verticali
\mor{R}{V}{$\psi_{2}$}
\mor{A}{M}{$\psi_{3}$}


\enddc
\]

\noindent
Nel prossimo capitolo amplieremo il diagramma rappresentando
$\quotient{ \mathbb{F}[x] }{ (x^r -1 )}$ come prodotto di campi tramite la fattorizzazione di $x^r - 1$.

%%%%%%%%%%%%%%%%%%%%%%%%%%%%%%%%%%%%%%%%%%%%%%%%
%%%%%%%%%%%%%%%%%%%%%% Paragrafo  %%%%%%%%%%%%%%%%%%%%
\section{Strutture algebriche} \label{se:struttalg}

%%%%%%%%%%%%%%%%%%%%%% Paragrafo  %%%%%%%%%%%%%%%%%%%%
\subsubsection{Algebra delle matrici circolanti}

Una matrice quadrata è detta circolante se, fissata la prima riga, ogni riga successiva è il risultato di una permutazione ciclica di un posto verso destra della riga precedente.  
\begin{align*}
C = 
\left(
\begin{array} {c c c c }
a_0 & a_1 & \dots & a_{r-1}   \\
a_{r-1} & a_0 & \dots & a_{r-2}   \\
\vdots & \vdots &  & \vdots   \\
a_{1} & a_2 & \dots & a_{0}   \\       
\end{array}
\right)
\end{align*}
L'insieme delle matrici circolanti $r\times r$ a coefficienti nel campo $\mathbb{F}$ è indicato con $\mathcal{M}_{r,\mathbb{F} }^{c}$ e la permutazione ciclica è chiamata {\bf shift}.
\begin{definizione}
Sia $A$ matrice $r \times r$ i cui elementi in $\mathbb{F}$ sono indicati con $(A)_{i,j} = a_{i,j}$. $A$ è
detta {\bf matrice circolante} se  $a_{i,j} = a_{k,l}$ quando 
\begin{align*}
   j-i \equiv l-k \mod{r}
\end{align*}
Sia $\mathbf{a} \in \mathbb{F}^{r}$ vettore corrispondente alla
prima riga di una matrice circolante, allora $\mathbf{a}$ prende il nome di {\bf
vettore circolante} e la matrice circolante completamente determinata da tale
vettore viene indicata con $circ(\mathbf{a})$.  
\end{definizione}

\noindent
Possiamo considerare le matrici circolanti con la struttura algebrica
ereditata
dalle matrici quadrate a coefficienti in $\mathbb{F}$,
quindi $\mathcal{M}_{r,\mathbb{F} }^{c}$ ha struttura di spazio vettoriale di
dimensione $r$ su $\mathbb{F}$ essendo chiusa per la somma ed il prodotto scalare. 

Ciascuna delle strutture algebriche presentate in questo capitolo possiede un
elemento particolare detto {\bf shifter}. Lo shifter delle matrici circolanti è
definito da $s_r$ come:
\begin{align*}
s_r = 
\left(
\begin{array} {c c c c c }
0 & 1 & 0 & \dots & 0   \\
0 & 0 & 1 & \dots & 0   \\
\vdots & \vdots &  & & \vdots   \\
0 & 0 & 0 & \dots & 1   \\
1 & 0 & 0 & \dots & 0   \\     
\end{array}
\right)
&
&
(s_r)_{i,j} =  
\left\lbrace
\begin{array} {c l}
1 & i \equiv j+1 \mod{r}  \\
0 & altrimenti       
\end{array}
\right.
\end{align*}

\noindent
Osserviamo che $s_{r} = circ((0,1,0, \dots, 0))$ e che l'insieme delle potenze
di $s_{r}$ forma un gruppo ciclico di ordine $r$. \\
Ad esempio per $r=3$ si ha
\begin{align*}
s_3 = 
\left(
\begin{array} {c c c  }
0 & 1 &  0   \\
0 & 0 & 1   \\
1 & 0 & 0   \\     
\end{array}
\right)
%%%%%
\qquad
s_3^2 = 
\left(
\begin{array} {c c c  }
0 & 0 & 1   \\
1 & 0 & 0   \\
0 & 1 & 0   \\     
\end{array}
\right)
\qquad
s_3^3 = 
\left(
\begin{array} {c c c  }
1 & 0 & 0   \\
0 & 1 & 0   \\
0 & 0 & 1   \\     
\end{array}
\right)
\end{align*}

\noindent
Quindi ogni matrice circolante è esprimibile
come combinazione lineare degli elementi del gruppo delle potenze di $s_r$:
indicata con $M$ la generica matrice circolante, $M = circ( \mathbf{a} )
$, dove $\mathbf{a} = (a_0, a_1, \dots, a_{r-1} )$, allora
\begin{align*}
M &= a_0  s_{r}^{r} + a_1  s_{r}^{1} + \dots + a_{r-1} s_{r}^{r-1} \\
&= a_0  I_{r} + a_1  s_{r}^{1} + \dots + a_{r-1}  s_{r}^{r-1}
\end{align*}
Possiamo vedere ogni matrice circolante come un
polinomio a coefficienti in $\mathbb{F}$ la cui indeterminata è $s_{r}$
\begin{align*}
\mathcal{M}_{r,\mathbb{F} }^{c} = \lbrace \sum_{j=0}^{r-1} a_{j}s_{r}^{j} \mid
a_{j}
\in \mathbb{F} \rbrace
\end{align*}

\noindent
Alla somma ed al prodotto scalare ereditati dalla struttura matriciale,
aggiungiamo il prodotto fra matrici circolanti, che risulta
ben definito e compatibile con il prodotto scalare dalla possibilità di esprimere
ogni matrice circolante come un polinomio ad indeterminate appartenenti ad un
gruppo ciclico.
Quindi $\mathcal{M}_{r,\mathbb{F}}^{c}$ ha la struttura di algebra.

%%%%%%%%%%%%%%%%%%%%%% Paragrafo  %%%%%%%%%%%%%%%%%%%%

\subsubsection{Algebra dei vettori circolanti}

Dato lo spazio vettoriale $r$-dimensionale sul campo $\mathbb{F}$, i cui
elementi vettoriali hanno convenzionalmente i pedici numerati da $0$ ad $r-1$,
vogliamo definire un prodotto che ne determini la struttura di algebra. Indichiamo con $(\mathbf{a})_{i} = a_{i}$
l'elemento dell'$i$-esimo posto nel vettore, allora per $\mathbf{a}$, $\mathbf{b}$ vettori, definiamo il vettore $\mathbf{a} \star \mathbf{b}$, i cui
elementi sono determinati dalla sommatoria
\begin{align*}
(\mathbf{a} \star \mathbf{b})_{i} = \sum_{j \in \mathbb{Z}_{r} } a_{j} b_{i-j}
\qquad \forall i \in \mathbb{Z}_{r}
\end{align*}
che chiameremo {\bf prodotto di convoluzione} di $\mathbf{a}$ per $\mathbf{b}$.
Possiamo anche definirlo come il prodotto del vettore $\mathbf{a}$ per la matrice
circolante definita da $\mathbf{b}$. Ad esempio per $r = 3$
\begin{align*}
\mathbf{a} \star \mathbf{b} = 
\left(
\begin{array} {c c c }
a_0 & a_1 & a_2    
\end{array}
\right) 
\left(
\begin{array} {c c c}
b_0 & b_1 & b_2   \\
b_2 & b_0 & b_1   \\
b_1 & b_2 & b_0        
\end{array}
\right)
\end{align*}
In questa struttura lo {\bf shifter} è dato dal vettore $(0,1,0,\dots, 0)$,
infatti si verifica immediatamente che 
\begin{align*}
\mathbf{a} \star (0,1,0,\dots, 0) =  (0,1,0,\dots, 0) \star \mathbf{a} = 
(a_{r-1}, a_0,  \dots , a_{r-2} )
\end{align*}
Possiamo verificare che con il prodotto di convoluzione
l'insieme dei vettori non nulli formano un semigruppo, quindi
l'insieme $\mathcal{V}_{r,\mathbb{F} }^{c}$ è un'algebra
detta algebra dei vettori circolanti.

%%%%%%%%%%%%%%%%%%%%%% Paragrafo  %%%%%%%%%%%%%%%%%%%%
\subsubsection{Algebra $\mathbb{F}C_{r}$}
% \begin{center}
%  \fbox{\phantom{  } $\mathbb{K}C_{n}$  \phantom{  } }
% \end{center}
Consideriamo la struttura definita sul gruppo ciclico di ordine
$r$ generato da $g$, indicato con $C_{r} = < g >$, su quale agisce il campo
$\mathbb{F}$. 
Si tratta dell'algebra i cui elementi sono i polinomi di grado
inferiore ad $r$ e la cui indeterminata è il generatore del gruppo ciclico.
\begin{align*}
\mathbb{F}C_{r} = \lbrace
a_0 + a_1 g + a_2 g^2 + \dots + a_{r-1} g^{r-1}
\mid a_{i} \in \mathbb{F}
\rbrace
\end{align*}
%\sum_{i=0}^{r-1} a_{i}g^{i}
Osserviamo che, rappresentando i polinomi come vettori i cui
termini sono i coefficienti delle indeterminate ordinati, la
struttura introdotta può essere scritta come
\begin{align*}
\mathbb{F}C_r = \lbrace (a_0, a_1, \dots, a_{r-1} ) \mid a_{i} \in \mathbb{F}
\rbrace
\end{align*}
Lo {\bf shifter} dell'algebra $\mathbb{F}C_{r}$ è dato dal prodotto per $g$, infatti 
\begin{align*}
g(a_0 + a_1 g + a_2 g^2 + \dots + a_{r-1} g^{r-1} ) 
&= a_0 g + a_1 g^2 + a_2 g^3 + \dots + a_{r-1} g^{r} 
\\
&= a_{r-1} + a_0 g + a_1 g^2 + \dots + a_{r-2} g^{r-1}
\end{align*}
E' noto che la struttura $\mathbb{F}C_{r}$ con il prodotto scalare, la somma
ed il prodotto usuale è un'algebra\footnote{Ad esempio in \cite{jacobson} pag $408$.}.

Nei prossimi capitoli l'algebra $\mathbb{F}C_{r}$ sarà indicato con 
$\mathcal{A}_{r, \mathbb{F}}$ o con $\mathcal{A} $ quando non ci sono
ambiguità su $r$ ed $\mathbb{F}$.

%%%%%%%%%%%%%%%%%%%%%% Paragrafo  %%%%%%%%%%%%%%%%%%%%
\subsubsection{Algebra dei polinomi modulo $x^{r} -1$}

Dato il campo $\mathbb{F}$ definiamo il campo campo quoziente
\begin{align*}
\mathcal{R}_{r, \mathbb{F}} := \quotient{\mathbb{F} \lbrack x \rbrack  }{ x^{r}
- 1} = \lbrace a_0 + a_1 x + a_2 x^2 + \dots + a_{r-1} x^{r-1} \mid a_{j} \in \mathbb{F} \rbrace
\end{align*}
i cui elementi sono polinomi di grado inferiore ad
$r$ e la cui indeterminata soddisfa la relazione 
\begin{align*}
x^r = 1
\end{align*}
Moltiplicando ambo i membri della relazione precedente per $x$ risulta evidente
che l'insieme delle potenze dell'indeterminata è isomorfa al gruppo ciclico di
ordine $r$. La struttura è un'algebra per l'usuale somma e prodotto di
polinomi modulo $x^{r} - 1$. 

Nella prossima sezione vedremo gli isomorfismi fra le algebre appena introdotte. 
Quando non ci saranno ambiguità su $r$ ed $\mathbb{F}$, l'algebra dei polinomi
modulo $x^{r} -1$ sarà indicata con $\mathcal{R}$.


%%%%%%%%%%%%%%%%%%%%%%%%%%%%%%%%%%%%%%%%%%%%%%%%
%%%%%%%%%%%%%%%%%%%%%% Paragrafo  %%%%%%%%%%%%%%%%%%%%

\section{Isomorfismi}

In questa sezione dimostriamo che ogni freccia del diagramma presentato
all'inizio del capitolo è un isomorfismo di algebre, cominciando con il seguente lemma la cui verifica è immediata:
\begin{lemmax}
Ciascuna delle strutture algebriche presentate nel paragrafo precedente è
isomorfa come spazio vettoriale ad $\mathbb{F}^r$.
\end{lemmax}
\noindent
Inoltre ogni omomorfismo di algebre $\psi_{i}$ è completamente determinato
dall'immagine dello shifter. Quindi è sufficiente dimostrare che sono
omomorfismi per il prodotto
affinché risultino isomorfismi di algebre.

%%%%%%%%%%%%%%%%%%%%%%%%%psi1
\begin{prop}
La funzione
\begin{align*}
\psi_{1}: \mathcal{V}_{r, \mathbb{F}}^{c} 
          &\longrightarrow  
          \mathcal{M}_{r,\mathbb{F} }^{c}  \\
              (0,1,0,\dots , 0) 
              &\longmapsto 
              circ(0,1,0,\dots , 0) = s_{r}
\end{align*}
è un isomorfismo di algebre.
\end{prop}

\begin{proof}
Siano $\mathbf{a}$ e $\mathbf{b}$ elementi di 
$\mathcal{V}_{r,\mathbb{F}}^{c}$. L'immagine del loro prodotto tramite
$\psi_{1}$ è definita come
\begin{align*}
\psi_{1}(\mathbf{a} \star \mathbf{b}) 
&= \psi_{1}( 
\sum_{j \in \mathbb{Z}_{r} } a_{j} b_{-j}, 
\sum_{j \in \mathbb{Z}_{r} } a_{j} b_{1-j}, 
\dots ,
\sum_{j \in \mathbb{Z}_{r} } a_{j} b_{r-1-j}) \\
&=circ(
\sum_{j \in \mathbb{Z}_{r} } a_{j} b_{-j}, 
\sum_{j \in \mathbb{Z}_{r} } a_{j} b_{1-j}, 
\dots ,
\sum_{j \in \mathbb{Z}_{r} } a_{j} b_{r-1-j}
) 
\end{align*}
Da cui segue che
\begin{align*}
\psi_{1}(\mathbf{a} \star \mathbf{b}) 
&=
\left(
\begin{array} {c c c c}
\sum_{j \in \mathbb{Z}_{r} } a_{j} b_{-j} 
& 
\sum_{j \in \mathbb{Z}_{r} } a_{j}b_{1-j} 
&  
\dots 
& 
\sum_{j \in \mathbb{Z}_{r} } a_{j} b_{r-1-j} \\
\sum_{j \in \mathbb{Z}_{r} } a_{j} b_{r-1-j} 
& 
\sum_{j \in \mathbb{Z}_{r} } a_{j} b_{-j} 
& 
\dots 
&   
\sum_{j \in \mathbb{Z}_{r} } a_{j} b_{r-2-j} \\
\vdots &  &  & \vdots   \\
\sum_{j \in \mathbb{Z}_{r} } a_{j} b_{1-j} 
& 
\sum_{j \in \mathbb{Z}_{r} } a_{j} b_{2-j} 
& 
\dots 
& 
\sum_{j \in\mathbb{Z}_{r} } a_{j} b_{-j}   
\end{array}
\right)
\\
&=
\left(
\begin{array} {c c c c}
a_{0} 
& 
a_{1}
&  
\dots 
& 
a_{r-1} \\
a_{r-1} 
& 
a_{0} 
& 
\dots 
&   
a_{r-2}  \\
\vdots &  &  & \vdots   \\
a_{1}  
& 
a_{2}  
& 
\dots 
& 
a_{0}   
\end{array}
\right)
\left(
\begin{array} {c c c c}
b_{0} 
& 
b_{1}
&  
\dots 
& 
b_{r-1} \\
b_{r-1} 
& 
b_{0} 
& 
\dots 
&   
b_{r-2}  \\
\vdots &  &  & \vdots   \\
b_{1}  
& 
b_{2}  
& 
\dots 
& 
b_{0}   
\end{array}
\right)    \\
&= circ(\mathbf{a}) circ(\mathbf{b})  
\\
&= \psi_{1}(\mathbf{a})\psi_{1}(\mathbf{b}) 
\end{align*}
Quindi le algebre $\mathcal{V}_{r, \mathbb{F}}^{c}$ e $\mathcal{M}_{r,\mathbb{F}
}^{c} $ sono isomorfe.
         
\end{proof}

         

%%%%%%%%%%%%%%%%%%%%%%%%%psi2
\begin{prop}
La funzione
\begin{align*}
\psi_{2}: \quotient{ \mathbb{F}[x] }{ (x^r -1 )} 
          &\longrightarrow  
          \mathcal{V}_{r,\mathbb{F} }^{c}  \\
              x 
              &\longmapsto 
              (0,1,0,\dots , 0) 
\end{align*}
è un isomorfismo di algebre.
\end{prop}

\begin{proof}
Siano $a(x) = \sum_{j \in \mathbb{Z}_{r} }a_{j} x^{j}$ e $b(x)= \sum_{j \in
\mathbb{Z}_{r} } b_{j} x^{j}$ elementi di $\mathcal{R}_{r, \mathbb{F}} $.
Esaminiamo l'immagine del loro prodotto
\begin{align*}
\psi_{2}(a(x)b(x)) 
&= \psi_{2}( 
a_{0}(\sum_{j \in \mathbb{Z}_{r} } b_{j} x^{j}) +   
%(b_{0} + b_{1}x + \dots + b_{r-1}x^{r-1}) +
a_{1}x(\sum_{j \in \mathbb{Z}_{r} } b_{j} x^{j}) +
%(b_{0} + b_{1}x + \dots + b_{r-1}x^{r-1}) +
\dots
a_{r-1}x^{r-1}(\sum_{j \in \mathbb{Z}_{r} } b_{j} x^{j})
%(b_{0} + b_{1}x + \dots + b_{r-1}x^{r-1}) 
)
\end{align*}
come visto nella sezione precedente, moltiplicare un polinomio di
$\mathcal{R}_{r, \mathbb{F}}$ per $x^k$ equivale ad effettuare uno shift sui
suoi coefficienti
di $k$ posti verso destra. Quindi
\begin{align*}
\psi_{2}(a(x)b(x)) 
&= \psi_{2}( 
a_{0}(\sum_{j \in \mathbb{Z}_{r} } b_{j} x^{j}) +   
%(b_{0} + b_{1}x + \dots + b_{r-1}x^{r-1}) +
a_{1}(\sum_{j \in \mathbb{Z}_{r} } b_{j+1} x^{j}) +
%(b_{0} + b_{1}x + \dots + b_{r-1}x^{r-1}) +
\dots
a_{r-1}(\sum_{j \in \mathbb{Z}_{r} } b_{j+r-1} x^{j})
%(b_{0} + b_{1}x + \dots + b_{r-1}x^{r-1}) 
) 
\\
&=
\psi_{2}(
\sum_{j \in \mathbb{Z}_{r} } a_{j} b_{-j} +
(\sum_{j \in \mathbb{Z}_{r} } a_{j} b_{1-j})x +
\dots +
(\sum_{j \in \mathbb{Z}_{r} } a_{j} b_{r-1-j})x^{r-1}) \\
&=\psi_{2}(a(x)) \star \psi_{2}(b(x))
\end{align*}
Quindi le algebre $\mathcal{R}_{r, \mathbb{F}}$ e $\mathcal{V}_{r,
\mathbb{F}}^{c}$ sono isomorfe.
\end{proof}
%%%%%%%%%%%%%%%%%%%%%%%%%%%
%%% esempio
Ad esempio per $r = 3$, rappresentando i polinomi di $\mathcal{R}_{3,
\mathbb{F}} $ come vettori
\begin{align*}
 a(x) = a_{0} + a_{1}x+ a_{2}x^2 = (a_{0}, a_{1},a_{2}) 
 \\
 b(x) = b_{0} + b_{1}x+ b_{2}x^2 = (b_{0}, b_{1},b_{2})
\end{align*}
si può considerare il prodotto convolutivo di $\psi_{2}(a(x))$
per $\psi_{2}(b(x))$ come
\begin{align*}
&\psi_{2}(a(x)) \star \psi_{2}(b(x)) 
=
\left(
\begin{array} {c c c }
a_0 & a_1 & a_2    
\end{array}
\right) 
\left(
\begin{array} {c c c}
b_0 & b_1 & b_2   \\
b_2 & b_0 & b_1   \\
b_1 & b_2 & b_0        
\end{array}
\right) 
\\
&=
(
a_{0}b_{0} + a_{1}b_{2} + a_{2}b_{1}, 
a_{0}b_{1} + a_{1}b_{0} + a_{2}b_{2},
%\\
%&
a_{0}b_{2} + a_{1}b_{1} + a_{2}b_{0}
)
\\
&=
\psi_{2}
( 
a_{0}b_{0} + a_{1}b_{2} + a_{2}b_{1} +  
(a_{0}b_{1} + a_{1}b_{0} + a_{2}b_{2})x +
(a_{0}b_{2} + a_{1}b_{1} + a_{2}b_{0})x^2
) 
\\
&=
\psi_{2}((a_{0} + a_{1}x+ a_{2}x^2)(b_{0} + b_{1}x+ b_{2}x^2)) 
\\
&=
\psi_{2}(a(x)b(x))
\end{align*}


Proseguiamo nel dimostrare che $\psi_{3}$ e $\psi_{4}$ sono isomorfismi di
algebre.

%%%%%%%%%%%%%%%%%%%%%%%%%psi3
\begin{prop}
La funzione
\begin{align*}
\psi_{3}: \mathbb{F}C_{r}
          &\longrightarrow  
          \mathcal{M}_{r,\mathbb{F} }^{c}  \\
              g 
              &\longmapsto 
              s_{r}
\end{align*}
è un isomorfismo di algebre.
\end{prop}

\begin{proof}
Siano $a$, $b$ elementi del algebra $\mathbb{F}C_{r}$ della forma
\begin{align*}
 a = a_0 + a_1 g + a_2 g^2 + \dots + a_{r-1} g^{r-1} 
\\
 b = b_0 + b_1 g + b_2 g^2 + \dots + b_{r-1} g^{r-1}
\end{align*}
Allora l'immagine del loro prodotto tramite $\psi_{3}$ è data da
\begin{align*}
\psi_{3}(ab) 
&= \psi_{3}( 
\sum_{j \in \mathbb{Z}_{r} } a_{j} b_{-j} + 
(\sum_{j \in \mathbb{Z}_{r} } a_{j} b_{1-j})g + 
\dots +
(\sum_{j \in \mathbb{Z}_{r} } a_{j} b_{r-1-j})g^{r-1} )
\\
&= circ
(
\sum_{j \in \mathbb{Z}_{r} } a_{j} b_{-j} + 
(\sum_{j \in \mathbb{Z}_{r} } a_{j} b_{1-j})g + 
\dots +
(\sum_{j \in \mathbb{Z}_{r} } a_{j} b_{r-1-j})g^{r-1}
) 
\\
&=
circ(a_0, a_1, a_2, \dots , a_{r-1})
circ( b_0, b_1, b_2 , \dots , b_{r-1} )
\\
&= \psi_{3}(a) \psi_{3}(b) 
\end{align*}
Quindi le algebre $\mathbb{F}C_{r}$ e 
$\mathcal{M}_{r,\mathbb{F} }^{c}$ sono
isomorfe.
\end{proof}

%%%%%%%%%%%%%%%%%%%%%%%%%psi4
\begin{prop}
La funzione
\begin{align*}
\psi_{4}: \quotient{ \mathbb{F}[x] }{ (x^r -1 )} 
          &\longrightarrow  
          \mathbb{F}C_{r} \\
              x 
              &\longmapsto 
              g
\end{align*}
è un isomorfismo di algebre.
\end{prop}

\begin{proof}
Siano $a(x) = \sum_{j \in \mathbb{Z}_{r} }a_{j} x^{j}$ e $b(x)= \sum_{j \in
\mathbb{Z}_{r} } b_{j} x^{j}$ elementi di $\mathcal{R}_{r, \mathbb{F}} $.
Esaminiamo allora l'immagine del loro prodotto tramite $\psi_{4}$:
\begin{align*}
\psi_{4}(a(x)b(x)) 
&= \psi_{4}( 
a_{0}(\sum_{j \in \mathbb{Z}_{r} } b_{j} x^{j}) +  
a_{1}x(\sum_{j \in \mathbb{Z}_{r} } b_{j} x^{j}) +
\dots
a_{r-1}x^{r-1}(\sum_{j \in \mathbb{Z}_{r} } b_{j} x^{j})
)
\\
&= 
\psi_{4}( 
a_{0}(\sum_{j \in \mathbb{Z}_{r} } b_{j} x^{j}) +  
a_{1}(\sum_{j \in \mathbb{Z}_{r} } b_{j+1} x^{j}) +
\dots
a_{r-1}(\sum_{j \in \mathbb{Z}_{r} } b_{j+r-1} x^{j})
) 
\\
&=
\sum_{j \in \mathbb{Z}_{r} } a_{j} b_{-j} +
(\sum_{j \in \mathbb{Z}_{r} } a_{j} b_{1-j})g +
\dots +
(\sum_{j \in \mathbb{Z}_{r} } a_{j} b_{r-1-j})g^{r-1}
\\
&=
\psi_{4}(a(x))\psi_{4}(b(x))
\end{align*}
\end{proof}

Abbiamo quindi quattro algebre isomorfe ciascuna delle quali possiede un
elemento particolare, che genera il gruppo su cui le algebre sono definite e il
cui prodotto con uno degli elementi della sua algebra è uno shift dei coefficienti sugli elementi della base.
\begin{center}
\begin{tabular}{ c | c c c c }
Strutture 
& 
$\mathcal{V}_{r, \mathbb{F}}^{c}$ 
& 
$\mathcal{M}_{r,\mathbb{F} }^{c} $ 
& 
$\mathbb{F}C_{r} $ 
& 
$\mathcal{R}_{r, \mathbb{F}} $ 
\\
\hline
Shifter & $(0,1,0,\dots,0)$ & $s_n$ & $g$ & $x$ 
\end{tabular}
\end{center}
Per $r$ ed $\mathbb{F}$ generici $\mathcal{R}_{r, \mathbb{F}} $ non è un campo,
quindi non tutti i suoi elementi possiedono inverso. È possibile scomporre
tale struttura in un prodotto di campi? 
Affrontiamo il problema nel prossimo capitolo.


% 
% 
% 
% %%%%%%%%%%%%%%%%%%%%%%%%%%%%%% LEMMA
% \begin{lemmax}
% Sia $\omega = e^{\frac{2\pi i}{n}}$ radice primitiva $n-$esima dell'unità,
% $C_{n}$ gruppo ciclico generato da $g$, allora 
% $$
%  \mathbb{K}\lbrack \omega \rbrack   \cong \mathbb{K} C_{n} 
% $$
% \end{lemmax}
% 
% \begin{proof}
% La funzione
% \begin{align*}
% \delta:\mathbb{K} \lbrack \omega \rbrack  &\longrightarrow  \mathbb{K} C_{n}  
% \\
% a_0 + a_1\omega + a_2\omega^2 + \dots + a_{n-1}\omega^{n-1}    &\longmapsto a_0
% + a_1 g + a_2 g^2 + \dots + a_{n-1} g^{n-1}
% \end{align*}
% è un isomorfismo di algebre, infatti è ben definito, è isomorfismo di spazi
% vettoriali e per il prodotto, i gruppi soggiacenti sono isomorfi e l'immagine di
% un generatore è un generatore.
% \end{proof}
% 
% 
% %%%%%%%%%%%%%%%%%%%
% \begin{prop}
% $$
%  circ_{n}(\mathbb{K}) \cong \mathfrak{C}_{\mathbb{K},n} 
% $$
% \end{prop}
% 
% \begin{proof}
% Sia 
% \begin{align*}
% \alpha: circ_{n}(\mathbb{K}) &\longrightarrow   \mathfrak{C}_{\mathbb{K},n}   \\
%    \mathbf{a} &\longmapsto circ(\mathbf{a})
% \end{align*}
% funzione che manda il vettore circolante nella matrice circolante
% corrispondente. E' un isomorfismo di spazi vettoriali dato che sia il dominio
% che il codominio sono isomorfi a $\mathbb{K}^{n}$ come spazi vettoriali.
% Inoltre si verifica facilmente
% \begin{align*}
% \alpha(\mathbf{a} \star \mathbf{b} ) = \alpha(\mathbf{a}) \alpha(\mathbf{b})
% \end{align*}
% dato che il prodotto di convoluzione coincide con la prima riga del prodotto
% delle due matrici circolanti corrispondenti.
% \end{proof}
% 
% 
% 
% %%%%%%%%%%%%%%%
% \begin{prop}
%  $$
%  \mathfrak{C}_{\mathbb{K},n} \cong  \mathbb{K}C_{n} 
% $$
% \end{prop}
% 
% \begin{proof}
% Sia $C_{n} = <g>$:
% \begin{align*}
% \beta: \mathfrak{C}_{\mathbb{K},n} &\longrightarrow   \mathbb{K}C_{n}    \\
%    circ(a_0, a_1, \dots , a_{n-1}) &\longmapsto a_0 + a_1 g + \dots + a_{n-1}
% g^{n-1}
% \end{align*}
% funzione che manda la matrice circolante nel polinomio a coefficienti
% corrispondenti sull'indeterminata che genera $C_{n}$. E' ben definita dato che
% $g^{0}, g^{1} , \dots , g^{n-1}$ sono distinti; come nel caso precedente è un
% isomorfismo di spazi vettoriali dato che sia il dominio che il codominio sono
% isomorfi a $\mathbb{K}^{n}$ come spazi vettoriali.
% Per il prodotto si può procedere in due modi: calcolando direttamente $\beta(
% circ(\mathbf{a})circ(\mathbf{b}))$ oppure considerando gli elementi di
% $\mathfrak{C}_{\mathbb{K},n}$ come polinomi di $\mathbb{K} \lbrack s_n \rbrack$
% ed applicando il lemma.
% \end{proof}
% 
% %%%%%%%%%%%%%%%%%%
% \begin{prop}
% $$
% \mathbb{K}C_{n}  \cong  \mathcal{R}_{\mathbb{K},n}  
% $$
% \end{prop}
% 
% \begin{proof}
%  Sia $C_{n} = <g>$:
% \begin{align*}
% \gamma: \mathbb{K}C_{n} &\longrightarrow  \mathcal{R}_{\mathbb{K},n}    \\
%      a_0 + a_1 g + \dots + a_{n-1} g^{n-1} &\longmapsto a_0 + a_1 x + \dots +
% a_{n-1} x^{n-1}
% \end{align*}
% funzione che manda il polinomio nell'indeterminata $g$ nel corrispondente ad
% indeterminata $x$. Osservando che in $\mathcal{R}_{\mathbb{K},n}$ le
% indeterminate formano un gruppo ciclico di ordine $n$, dato che $x^n = 1$ allora
% $gamma$ è un isomorfismo di algebre. Inftti è ben definito, è isomorfismo di
% spazi vettoriali, per il prodotto i gruppi soggiacenti che ne determinano la
% base sono isomorfi e l'immagine di un generatore è un generatore.
% \end{proof}
% 
% 
% %%%%%%%%%%%%%%%%%
% \begin{corollario}
% $$
% \mathcal{R}_{\mathbb{K},n}   \cong  circ_{n}(\mathbb{K})
% $$
% \end{corollario}
% 
% \begin{proof}
%  Conseguenza del diagramma presentato all'inizio del paragrafo e delle ultime
% proprietà.
% \end{proof}
% 
% 
% \subsubsection{Shifters} 
% Ciascuna delle strutture precedentemente esaminate possiede un elemento
% particolare, detto {\bf shift} il cui prodotto corrisponde allo spostamento di
% un posto verso destra degli elementi del vettore o della prima riga della
% matrice circolante o dei coefficienti del polinomio. La situazione è riassunta
% dalla seguente tabella: \\
% 
% \begin{center}
% \begin{tabular}{ c | c c c c }
% Strutture & $circ_{n}(\mathbb{K})$ & $\mathfrak{C}_{\mathbb{K},n} $ & $
% \mathbb{K}C_{n}  $ & $\mathcal{R}_{\mathbb{K},n}$ \\
% \hline
% Shifter & $(0,1,0,\dots,0)$ & $s_n$ & $g$ & $x$ 
% \end{tabular}
% \end{center}
% 
% 
% 
% 
% 

%%%%%%%%%%%%%%%%%%%%%%%%%%%%%%%%%%%%%%%%%%%%%%%%
%%%%%%%%%%%%%%%%%%%%%% CAPITOLI     %%%%%%%%%%%%%%%%%%%%




% pre riassumere, usiamo gli shifters
%ovviamente l'immagine dello shifter è uno shifter.



%%%%%%%%%%%%%%%%%%%%%%%%%%%%%%%%%%%%%%%%%%%%%%%%
%%%%%%%%%%%%%%%%%%%%%% Paragrafo  %%%%%%%%%%%%%%%%%%%%





