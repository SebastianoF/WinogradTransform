
%\appendixname
\chapter{Appendix}

%%%%%%%%%%%%%%%%%%%%%%%%%%%%%%%%%%%%%%%%%%%%%%%%
%%%%%%%%%%%%%%%%%%%%%%%%%%%%%%%%%%%%%%%%%%%%%%%%
%%%%%%%%%%%%%%%%%%%%%% Section    %%%%%%%%%%%%%%%%%%%%

\section{Recipe: transform a sentence into a list of binary numbers and backwards}

Given an alphabet $\mathcal{A}$, that is an array of symbols (typically letters, though any kind of symbols would work), we want to transform any sequence of its elements (or a sentence) into an array of binary numbers. \\

\noindent
Consider
\begin{align}
   \mathcal{A} = [ ``~~", ``n", ``a", ``u", ``y", ``l", ``a", ``f", ``m" ]
\end{align}
where the first element is the empty space, and consider the sentence
$$
   ``my~funny~val".
$$

\noindent
We first replace the sentence with the sequence of the numbers corresponding to the positions of the letters in the list $\mathcal{A}$:
\begin{align*}
   ``my~funny~val" \longrightarrow [8, 4, 0, 7, 3, 1, 1, 4, 0, 6, 2, 5].
\end{align*}
Then we chose a length to group the numbers in the array together. Here we chose 5, and we pad the last two numbers with zeros to have all the sub-lists of the same length:
\begin{align*}
   [8, 4, 0, 7, 3, 1, 1, 4, 0, 6, 2, 5] \longrightarrow [[8, 4, 0, 7, 3], [1, 1, 4, 0, 6], [2, 5, 0, 0, 0]].
\end{align*}
Each one of the sub-list is called here a \emph{worm}, and its length \emph{worm length}. Now we must convert a worm into an integer, that will then be represented into its binary form. To this end we use the length of the alphabet as the base for the number:
\begin{align*}
   [8, 4, 0, 7, 3]
   \longrightarrow
   8 + 4 \vert \mathcal{A}\vert + 0 \vert \mathcal{A}\vert^2 + 7\vert \mathcal{A}\vert^3 + 3\vert \mathcal{A}\vert^4 = 24830
\end{align*}
And in general:
\begin{align}\label{eq:worm_to_int}
   [w_1, w_2, \dots, w_{\text{w}}]
   \longrightarrow
   \sum_{j=0}^{\text{w} - 1} w_j \vert \mathcal{A}\vert^j
\end{align}
We can then convert all the worms into:
\begin{align*}
   [[8, 4, 0, 7, 3], [1, 1, 4, 0, 6], [2, 5, 0, 0, 0]]
   \longrightarrow
   [24830, 39700, 47].
\end{align*}
And then all the integers into their binary representation:
\begin{align*}
   [24830, 39700, 47] \longrightarrow [0110000011111110, 1001101100010100, 0000000000101111]
\end{align*}
Where the binaries numbers have been padded with zeros so to have all the same length (the maximal length is the length of the binary representation of the maximal number of that can be obtained in this system, which is $\vert \mathcal{A}\vert^{\text{w}} - 1$).

To go backward, the only critical passage is the inverse of \ref{eq:worm_to_int}, transforming an integer $n$ into the corresponding worm $w$. This can be done recursively with:
\begin{align*}
   &w[0] := n ~\text{mod}~ \vert\mathcal{A}\vert \\
   &\text{for}~i=1\dots \text{w}-1\\
   &~~~~ n := \frac{n - w[i-1]}{\vert\mathcal{A}\vert} \\
   &~~~~ w[i] := n ~\text{mod}~ \vert\mathcal{A}\vert
\end{align*}
where $n ~\text{mod}~ \vert\mathcal{A}\vert$ is the reminder of $n$ modulo n $\vert\mathcal{A}\vert$.

% \section{Recipe to find minimal polynomials over finite fields}
% Nella pratica, per trovare i polinomi minimi sui campi finiti, esiste un procedimento che non implica la ricerca del gruppo $H$ isomorfo al gruppo di Galois $Gal(\mathbb{F}(\xi), \mathbb{F})$ presentato nel capitolo \ref{cap:fattorizzazione}, e che risulta essere quindi più maneggevole per le implementazioni\footnote{Come proposto ad esempio in \cite{cattaneo} pag.83.}. Si riconduce alla seguente definizione che riformula la definizione di elementi coniugati, classi ciclotomiche e polinomio minimo.
% \begin{definizione}
%    Sia $\xi \in \mathbb{F}_{q^m}$ non nullo e sia $t$ il più piccolo intero positivo tale che $\xi^{p^{t}} = \xi$. Allora l'insieme
%    \begin{align*}
%       C(\xi) = \lbrace \xi, \xi^{p}, \xi^{p^{2}}, \dots ,\xi^{p^{t-1}} \rbrace
%    \end{align*}
%    è detto {\bf classe ciclotomica} e due elementi di tale insieme sono detti {\bf coniugati}.\\
%    Si definisce {\bf polinomio minimo} di $\xi$ il più piccolo polinomio in $\mathbb{F}_{q}$ che ammette $\xi$ come radice.
% \end{definizione}
% \noindent
% Con questa definizione si dimostra\footnote{Da \cite{cattaneo}, teoremi $4.36$ e $4.38$.} che il polinomio minimo di un elemento di $\mathbb{F}_{q}$ non nullo è il più piccolo polinomio che contiene tutti gli elementi di una stessa classe ciclotomica:
% \begin{teorema}
%    Se $\xi$ è un elemento non nullo di $\mathbb{F}_{q}$ allora il suo polinomio minimo $M_{\xi}(x)$ è un polinomio irriducibile in $\mathbb{F}_{q}[x]$ ed è definito come
%    \begin{align*}
%       M_{\xi}(x) = \prod_{\beta \in C(\xi)} (x-\beta)
%    \end{align*}
% \end{teorema}
% \begin{proof}
%    Se per assurdo $M_{\xi}(x) $ polinomio minimo di $\xi$ non è un polinomio irriducibile allora può essere scritto come prodotto di due polinomi in $\mathbb{F}_{q}[x]$ $M_{\xi}(x) = f_{1}(x) f_{2}(x)$ con $0 \leq deg(f_{j}(x)) < deg(M_{\xi}(x))$. Dato che $M_{\xi}(\xi) = f_{1}(\xi) f_{2}(\xi) = 0 $ e dato che $\mathbb{F}_{q}$ è un campo, allora uno dei due polinomi della fattorizzazione di $M_{\xi}(x)$ ammette $\xi$ come radice in contraddizione con la minimalità di $M_{\xi}(x)$.\\
%    Rimane da dimostrare che il polinomio minimo è proprio determinato dal prodotto $ \prod_{\beta \in C(\xi)} (x-\beta)$. \\
%    Sia $M_{\xi}(x) = \sum_{j=0}^{t} m_{j}x^{j}$ definito come polinomio minimo di $\xi$. Dato che se $\xi$ è una sua radice, lo deve necessariamente essere anche $\xi^{p}$:
%    \begin{align}\label{app:eq:alfaEalfa}
%        M_{\xi}(\xi^{p})
%        &= \sum_{j=0}^{t} m_{j}(\xi^{p})^{j} = \sum_{j=0}^{t} m_{j}^{p}(\xi^{p})^{j} \\
%        &= \sum_{j=0}^{t} (m_{j}\xi^{j})^{p} = (\sum_{j=0}^{t} m_{j}\xi^{j})^{p} \\
%        &= (M_{\xi}(\xi))^{p} = 0
%    \end{align}
%    inoltre per la sua minimalità $M_{\xi}(x)$ non ammette altre radici. Quindi
%    \begin{align*}
%       M_{\xi}(x) = \prod_{\beta \in C(\xi)} (x-\beta)
%    \end{align*}
%    La \ref{app:eq:alfaEalfa} è valida solo se si può affermare che $m_{j}^{p} = m_{j}$ cosa che accade solo dimostrando che $m_{j} \in \mathbb{F}_{q}$. Questo completa la dimostrazione: \\
%    da un lato segue che
%    \begin{align*}
%        (M_{\xi}(x))^{p}
%        &= \prod_{\beta \in C(\xi)} (x-\beta)^{p} = \prod_{\beta \in C(\xi)} (x^{p} -\beta^{p}) \\
%        &= \prod_{\beta \in C(\xi)} (x^{p} -\beta) = (M_{\xi}(x^{p})) = \sum_{j=0}^{t} m_{j}x^{jp}
%    \end{align*}
%    e d'altra parte
%    \begin{align*}
%        (M_{\xi}(x))^{p} =  \sum_{j=0}^{t} (m_{j}x^{j})^{p} =  \sum_{j=0}^{t} m_{j}^{p}x^{jp}
%    \end{align*}
%    Pertanto $m_{j}^{p} = m_{j}$ ed $m_{j} \in  \mathbb{F}_{q}$.
% \end{proof}

% %%%%%%%%%%%%%%%%%%%%%%%%%%%%%%%%%%%%%%%%%%%%%%%%
% %%%%%%%%%%%%%%%%%%%%%%%%%%%%%%%%%%%%%%%%%%%%%%%%
% %%%%%%%%%%%%%%%%%%%%%% SEZIONE    %%%%%%%%%%%%%%%%%%%%
% \section{Ricorrenze lineari}

% Accenniamo al rapporto fra l'algebra $\mathcal{R}_{r,q}$ e le ricorrenze lineari cominciando con un esempio:
% \begin{esempio}\label{ese:ricorr1}
%    Dato il polinomio $s(x) = x^4 + x+ 1$ nell'algebra $\mathcal{R}_{7,2}$, indichiamo con $\mathscr{F}_{s(x)}$ un vettore di lunghezza infinita verso destra costruito ripetendo $\psi_{2}(s(x))$ senza alterazioni:
%    \begin{align*}
%       \mathscr{F}_{s(x)} = (1,1,0,0,1,0,0 | 1,1,0,0,1,0,0  | 1,1,0,0,1,0,0 | 1,1,0, \dots )
%    \end{align*}
%    Indicando con $\mathscr{F}_{s(x)}(j)$, per $j$ intero non negativo, il suo $j$-esimo elemento, allora $\mathscr{F}_{s(x)}$ è caratterizzato dall'equazione
%    \begin{align*}
%       \mathscr{F}_{s(x)} (j) = \mathscr{F}_{s(x)} (j-7) \qquad \forall j \geq 7
%    \end{align*}
%    Variando $s(x)$ in $\mathcal{R}_{7,2}$, l'insieme costituito da tutti i vettori di lunghezza infinita $\mathscr{F}_{s(x)}$ è uno spazio vettoriale isomorfo a $\mathbb{F}_{2}^{7}$ ed è un'algebra isomorfa a $\mathcal{R}_{7,2}$ se dotata del prodotto di convoluzione in modo naturale.\\
%    Da un altro punto di vista, il vettore $\psi_{2}(s(x))$ definisce una funzione da $\mathbb{Z}_{7}$ in $\mathbb{F}_{2}$ che associa ad ogni $j$ in $\mathbb{Z}_{7}$ il $j$-esimo elemento del vettore $\psi_{2}(s(x))$, indicato con $s_j$. Tale funzione può essere considerata anche da $\mathbb{Z}$ in $\mathbb{F}_{2}$; in questo caso a $j \in \mathbb{Z}$ è associato il $j$-esimo elemento di una stringa infinita anche verso sinistra costruita ripetendo il vettore $\psi_{2}(s(x))$.\\
%    Se indichiamo tale stringa con $\mathscr{I}_{s(x)}$ ed il suo $j$-esimo elemento con $\mathscr{I}_{s(x)}(j)$ allora abbiamo
%    \begin{align*}
%       \mathscr{I}_{s(x)} &= ( \dots 0,0 | 1,1,0,0,1,0,0  | 1,1,0,0,1,0,0 | 1,1,0, \dots ) \\
%       j \longmapsto & \mathscr{I}_{s(x)}(j) = \mathbf{s}_{j \mod{7}}
%    \end{align*}
%    Come prima, considerando l'insieme costituito da tutte le funzioni sulla stringa infinita $\mathscr{I}_{s(x)}$ al variare di $s(x)$ otteniamo uno spazio vettoriale isomorfo a $\mathbb{F}_{2}^{7}$ ed isomorfa a $\mathcal{R}_{7,2}$ come algebra se considerato con il prodotto di convoluzione sui vettori che generano le stringhe infinite. Infatti ogni funzione è univocamente determinata dal vettore $\psi_{2}(s(x))$ che ne definisce il dominio. \\
% \end{esempio}
% Giocando su vettori e stringhe, nel precedente esempio abbiamo costruito due algebre
%    \begin{align*}
%       \mathscr{F} &= \lbrace  \mathscr{F}_{s(x)} \mid s(x) \in \mathcal{R}_{7,2} \rbrace
%                 = \lbrace f:\mathbb{Z}_{7}\rightarrow\mathbb{F}_{2} \mid f(j)=\mathscr{F}_{s(x)}(j) \rbrace
%       \\
%       \mathscr{I} &= \lbrace  \mathscr{I}_{s(x)} \mid s(x) \in \mathcal{R}_{7,2} \rbrace
%            =\lbrace f:\mathbb{Z} \rightarrow  \mathbb{F}_{2} \mid f(j) = \mathscr{I}_{s(x)}(j) \rbrace
%    \end{align*}
% le quali, oltre ad essere due ulteriori varianti di $\mathcal{R}_{7,2}$, sono un caso particolare di successioni lineari ricorrenti $7$-periodiche e di funzioni lineari $7$-periodiche, che introdurremo in questo paragrafo\footnote{Per una esposizione completa delle successioni lineari ricorrenti sui campi finiti: \cite{lidl} pagine 190 e seguenti.}.
% \begin{definizione}
%    Siano $\mathbf{s} = (s_0, s_1, \dots , s_{r-1})$ vettore dei {\bf valori iniziali} ed $\mathbf{a} = (a_0, a_1, \dots , a_{r-1})$ vettore dei {\bf coefficienti}, elementi di $\mathcal{V}_{r, q}^{c}$, allora una successione $F_{j} = F_{n}(\mathbf{a},\mathbf{s})$ che soddisfa la relazione di ricorrenza
%    \begin{displaymath}
%      \left\{
%      \begin{array}{l c}
%      F_{j}(\mathbf{a},\mathbf{s}) = s_{j} & 0 \leq j \leq r-1 \\
%       F_{j}(\mathbf{a},\mathbf{s}) = \sum_{k=0}^{r-1} a_{k} F_{j-r+k}(\mathbf{a},\mathbf{s})  & n \geq r
%      \end{array}
%      \right.
%      \end{displaymath}
%    è detta {\bf sequenza lineare ricorrente} di ordine $r$.
% \end{definizione}
% Ad ogni sequenza lineare ricorrente si può associare un polinomio ed una matrice caratteristica che permettono di avere a disposizione degli strumenti in più per il loro studio.
% \begin{definizione}
%    Sia $F_{j}(\mathbf{a},\mathbf{s}) $ sequenza lineare ricorrente, allora il polinomio $c(x) \in \mathbb{F}_{q}\lbrack x \rbrack$ definito come
%    \begin{align*}
%       c(x)= x^{r} - a_{r-1}x^{r-1} - a_{r-2}x^{r-2} - \dots - a_{1}x - a_{0}
%    \end{align*}
%    è detto {\bf polinomio caratteristico} di $F_{j}$. \\
%    Mentre la matrice definita sul vettore $\mathbf{a}$ da
%    \begin{align*}
%       A =
%       \left(
%       \begin{array} {c c c c c c}
%       0 & 0 & 0 & \cdots & 0 & a_{0}    \\
%       1 & 0 & 0 & \cdots & 0 & a_{1}    \\
%       0 & 1 & 0 & \cdots & 0 & a_{2}    \\
%       0 & 0 & 1 & \cdots & 0 & a_{3}    \\
%        & \vdots  &  &  &  & \vdots    \\
%       0 & 0 & 0 & \cdots & 1 & a_{r-1}    \\
%       \end{array}
%       \right)
%     \end{align*}
%     è detta {\bf matrice caratteristica} o matrice compagna.
% \end{definizione}
% Osserviamo che per $a_{0} \neq 0$ la matrice $A$ è invertibile ed appartiene al gruppo lineare.
% La matrice caratteristica di una ricorrenza lineare genera la ricorrenza lineare, infatti definendo $v_{k} = v_{k}(\mathbf{a},\mathbf{s}) = (F_{k}, F_{k+1}, \dots , F_{k+r-1})$ il vettore costituito da una successione di $r$ elementi della sequenza lineare ricorrente $F_{j}$ a partire dall'elemento $k$-esimo si verifica che
% \begin{align*}
%   v_{k} = v_{0}A^{k}
% \end{align*}
% dove per definizione $v_{0}$ coincide con il vettore dei valori iniziali.\\
% Grazie alla definizione di matrice compagna si può dimostrare\footnote{Per brevità rimandiamo i dettagli al già citato \cite{lidl}. } che se la sequenza lineare ricorrente $F_{j}(\mathbf{a},\mathbf{s}) $ è omogenea, allora è periodica, cioè esiste un intero positivo $f$ tale che $F_{j+f} = F_{j}$ per ogni $j$ positivo. \\
% Il più piccolo $f$ che soddisfa l'equazione precedente è detto {\bf periodo} di $F_{n}$, e la sequenza lineare è detta {\bf $f$-periodica}.
% \begin{definizione}
%    Lo spazio vettoriale costituito dall'insieme delle ricorrenze lineari aventi $c(x)$ come polinomio caratteristico è indicato con $Rec(c(x))$:
%    \begin{align*}
%       Rec(c(x)) = \lbrace F_{j}(\psi_{2}(c(x)),\mathbf{s}) \mid \mathbf{s} \in \mathcal{V}_{r, q}^{c}   \rbrace
%    \end{align*}
% \end{definizione}
% Come presentato nell'esempio introduttivo, ogni polinomio $a(x)$ di $\mathcal{R}_{r,q}$ il cui vettore circolante associato è dato da $\psi_{2}(a(x)) = (a_{0},a_{1}, \dots, a_{r-1})$ definisce una funzione da $\mathbb{Z}_{r}$ in $\mathbb{F}_{q}$:
% \begin{align*}
% F\bigr|_{\mathbb{Z}_{r}}  : \mathbb{Z}_{r}  &\longrightarrow  \mathbb{F}_{q}  \\
%               j &\longmapsto a_{j}
% \end{align*}
% il cui dominio può essere esteso ai numeri interi, considerando la composizione con la proiezione $\pi$ da $\mathbb{Z}$ in $\mathbb{Z}_{r}$:
%       \vspace{0.2cm}

%       \[
%       \begindc{\commdiag}[30]
%       %insiemi
%       \obj(0,20)[Z]{$ \mathbb{Z} $}
%       \obj(25,20)[Zr]{$\mathbb{Z}_{r} $}
%       \obj(25,0)[F]{$ \mathbb{F}_{q} $}

%       %frecce
%       \mor{Z}{Zr}{$\pi$}[1,5]
%       \mor{Zr}{F}{$F\bigr|_{\mathbb{Z}_{r}}$}
%       \mor{Z}{F}{$ F $}

%       \enddc
%       \]

%       \vspace{0.2cm}

% \begin{align*}
% F = F\bigr|_{\mathbb{Z}_{r}} \circ\pi: \mathbb{Z} &\longrightarrow \mathbb{F}_{q}  \\
%               j &\longmapsto a_{j \mod{r}}
% \end{align*}
% Quindi la funzione $F(j) = F_{j}$ è $r$-periodica e può essere vista come una ricorrenza lineare con polinomio caratteristico $x^r -1$. Le due strutture si equivalgono.\\
% Nell'esempio \ref{ese:ricorr1} abbiamo considerato la ricorrenza lineare avente come polinomio caratteristico $x^7 - 1$. In generale le ricorrenze lineari aventi come polinomio caratteristico $x^r-1$ danno luogo a sequenze $r$-periodiche nelle quali il vettore $s(x)$ si ripete indefinitamente e senza variazioni. La struttura $Rec(x^r-1)$ è uno spazio vettoriale $r$-dimensionale isomorfo a $\mathbb{F}_{q}^{r}$ e considerando il prodotto di convoluzione fra i vettori dei valori iniziali è un'algebra isomorfa a $\mathcal{R}_{r,q}$:
% \begin{teorema}
%    La funzione
%    \begin{align*}
%       \psi_{2}: \mathcal{R}_{r,q}  & \longrightarrow Rec(x^r-1)   \\
%                            s(x)         &\longmapsto F_{j}(\psi_{2}(x^r-1),\psi_{2}(s(x)))
%    \end{align*}
%    è un isomorfismo di algebre
% \end{teorema}
% \begin{proof}
%    I due spazi vettoriali sono entrambi isomorfi a  $\mathbb{F}_{2}^{7}$, inoltre per $s(x)$ e $t(x)$ in $\mathcal{R}_{r,q} $ si ha che
%    \begin{align*}
%       \psi_{2} (s(x) t(x)) = \psi_{2} (s(x)) \psi_{2} (t(x))
%    \end{align*}
%    per come è stato definito il prodotto su $Rec(x^r-1)$.
% \end{proof}
% È stato dimostrato (\cite{cerruti} pag. $21$) che un ideale generato da $a(x)$ divisore di $x^r-1$ coincide con lo spazio delle ricorrenze lineari $r$-periodicheaventi polinomio caratteristico
% \begin{align*}
%    h(x) = \frac{\hat{a}(x)^{\perp}}{a_{0}}
% \end{align*}
% ed inoltre l'ideale minimale $(M^{(v)}(x))$ è costituito da tutte le ricorrenze lineari $r$-periodiche aventi polinomio caratteristico $M^{(-v)}(x)$.\\
% Una ulteriore indagine potrebbe essere sviluppata per rispondere a questa domanda:
% Cosa accade se anziché considerare $Rec(x^r-1)$, consideriamo $Rec(a(x))$ per $a(x)$ divisore di $\mathcal{R}$?

%%%%%%%%%%%%%%%%%%%%%%%%%%%%%%%%%%%%%%%%%%%%%%%%
%%%%%%%%%%%%%%%%%%%%%%%%%%%%%%%%%%%%%%%%%%%%%%%%
% %%%%%%%%%%%%%%%%%%%%%% SEZIONE    %%%%%%%%%%%%%%%%%%%%
% \section{Diagonalizzazione a blocchi}
%
% Possiamo diagonalizzare una matrice circolante usando la trasformata di Fourier discreta. Valutando la trasformata di Winograd come caso particolare della trasformata di Fourier come ha fatto Winograd nella sua idea originale \cite{winograd}, oltre che come trasformazione lineare come fatto in questa tesi basata sull'articolo \cite{cerruti} possiamo ottenere il risultato che {\bf la trasformata di Winograd diagonalizza a blocchi una matrice circolante}. \\
%
% \subsection{Trasformata di Fourier discreta unitaria}
%
% In generale un segnale è una funzione discreta sui complessi $v(k)$ periodica di periodo $N$. Può quindi essere interpretata come una successione finita di $n$ punti complessi: $ (v_0, v_1, \dots, v_{n-1} )$.
% La trasformata di Fourier dell'insieme di punti $v_n$ è una successione di punti complessi $ (V_0, V_1, \dots, V_{n-1} )$ derivata dalla prima successione attraverso la formula:
% \begin{align*}
% V_k = \frac{1}{n} \sum_{h=0}^{n-1} v_h e^{\frac{2\pi i k h}{n}}
% \end{align*}
% Da cui si ottiene che la funzione complessa discreta inizialmente data può essere scritta come sovrapposizione finita dei fasori, definiti come gli addendi della seguente sommatoria:
% \begin{align*}
% v_k = \sum_{h=0}^{n-1} V_h e^{ - \frac{2\pi i k h}{n}}
% \end{align*}
% Quindi si può interpretare la trasformata discreta di Fourier come una funzione lineare dallo spazio vettoriale complesso in se stesso, che mappa un vettore di $n$ punti in un altro.
% \begin{align*}
% &&\mathcal{F} : \mathbb{C}^{n}  \longrightarrow  \mathbb{C}^{n} \\
% \mathcal{F}(v_k) &=  \sum_{h=0}^{n-1} V_h e^{ - \frac{2\pi i k h}{n}} &
% &\mathcal{F}^{-1}(V_k) = \frac{1}{n} \sum_{h=0}^{n-1} v_h e^{ \frac{2\pi i k h}{n}}
% \end{align*}
% avendo indicato con $\mathcal{F}$ la trasformata di Fourier.
%
% \noindent
% I coefficienti $e^{ \frac{2\pi i k h}{n}}$ sono detti fasi lineari e sono generate come potenze della radice primitiva $n-$esima dell'unità $\xi_{n} = e^{\frac{2\pi i}{n}}$. Si osserva che moltiplicando $v_k$ per una fase lineare si ottiene uno shift circolare dell'elemento. \\\\
% \noindent
% $\mathcal{F}$ e la sua inversa possono essere rappresentate in forma matriciale, come $F = (\overline{\xi}_{n}^{(i-1)(j-1)})_{i,j}$ ed  $F^{-1} =\frac{1}{n}(\xi_{n}^{(i-1)(j-1)})_{i,j}$:
% \begin{align*}
% \mathbf{v}^{t} = F \mathbf{V}^t  & & & \mathbf{V}^{t} = F^{-1} \mathbf{v}^t
% \end{align*}
%
% \begin{align*}
% \left(
% \begin{array} {c }
% v_0\\
% v_1\\
% v_2\\
% \vdots \\
% v_{n-1}
% \end{array}
% \right)
% =
% \left(
% \begin{array} {c c c c c }
% 1   &  1  &  1  & 1 & 1\\
% 1 & \overline{\xi}_{n} & \overline{\xi}_{n}^{2} &\dots & \overline{\xi}_{n}^{n-1} \\
% 1 & \overline{\xi}_{n}^{2} & \overline{\xi}_{n}^{4} &\dots & \overline{\xi}_{n}^{2(n-1)} \\
% \vdots & & &  & \vdots \\
% 1 & \overline{\xi}_{n}^{n-1} & \overline{\xi}_{n}^{2(n-1)} &\dots & \overline{\xi}_{n}^{(n-1)(n-1)}
% \end{array}
% \right)
% \left(
% \begin{array} {c }
% V_0\\
% V_1\\
% V_1\\
% \vdots \\
% V_{n-1}
% \end{array}
% \right)
% \end{align*}
% \begin{align*}
% \left(
% \begin{array} {c }
% V_0\\
% V_1\\
% V_2\\
% \vdots \\
% V_{n-1}
% \end{array}
% \right)
% =
% \frac{1}{n}
% \left(
% \begin{array} {c c c c c }
% 1   &  1  &  1  & 1 & 1\\
% 1 & \xi_{n} & \xi_{n}^{2} &\dots & \xi_{n}^{n-1} \\
% 1 & \xi_{n}^{2} & \xi_{n}^{4} &\dots & \xi_{n}^{2(n-1)} \\
% \vdots & & &  & \vdots \\
% 1 & \xi_{n}^{n-1} & \xi_{n}^{2(n-1)} &\dots & \xi_{n}^{(n-1)(n-1)}
% \end{array}
% \right)
% \left(
% \begin{array} {c }
% v_0\\
% v_1\\
% v_1\\
% \vdots \\
% v_{n-1}
% \end{array}
% \right)
% \end{align*}
% Dove $F$ è detta matrice della trasformata di Fourier, ed $F^{-1}$ è la sua inversa.
% %%%%%%%%%%%%%%%%%%%%%%%%%%%%%%%%%%%%%%%%%%%%%%%%%%%%%%%%%%%%%%%%%%%%%%%%%%%%%%%%%%%% lemma
% \begin{lemmax}
%  Sia $ \lbrace \xi_{n}^{j} \rbrace_{j \in \mathbb{Z}_{n} }$ insieme delle potenze $n-$esime della radice primitiva dell'unità, allora
%  \begin{align*}
%   \sum_{j \in \mathbb{Z}_{n} } \xi_{n}^{j} = 0
%  \end{align*}
% \end{lemmax}
% \begin{proof}
% \begin{align*}
% z \in \mathbb{C}: \quad & 1-z^{n} = (1 - z)(z^{n-1} + z^{n-2} + \dots + 1)\\
% &   \frac{1-z^{n}}{(1 - z)} = (z^{n-1} + z^{n-2} + \dots + 1)
% \end{align*}
% Per $z = \xi_{n}$.
% \end{proof}
%
% \noindent
% Dal lemma si verfica che $F^{-1}$ è l'inversa anche nel senso dell'algebra lineare.
% \begin{align*}
% (F F^{-1})_{i,j} = \frac{1}{n} \big( \sum_{h \in \mathbb{Z}_{n}} \overline{\xi}_{n}^{(i-1)(h-1)} \xi_{n}^{(h-1)(j-1)} \big)_{i,j} =
% \left\lbrace
% \begin{array} {c l}
% 1 & i=j  \\
% 0 & altrimenti
% \end{array}
% \right.
% \end{align*}
%
% \noindent
% Indicando con $F^{\star}$la coniugata della trasposta di $F$, si può inoltre osservare che
% \begin{align*}
% F^{-1} =  \frac{1}{n} F^{\star} & & F F^{\star} = n I_{n}
% \end{align*}
% Segue che F non è unitaria. Si introduce allora la normalizzazione unitaria di F come
% \begin{align*}
% U =  \frac{F}{\sqrt{n}} & & U^{\star} = \frac{F^{\star}}{\sqrt{n}}
% \end{align*}
% Per cui $U$ prende il nome di matrice di Fourier unitaria, ed $U^{-1} = U^{\star}$ è l'inversa.
%
%
% \noindent
% Oltre ad $s_n$ c'è una matrice con un ruolo importante nella trasformata di Fourier discreta, che sarà usata nel prossimo teorema:
% \begin{align*}
% t_n =
% \left(
% \begin{array} {c | c c c c }
% 1 & 0 & 0 & \dots & 0    \\
% \hline
% 0 & 0 & 0 & \dots & 1   \\
% \vdots & & \vdots &  & \vdots   \\
% 0 & 0 & 1 & \dots & 0   \\
% 0 & 1 & 0 & \dots & 0
% \end{array}
% \right)
% \end{align*}
% %%%%%%%%%%%%%%%%%%%%%%%%%%%%%%%%%%%% teorema
% \begin{teorema}
% Sia $U$ matrice di Fourier unitaria:
%  \begin{align*}
% (U^{\star})^{2} = U^{\star}U^{\star} = (U)^{2} = t_n
% \end{align*}
% \end{teorema}
% %%%%%%%%%%%%%%%%%%%%%%%%%%%%%%% corollari
% \begin{corollario}
% \begin{align*}
% (U^{\star})^{4} = (t_n)^{2} = I & &  (U)^{4} = (t_n)^{2} = I
% \end{align*}
% \end{corollario}
% Questo fornisce la curiosa notazione\footnote{\cite{davis}, cap 2.}
% \begin{align*}
% U = \sqrt[4]{I}
% \end{align*}
%
% \subsection{Teorema di diagonalizzazione delle matrici circolanti e diagonalizzazione a blocchi}
%
% Affrontiamo alcune proprietà sulle matrici circolanti per dimostrare che sono diagonalizzate dalla trasformata di Fourier discreta.
% \begin{teorema}(caratterizzazione delle matrici circolanti)
%   Sia $s_n = circ(0,1,0,\dots,0)$. Sia $A$ matrice $n\times n$ a coefficienti in $\mathbb{F}$:
%   \begin{align*}
%     A \in \mathfrak{C}_{\mathbb{F},n} \quad \iff \quad A s_n = s_n A
%   \end{align*}
% \end{teorema}
% \begin{proof}
%   La generica permutazione $\tau \in S_n$ induce sulla generica matrice $A = (a_{i,j}) $ la seguente permutazione degli indici
%   \begin{align*}
%      \tau A \tau^{-1} = \tau (a_{i,j}) \tau^{-1} = (a_{\tau(i),\tau(j)})
%   \end{align*}
%   Sia $\sigma = (1,2,\dots , n)$ permutazione corrispondente allo shift di un posto verso destra. La matrice di permutazioni corrispondente è $s_n$:
%   \begin{align*}
%      A \in \mathfrak{C}_{\mathbb{K},n} \quad \iff \quad (a_{i,j}) = (a_{\sigma(i),\sigma(j)})  \quad \iff \quad s_n A s_n^{\star} = A
%   \end{align*}
% \end{proof}
% %%%%%%%%%%%%%%%%%%%%%%%%%%%%corollario
% \begin{corollario}
%    $A$ è circolante $\iff$ $A^{\star}$ è circolante.
% \end{corollario}
% %%%%%%%%%%%%%%%%%%%%%%%%%% definizone diagonale
% \begin{definizione}
% Dato il vettore circolante $(a_0,a_1,\dots,a_{n-1}) \in\mathbb{F}^{n}$, allora la matrice diagonale di $(a_0,a_1,\dots,a_{n-1})$ è definita come
% \begin{align*}
%   diag(a_0,a_1,\dots,a_{n-1}) =
%   \left(
%   \begin{array} { c c c c }
%     a_0 & 0 & \dots & 0 \\
%     0 & a_1 & \dots & 0 \\
%     \vdots & & & \vdots \\
%     0 & 0 & \dots & a_{n-1}
%    \end{array}
%   \right)
%   \end{align*}
% \end{definizione}
% Sia $\xi = diag(1, \xi_{n}, \xi_{n}^{2}, \dots , \xi_{n}^{n-1} )$ allora si osserva che \\
% \mbox{$\xi^{k} = diag(1, \xi_{n}^{k}, \xi_{n}^{2k}, \dots , \xi_{n}^{(n-1)k} )$}.
%
% %%%%%%%%%%%%%%%%%%%%%% teorema
% \begin{teorema}(diagonalizzazione delle matrici circolanti)
% \begin{align*}
% s_n = U^{\star} \xi U
% \end{align*}
% \end{teorema}
%
% \begin{proof}
% Si indica la $i-$esima riga di $U^{\star}$ con $U_{i,\sim}^{\star}$ e la $j-$esima colonna con $U_{\sim,j}^{\star}$.
% \begin{align*}
% U_{i,\sim}^{\star} = \frac{1}{n} (\xi_{n}^{(i-1)0}, \xi_{n}^{(i-1)1}, \dots , \xi_{n}^{(i-1)(n-1)} )
% \end{align*}
% L'elemento di posto $(i,r)$ di $U^{\star}\xi$ è dato da
% \begin{align*}
% (U^{\star}\xi)_{i,r} = \frac{1}{\sqrt{n}} (\xi_{n}^{(i-1)0}, \xi_{n}^{(i-1)1}, \dots , \xi_{n}^{(i-1)(n-1)} )(0,\dots,0,\xi_{n}^{r},0,\dots,0)^{t}
% \end{align*}
% Quindi la $i-$ esima riga di $U^{\star}\xi$ è data da
% \begin{align*}
% \big(U^{\star}\xi\big)_{i,\sim} = \frac{1}{\sqrt{n}} (\xi_{n}^{(i)0}, \xi_{n}^{(i)1}, \dots , \xi_{n}^{ir}, \dots, \xi_{n}^{i(n-1)})
% \end{align*}
% Mentre la $j-$esima colonna di $U$ è data da
% \begin{align*}
% U_{\sim,j} = \frac{1}{\sqrt{n}} (\overline{\xi}_{n}^{0(j-1)}, \overline{\xi}_{n}^{1(j-1)},  \dots , \overline{\xi}_{n}^{r(j-1)}, \dots , \overline{\xi}_{n}^{(n-1)(j-1)} )^{t}
% \end{align*}
% Quindi l'elemento di posto $(i,j)$ di $U^{\star}\xi U$ è dato da
% \begin{align*}
% &\big( U^{\star}\xi U \big)_{i,j}=\\
% &= \frac{1}{n} (\xi_{n}^{(i)0}, \dots , \xi_{n}^{ir}, \dots, \xi_{n}^{i(n-1)}) (\overline{\xi}_{n}^{0(j-1)},\dots , \overline{\xi}_{n}^{r(j-1)}, \dots , \overline{\xi}_{n}^{(n-1)(j-1)} )^{t} \\
% & = \frac{1}{n} \Big( \sum_{r = 0}^{n-1} \xi_{n}^{ir} \overline{\xi}_{n}^{r(j-1)} \Big) \\
% & = \frac{1}{n} \Big( \sum_{r \in \mathbb{Z}_n} \xi_{n}^{r(i-j+1)} \Big)
% =
% \left\lbrace
% \begin{array} {c l}
% 1 & i \equiv j+1 \mod{n} \\
% 0 & altrimenti
% \end{array}
% \right.
% \end{align*}
% La matrice ottenuta in questo modo è proprio $s_n$.
% \end{proof}
%
% \noindent
% Dato $\mathbf{a} = (a_{0}, a_{1}, \dots , a_{n-1})$ vettore circolante di dimensione $n$, si introduce la notazione $ \gamma_{\mathbf{a}}(x)$  per indicare il polinomio di grado $n$ avente gli stessi coefficienti di $\mathbf{a}$:
% \begin{align*}
% \gamma_{\mathbf{a}}(x) =  \sum_{h=0}^{n-1} a_{h} x^{h}
% \end{align*}
%
% %%%%%%%%%%%%%%%%%%%%%%%%%%%%%%%%%%%%%% corollario
% \begin{corollario}
% Se $A$ è una matrice circolante, allora è diagonalizzata da $U$.
% \end{corollario}
% \begin{proof}
%  $A \in \mathfrak{C}_{\mathbb{K},n}$, con $A = circ(\mathbf{a}) = circ(a_0, a_1, \dots, a_{n-1})$. Osservando che
% \begin{align*}
% U s_{n}U^{\star} = \xi \quad \Longrightarrow \quad U s_{n}^{m}U^{\star} = \xi^{m}
% \end{align*}
% e che
% \begin{align*}
% A = a_0  I_{n} + a_1  s_{n}^{1} + \dots + a_{n-1}  s_{n}^{n-1}
% \end{align*}
% segue
% \begin{align*}
% UAU^{\star} &= a_0  U I_{n}U^{\star} + a_1  U s_{n}^{1}U^{\star} + \dots + a_{n-1} U s_{n}^{n-1}U^{\star} U s_{n}^{n-1}U^{\star}\\
% &= a_0 I_{n} + a_1 \xi^{1} + \dots + a_{m}  \xi^{m} \dots + a_{n-1}  \xi^{n-1}
% \end{align*}
% Il secondo membro equivale alla matrice diagonale:
% \begin{align*}
% \Lambda_{\mathbf{a}} :=  diag(\gamma_{\mathbf{a}}(1), \gamma_{\mathbf{a}}(\xi), \dots ,\gamma_{\mathbf{a}}(\xi^{t}), \dots , \gamma_{\mathbf{a}}(\xi^{n-1}))
% \end{align*}
% Quindi
% \begin{align*}
% UAU^{\star} = \Lambda_{\mathbf{a}}
% \end{align*}
% Osserviamo che gli autovalori sono della forma $\lambda_{m} = \gamma_{\mathbf{a}}(\xi^{m-1}) $
% \end{proof}
% Viceversa la trasformata di Fourier discreta unitaria applicata ad una matrice diagonale, genera una matrice circolante:
% %%%%%%%%%%%%%%%%%%%%%%%%%%%%%%%%%%%%%%%%%%%%%%%%%%%%%% corollario
% \begin{corollario}
% Sia $\mathbf{\lambda} = (\lambda_{1}, \dots, \lambda_{n})$ vettore in $\mathbb{K}$, allora
% \begin{align*}
% U^{\star} diag(\mathbf{\lambda}) U
% \end{align*}
% è una matrice circolante di autovalori $\lambda_{1}, \dots, \lambda_{n}$.
% \end{corollario}
% %%
% Ulteriori dettagli si trovano a pag. 73 di \cite{davis}.\\
% In modo analogo a quanto fatto con la trasformata di Fourier possiamo usare la trasformata di Wiongrad per diagonalizzare a blocchi una matrice circolante. Diamo senza dimostrazione il teorema di diagonalizzazione a blocchi:
% \begin{teorema}
%    Sia $\Gamma$... osservazione sulla dimensione dei blocchi...
% \end{teorema}
% definizione di {\bf blocchi della scomposizione di Winograd}.
% $W_{v}$
%
%
% %%%%%%%%%%%%%%%%%%%%%%%%%%%%%%%%%%%%%%%%%%%%%%%%
% %%%%%%%%%%%%%%%%%%%%%%%%%%%%%%%%%%%%%%%%%%%%%%%%
% %%%%%%%%%%%%%%%%%%%%%% SEZIONE    %%%%%%%%%%%%%%%%%%%%
% \section{Scomposizione con gli idempotenti}
%
% Usando gli idempotenti possiamo scomporre una matrice circolante in un prodotto di
% {\bf blocchi della scomposizione per idempotenti} $C_{v}$...
%
%
%
% Possiamo osservare che in alcuni casi (in tutti quelli esaminati) $det(C_{v}) = det(W_{v})$: il determinante blocchi di Winograd e il determinante dei blocchi della scomposizione con gli idempotenti coincidono. Questo risultato vale sempre?