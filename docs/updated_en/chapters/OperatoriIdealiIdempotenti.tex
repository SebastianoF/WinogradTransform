
%%%%%%%%%%%%%%%%%%%%%%%%%%%%%%%%%%%%%%%%%%%%%%%%
%%%%%%%%%%%%%%%%%%%%%% CAPITOLI     %%%%%%%%%%%%%%%%%%%%
%%%%%%%%%%%%%%%%%%%%%%%%%%%%%%%%%%%%%%%%%%%%%%%%
\chapter{Operatori, ideali e idempotenti minimali}

Fino a qui abbiamo portato avanti in parallelo il discorso della
fattorizzazione di $x^r - 1$ sia sul campo dei razionali che sui campi finiti
per l'analogia significativa che si svela nel calcolo dei polinomi minimi.
Dato che lo scopo della tesi
riguarda le applicazioni alla teoria dei codici correttori, procederemo solo sui campi di caratteristica $p$, tenendo presente che molti dei risultati possono essere generalizzati anche ai campi perfetti di caratteristica zero. \\
Ogni campo finito con $q = p^{n}$ elementi è definito in modo unico a meno di isomorfismi, quindi nei pedici delle algebre introdotte, non indichiamo più $\mathbb{F}_{q}$, ma semplicemente $q$. Matrici e vettori circolanti saranno indicati rispettivamente con $\mathcal{V}_{r, q}^{c}$ ed $\mathcal{M}_{r,q }^{c}$; l'algebra $\mathbb{F}_{q}C_{r}$ e l'algebra dei polinomi modulo $x^r-1$ saranno indicati con $\mathcal{A}_{r, q} $ ed $\mathcal{R}_{r,q} $ rispettivamente e le fattorizzazioni in prodotto di campi delle algebre saranno indicati con $\mathcal{P}_{r, q} $ e $\mathcal{Q}_{r, q} $.

\[
\begindc{\commdiag}[3]
%sotto
\obj(0,0)[V]{$ \mathcal{V}_{r, q}^{c} $}
\obj(40,0)[M]{$\mathcal{M}_{r,q }^{c}$}


%metà
\obj(0,20)[R]{$ \mathcal{R}_{r,q } $}
\obj(40,20)[A]{$ \mathcal{A}_{r,q } $}
\obj(-40,20)[P]{$ \mathcal{P}_{r,q } $}

%sopra
\obj(0,40)[Q]{$ \mathcal{Q}_{r,q } $}

%frecce orizzontali
\mor{V}{M}{$\psi_{1}$}
\mor{R}{A}{$\psi_{4}$}
\mor{R}{P}{$\eta$}

%frecce verticali
\mor{R}{V}{$\psi_{2}$}
\mor{A}{M}{$\psi_{3}$}
\mor{R}{Q}{$\gamma$}

%frecce oblique
\mor{P}{Q}{$\mu$}
%\mor{A}{Q}{$\mu_{2}$}

\enddc
\]

\noindent
Il capitolo che state per leggere ha tre scopi principali: definire cinque operatori che ci permetteranno di maneggiare più agevolmente i polinomi modulo $x^r-1$ ed analizzare alcune delle loro principali conseguenze; studiare la forma degli ideali di $\mathcal{R}$, che avranno un ruolo particolare nello studio di $\mathcal{Q}$; infine usare $\gamma$ per trovare gli idempotenti di $\mathcal{R}$ ed utilizzarli come generatori dei suoi ideali.


%%%%%%%%%%%%%%%%%%%%%%%%%%%%%%%%%%%%%%%%%%%%%%%%
%%%%%%%%%%%%%%%%%%%%%% SEZIONI    %%%%%%%%%%%%%%%%%%%%
\section{Operatori su $\mathcal{R}$}
Sia $r$ intero positivo, $\mathbb{F}_{q}$ campo finito di ordine $q=p^n$ con $(r,p)=1$, e sia $G \trianglelefteq \mathbb{Z}_{r}^{\star}$ isomorfo a $ Gal(\mathbb{F}_{q}(\xi), \mathbb{F}_{q})$ per $\xi$ radice primitiva $r$-esima dell'unità.
\begin{definizione}
   Dato $a(x) \in \mathcal{R}_{r,q}$ si definisce {\bf k-shift} di $a(x)$ il polinomio $x^{k}a(x)$ modulo $x^r - 1$, per $k \in \mathbb{Z}$.
   I k-shift sono indicati con $\sigma_{k}$:
   \begin{align*}
     \sigma_{k} : \mathcal{R} &\longrightarrow  \mathcal{R}  \\
                         a(x) &\longmapsto x^k a(x)
   \end{align*}
   Si definisce {\bf g-coniugato} di $a(x)$ il polinomio $a(x^g)$ modulo $x^r-1$ per $g \in G$, e si indica con $\tau_{g}$:
   \begin{align*}
     \tau_{g} : \mathcal{R} &\longrightarrow  \mathcal{R}  \\
                         a(x) &\longmapsto a(x^g)
   \end{align*}
\end{definizione}

Come primo risultato abbiamo la
\begin{prop}
   Con le notazioni precedenti, per $a(x) = \sum_{j \in \mathbb{Z}_{r}} a_{j} x^j$
   \begin{align*}
     & \sigma_{k} (a(x)) = \sum_{j \in \mathbb{Z}_{r}} a_{j-k} x^j \\
     & \tau_{g} (a(x)) = \sum_{j \in \mathbb{Z}_{r}} a_{j}x^{gj}
   \end{align*}
   inoltre per ogni $g$ nel gruppo $G$, $\tau_{g}$ è un isomorfismo di algebre.
\end{prop}
\begin{proof}
   Le prime due equazioni sono conseguenza diretta della definizione.
   L'operatore $\tau_{g}$ è lineare ed è definito sull'automorfismo del gruppo ciclico generato da $x$ in $\mathcal{R}$. Il fatto di elevare $x$ per un elemento del gruppo $G$ mantiene le eventuali radici che $a(x)$ ha in comune con $x^r - 1$ nella stessa orbita.
\end{proof}
\begin{definizione}
   Sia dato $a(x) \in \mathcal{R}$ il cui vettore circolante associato è dato da
   \begin{align*}
      \psi_{2}(a(x))=(a_0 , a_1 , \dots, a_{r-1})
   \end{align*}
   allora si definisce {\bf riflesso} di $a(x)$ il polinomio $a(x)^{R}$ il cui vettore circolante associato è definito come
   \begin{align*}
      \psi_{2}(a(x)^{R})=(a_{r-1} , a_{r-2} , \dots, a_{0})
   \end{align*}
\end{definizione}
\noindent
Vale la seguente
\begin{prop}
   Con le notazioni precedenti
   \begin{align*}
     a(x)^{R}  = \sum_{j \in \mathbb{Z}_{r}} a_{r-1-j}x^{j}
   \end{align*}
\end{prop}
\noindent
Con la prossima definizione concludiamo l'elenco degli operatori su $\mathcal{R}$ presentati in questa ricerca.
\begin{definizione}
   Dato $a(x) \in \mathcal{R}$ si definisce {\bf trasposto} di $a(x)$ il polinomio $a(x)^{T}$ il cui vettore circolante associato è dato da
   \begin{align*}
   \psi_{2}(a(x)^{T})=(a_{0} , a_{r-1} , a_{r-2} \dots, a_{2}, a_{1})
   \end{align*}
   Si definisce {\bf reciproco} del polinomio $a(x)$ il polinomio
   \begin{align*}
      a(x)^{\perp} = x^d a(x^{-1})
   \end{align*}
   dove $d$ è il grado di $a(x)$.
\end{definizione}
\noindent
Il trasposto di $a(x)$ è il polinomio determinato dalla prima riga della trasposta della matrice determinata da $a(x)$. Volendo rendere le cose comode da implementare (ma meno leggibili), possiamo definire il trasposto di $a(x)$ come
\begin{align*}
   a(x)^{T} = \psi_{2}^{-1} ( \psi_{1}^{-1} \Big( \psi_{1} ( \psi_{2}(a(x))) \Big)^{t}    )
\end{align*}
mentre il reciproco di $a(x)$ può essere definito come il polinomio riflesso, considerato però su uno spazio vettoriale di dimensione ridotta, pari al grado di $a(x)$ e poi riportato nello spazio originale di dimensione $r$. Per gli operatori trasposto e reciproco, abbiamo la
\begin{prop}
   Sia $a(x) \in \mathcal{R}$, allora
   \begin{align*}
     & a(x)^{T} = a_{0} + \sum_{j \in \mathbb{Z}_{r} \setminus \lbrace 0 \rbrace } a_{r-j} x^j
                 = \sum_{j \in \mathbb{Z}_{r} } a_{r-j} x^j \\
     & a(x)^{\perp} = \sum_{j \in \mathbb{Z}_{r}} a_{d-j}x^{j}
   \end{align*}
\end{prop}
\begin{proof}
   La prima è conseguenza immediata della definizione. Si dimostra la seconda:
   \begin{align*}
     a(x)^{\perp}  &=  x^d a(x^{-1}) =  x^d \sum_{i \in \mathbb{Z}_{r}} a_{i}(x^{-1})^{i} \\
       &= x^d \sum_{i \in \mathbb{Z}_{r}} a_{i}x^{-i} = \sum_{i \in \mathbb{Z}_{r}} a_{i}x^{d-i}  \\
       &= \sum_{j \in \mathbb{Z}_{r}} a_{d-j}x^{j}
   \end{align*}
   Avendo posto nell'ultimo passaggio $i=d-j$.
\end{proof}
\noindent
Possiamo esplorare alcune relazioni fra gli operatori di $\mathcal{R}$ appena definiti:
\begin{prop}
  Sia $a(x) \in \mathcal{R}$ di grado $d$, allora
  \begin{align*}
     & a(x)^{R} = x^{-(d+1)} a(x)^{\perp} \\
     & a(x)^{T} = x^{-d} a(x)^{\perp}
   \end{align*}
\end{prop}
\begin{proof}
   La prima equazione segue da:
   \begin{align*}
     x^{-(d+1)}a(x)^{\perp}  &= x^{-1} \sum_{i \in \mathbb{Z}_{r}} a_{i}x^{r-i} = \sum_{i \in \mathbb{Z}_{r}} a_{i}x^{r-i-1} = \sum_{i \in \mathbb{Z}_{r}} a_{r-j-1}x^{j}  = a(x)^{R}
   \end{align*}
   Mentre la seconda segue applicando la definizione di $a(x)^{\perp} =  x^d a(x^{-1}) $:
   \begin{align*}
     x^{-d} a(x)^{\perp}  &= x^{d-d}  a(x^{-1}) = a(x^{-1}) = a(x)^{T}
   \end{align*}
\end{proof}
%%%%%%%%%%%
\begin{prop}
   Se $deg(a(x)) = r-1$ allora il reciproco coincide con il riflesso
   \begin{align*}
      a(x)^{\perp}=a(x)^{R}
   \end{align*}
\end{prop}
\begin{proof}
    \begin{align*}
     a(x)^{\perp}  &= \sum_{j \in \mathbb{Z}_{r}} a_{j}x^{d-j} = \sum_{j \in \mathbb{Z}_{r}} a_{j}x^{r-1-j} = \\
                   &= \sum_{j \in \mathbb{Z}_{r}} a_{r-j-1}x^{j}  = a(x)^{R}
   \end{align*}
\end{proof}
Se $a(x)$ è un divisore di $x^r - 1$ (quindi un prodotto dei suoi fattori irriducibili $M^{(v)}(x)$) allora abbiamo alcune proprietà:

\begin{prop}\label{prop:reciprocoISdivisore}
   Se $a(x)\in \mathcal{R}$ è un divisore di $x^r - 1$ allora anche il suo reciproco $a(x)^{\perp}$ è un divisore di $x^r-1$ in $\mathcal{R}$.
\end{prop}
\begin{proof}
   Se $a(x) = a_{0} + a_{1}x + \dots + a_{d-1}x^{d-1} + a_{d}x^{d}$ allora abbiamo che, per definizione $a(x)^{\perp} = a_{d} + a_{d-1}x + \dots +  a_{1} x^{d-1} + a_{0}x^{d} = x^{d} a(1/d)$.\\
   Indichiamo con $\hat{a}(x)$ il polinomio di grado $r-d$ dato da
   \begin{align*}
      \hat{a}(x) = (x^r - 1)/a(x)
   \end{align*}
   ben definito dal fatto che $a(x)$ è divisore di $x^r - 1$. Allora
   \begin{align*}
       \hat{a}(x)a(x) &= x^r - 1 \\
       \hat{a}(1/x)a(1/x) &= (1/x)^r - 1 \\
       x^{r} \hat{a}(1/x)a(1/x) &= 1-x^r \\
       x^{r-d} \hat{a}(1/x)x^{d} a(1/x) &= 1-x^r \\
       -x^{r-d} \hat{a}(1/x) a(x)^{\perp} &= x^r - 1
   \end{align*}
   ed essendo $-x^{r-d} \hat{a}(1/x)$ polinomio di grado positivo o uguale a zero, segue che $a(x)^{\perp}$ è un divisore di $x^r - 1$.
\end{proof}
La prossima proprietà\footnote{\cite{cerruti} proprietà $1.11$ pag $11$.} come la precedente
caratterizza gli operatori introdotti quando agiscono sui divisori di $x^r - 1$
\begin{prop} \label{prop:multiplaOperatori}
    Con le notazioni introdotte si verificano i seguenti risultati:
   \begin{enumerate}
      \item \label{cap3:punto2} Per ogni $g \in G$, $k \in \mathbb{Z}$ le radici che $a(x)$, $\sigma_{k}(a(x))$, $\tau_{g}(a(x))$ hanno in comune con $x^r -1$ coincidono fra loro.
      \item \label{cap3:punto3} Le radici che $a(x)^{\perp}$ ha in comune con $x^r -1$ coincidono con i reciproci delle radici che $a(x)$ ha in comune con $x^r - 1$.
      \item \label{cap3:punto4} Le radici che $a(x)^{R}$, $a(x)^{T}$ hanno in comune con $x^r -1$ coincidono fra loro.
      \item \label{cap3:punto5} Per ogni $v\in \mathscr{L}$, $M^{(v)}(x)^{\perp} = \lambda M^{(-v)}(x)$, dove $\lambda$ è il termine noto di $M^{(v)}(x)$, indicato con $M_{0}^{(v)}$.
      \item \label{cap3:punto6} Se le orbite sono autoconiugate e $v\neq 0$ allora $\lambda = 1$ ed $M^{(v)}(x)^{\perp} = M^{(-v)}(x)$.
   \end{enumerate}
\end{prop}
\begin{proof}
   Dimostriamo i diversi punti separatamente:
   \begin{enumerate}
      \item Sia $\lbrace \xi^{t} \mid t \in O(v) \rbrace$ insieme delle radici del polinomio minimo $M^{(v)}(x)$ cioè l'insieme delle radici coniugate ad $\xi^{v}$, allora
      \begin{align*}
         a(\xi^{v}) = 0 \iff a(\xi^{vg}) = 0 \qquad \forall g \in G
      \end{align*}
      quindi $a(x)$ e $\tau_{g}(a(x))$ hanno le stesse radici in comune con $x^r-1$. \\
      Inoltre $\sigma_{k}(a(x)) = x^ka(x)$ modulo $x^r-1$, cioè in $\mathbb{F}_{q}\lbrack x \rbrack$:
      \begin{align*}
         \sigma_{k}(a(x)) = x^ka(x) + b(x)(x^r-1)  \qquad b(x)\in \mathbb{F}_{q}\lbrack x \rbrack
      \end{align*}
      Ogni $\xi^{t}$ che annulla $a(x)$ ed $x^r-1$ annulla anche $\sigma_{k}(a(x))$.
      \item Se $\xi^{v}$ è radice di $a(x)$ allora $\xi^{-v}$ è radice di $a(x^{-1})$ ed in particolare di $x^{d}a(x^{-1}) = a(x)^{\perp}$. Viceversa se $\xi^{v}$ è radice di $ a(x)^{\perp}$ allora $\xi^{-v}$ è radice di $a(x)$.
      \item Dal fatto che
        \begin{align*}
	  & a(x)^{R} = x^{-(d+1)} a(x)^{\perp} = x^{-1}a(x^{-1}) \\
	  & a(x)^{T} = x^{-d} a(x)^{\perp} = a(x^{-1})
	\end{align*}
      e dal punto precedente segue che almeno le radici che $a(x)^{R}$ ed $a(x)^{T}$ hanno in comune con $x^r -1$ coincidono fra loro.





      \item Le radici di $M^{(v)}(x)$ sono i reciproci delle radici di $M^{(v)}(x)^{\perp}$ e da quanto visto $M^{(v)}(x)^{\perp}$ è divisore di $x^r - 1$. Questi due polinomi hanno lo stesso grado e l'insieme delle radici di $M^{(v)}(x)^{\perp}$ è dato da
      \begin{align*}
	 \lbrace \xi^{-t} \mid t \in O(v) \rbrace
       \end{align*}
      quindi
      \begin{align*}
	 M^{(v)}(x)^{\perp} = \lambda (\prod_{t \in O(v)}(x-\xi^{-t}) ) = \lambda M^{(-v)}(x)
       \end{align*}
      Dato che $M_{0}^{(v)}$ è il coefficiente direttivo di $M^{(v)}(x)^{\perp}$, allora $\lambda = M_{0}^{(v)}$.
      \item Se le orbite sono autoconiugate allora $M^{(v)}(x)^{\perp} = M_{0}^{(v)} M^{(v)}(x)$. Osservando che
      \begin{align*}
	 M_{0}^{(v)}  = (-1)^{m(v)} \prod_{t \in O(v)}\xi^{t}
       \end{align*}
       ed applicando il lemma \ref{cap2:lemma110} per $r$ dispari e $v\neq 0$, $m(v)$ è pari ed
       \begin{align*}
	 M_{0}^{(v)}  = (-1)^{m(v)} \prod \xi^{t} \xi^{-t} = 1
       \end{align*}
       Mentre per $r$ pari e $v\neq 0$ allora si considera solo il caso $v=r/2$:
       \begin{align*}
	 M_{0}^{(r/2)}  = (-1)^{m(v)}  \xi^{r/2}  = (-1)(-1) = 1
       \end{align*}
   \end{enumerate}
\end{proof}
\noindent
In considerazione della proprietà precedente risulta ragionevole distinguere le radici del generico polinomio $a(x)$ in $\mathcal{R}$ dalle radici che tale polinomio condivide con $x^r - 1$.
\begin{definizione}
   Sia $a(x) \in \mathcal{R}$, allora le radici che $a(x)$ possiede in comune con $x^r - 1$ sono dette {\bf radici principali} (o zeri \footnote{Così definite in \cite{cerruti} pag. 8}).
\end{definizione}
\noindent
La distinzione fra radici e radici principali è anche conseguenza del fatto che le radici di $a(x)$ non sono in generale radici di $a(x)b(x) \mod{x^r-1}$ comunque scelto $b(x)$ in $\mathcal{R}$.

%%%%%%%%%%%%%%%%%%%%%%%%%%%%%%%%%%%%%%%%%%%%%%%%
%%%%%%%%%%%%%%%%%%%%%% SEZIONI    %%%%%%%%%%%%%%%%%%%%
\section{Ideali di $\mathcal{R}$ e sottospazi}
Gli ideali dell'algebra $\mathcal{R}$ sono generati da polinomi monici particolari ed hanno un ruolo interessante nel determinare il prodotto di campi $\mathcal{Q}$.

\begin{teorema} \label{teo:genDiIdeali}
   Sia $\mathfrak{a}$ ideale di $\mathcal{R}$, allora $\mathfrak{a}$ è principale ed è generato da un unico polinomio monico che divide $x^r - 1$.
\end{teorema}
\begin{proof}
   In generale gli anelli di polinomi a coefficienti in un campo sono ad ideali principali\footnote{Ad esempio \cite{berardi}, pag. 64.}. Riportiamo qui la dimostrazione per il caso specifico. \\
   Sia $\mathfrak{a} \unlhd \mathcal{R}$ ed $a(x)$ polinomio monico, di grado minimo, appartenente ad $\mathfrak{a}$. Dunque per definizione segue che
   \begin{align*}
      \mathfrak{a} \supseteq (a(x))
   \end{align*}
   Verifichiamo che vale anche l'inclusione inversa (tesi intermedia: $\mathfrak{a} \subseteq (a(x))$):
   sia $f(x) \in \mathfrak{a}$, allora
   \begin{align*}
      f(x) = a(x)q(x) + r(x)
   \end{align*}
   con $r(x)$ nullo o di grado strettamente inferiore al grado di $a(x)$. Inoltre $r(x) \in \mathfrak{a}$ dal fatto che
   \begin{align*}
      r(x) = f(x) - a(x)q(x)
   \end{align*}
   Allora se $r(x)$ non fosse nullo, avremmo trovato un polinomio in $\mathfrak{a}$ di grado superiore ad $a(x)$ in contraddizione con l'ipotesi di minimalità del suo grado.\\
   Quindi ora ogni ideale è generato da un polinomio monico. Dimostriamo che è un divisore di $x^r-1$ :
   con una strategia simile alla precedente consideriamo
   \begin{align*}
      x^r - 1 = a(x)q(x) + r(x)
   \end{align*}
   dove $q(x)$ ed $r(x)$ sono diversi da prima e con $r(x)$ nullo o di grado strettamente inferiore al grado di $a(x)$. Se quozientiamo l'equazione precedente per $x^r - 1$ otteniamo
   \begin{align*}
      0 \equiv a(x)q(x) + r(x) \qquad \mod{x^r - 1}
   \end{align*}
   e quindi $r(x) \in \mathfrak{a}$; ma dato che il suo grado non può essere inferiore a quello di $a(x)$, analogamente a prima abbiamo che $r(x)\equiv 0 \mod{x^r - 1}$ ed avendo grado minore del grado di $a(x)$ possiamo dire che è identicamente nullo. Dunque
   \begin{align*}
      x^r - 1 = a(x)q(x) + r(x)
   \end{align*}
   cioè
   \begin{align*}
      a(x) \mid x^r - 1
   \end{align*}
   Ora ogni ideale è generato da un polinomio monico che divide $x^r-1$. Rimane da dimostrare che è unico.
   Sia $\mathfrak{a} = (a_{1}(x)) = (a_{2}(x))$ allora possiamo calcolare due polinomi $q(x)$ ed $r(x)$ tali che
   \begin{align*}
      a_{1}(x) = a_{2}(x)q(x) + r(x)(x^r - 1)
   \end{align*}
   ma dato che per, quanto dimostrato prima, anche $a_{2}(x)$ è un divisore di $x^r-1$ esiste $k(x)$ tale che
   \begin{align*}
      a_{1}(x) = a_{2}(x)q(x) + r(x)a_{2}(x)k(x)
   \end{align*}
   quindi $a_{1}(x) \in (a_{2}(x))$. In modo analogo si dimostra che $a_{2}(x) \in (a_{1}(x))$ e quindi
   $a_{1}(x) = a_{2}(x) $.
\end{proof}
\noindent
Riassumendo
\begin{align*}
   \mathfrak{a} \unlhd \mathcal{R} \Leftrightarrow
      \exists ! \phantom{a}  a(x) \in \mathcal{R} \phantom{a} monico, a(x) \mid x^r - 1 , \mathfrak{a} = (a(x))
\end{align*}
A questo punto ci poniamo due problemi: dato il polinomio $f(x)$ come possiamo stabilire qual è il più piccolo ideale $\mathfrak{a}$ in $\mathcal{R}$ che lo contiene? Dato $\mathcal{R}$, quanti sono i suoi ideali?
\begin{corollario}
   Sia $f(x) \in \mathfrak{a} \unlhd \mathcal{R}$ allora $\mathfrak{a}$ è generato da $a(x)$ ricavato come il minimo comune multiplo fra $f(x)$ ed $x^r - 1$.
\end{corollario}
\begin{proof}
   Se $f(x) \in \mathfrak{a}$, allora per definizione di ideale $a(x) \mid f(x)$, cioè $f(x) = a(x)k(x)$ per qualche $k(x)$ in $\mathbb{F}[x]$. Inoltre, dal teorema precedente $a(x)$ è divisore di $x^r-1$, quindi $a(x) \mid x^r - 1$.\\ $\Longrightarrow$ $a(x)$ è divisore comune di $f(x)$ e di $x^r-1$, e se per assurdo non fosse il massimo dei divisori possibili avremmo una contraddizione con la sua minimalità postulata nel teorema precedente.
\end{proof}
\begin{corollario} \label{coroll:cardIdealiDiR}
   L'algebra $\mathcal{R}_{r,q}$ possiede sono esattamente $2^{l}$ ideali, per $l$ cardinalità dell'insieme delle etichette $\mathscr{L}$.
\end{corollario}
\begin{proof}
   Dal capitolo dedicato alla fattorizzazione di $x^r-1$ sappiamo che ci sono $l$ polinomi monici irriducibili che dividono $x^r-1$ e dal teorema precedente sappiamo che ogni ideale è generato da un divisore monico di $x^r - 1$.
   Inoltre per l'ipotesi $(r,q) = 1$ i fattori di $x^r-1$ sono tutti distinti.
   Dunque il generatore di un ideale è un prodotto di polinomi monici irriducibili che dividono $x^r-1$ ciascuno dei quali può essere scelto da un insieme di $l$ elementi.
\end{proof}

In un'algebra la dimensione di un ideale $\mathfrak{a}$ è la dimensione di $\mathfrak{a}$ come sottospazio vettoriale.
Il prossimo corollario\footnote{Proprietà $3.2.5$ \cite{montabone}.} lega la dimensione di $\mathfrak{a}$ con il grado del suo polinomio generatore.
\begin{corollario} \label{cor:baseDellIdeale}
   Sia $(a(x)) = \mathfrak{a} \unlhd \mathcal{R}$ e sia $deg(a(x)) = d = r-k$. Allora $dim(\mathfrak{a}) = r-d = k $.
\end{corollario}
\begin{proof}
   Partiamo dall'insieme $\lbrace \sigma_{j}(a(x)) = a(x)x^{j} \rbrace_{j=0}^{k-1}$. Questo costituisce una base di $\mathfrak{a}$; che sia un insieme di generatori per l'ideale generato da $a(x)$ è evidente. Dimostriamo che è un insieme di elementi linearmente indipendenti: \\
   la combinazione lineare
   \begin{align*}
      \lambda_{0} a(x)x^{0} + \lambda_{1} a(x)x^{1}+ \cdots \lambda_{k-1} a(x)x^{k-1}
   \end{align*}
   si annulla se e solo se
   \begin{align*}
      \lambda(x)a(x) = 0
   \end{align*}
   dove $\lambda(x) = \sum_{j=0}^{k-1}\lambda_{j}x^{j}$. Il polinomio $\lambda(x)a(x)$ ha al più grado $r-1$ ed appartiene all'ideale $\mathfrak{a}$. Per questo motivo si può annullare solo se $\lambda(x)$ è identicamente nullo, cioè se e solo se $\lambda_{j} = 0 $ per ogni $j$.
\end{proof}
\noindent
Aumentando quindi il grado del generatore diminuisce la dimensione dello spazio vettoriale e viceversa.
\begin{align*}
   deg(a(x)) + dim(\mathfrak{a}) = r
\end{align*}

Vogliamo ora sfruttare il fatto che ogni ideale, in quanto sottospazio vettoriale, può essere determinato da una matrice. Cominciamo col ricordare che in generale un sottospazio $U$ di uno spazio vettoriale $E$ è generato dalla matrice $G_{U}$ se le sue righe sono costituite dai vettori della base di $E$. Essa non è unica così come non è unica la scelta dei vettori che definiscono la base di $U$. Tuttavia, fissata una base per $E$ ogni elemento della base di $U$ si scrive in modo unico come combinazione lineare degli elementi della base di $E$.\\
La matrice generatrice di $U$, per $\lbrace \mathbf{e}_{j} \rbrace_{j=0}^{r-1}$ base di $E$ e per $\lbrace \mathbf{u}_{j} \rbrace_{j=0}^{r-1}$ base di $U$, è data da
\begin{align*}
G_{U}
=
\left(
\begin{array} {c}
  \mathbf{u}_{0} \\
  \mathbf{u}_{1} \\
  \vdots \\
  \mathbf{u}_{k-1}
\end{array}
\right)
=
\left(
\begin{array} {c c c c}
u_{0,0} & u_{0,1} & & u_{0,r-1}  \\
u_{1,0} & u_{1,1} & & u_{1,r-1}  \\
\vdots &  & & \vdots  \\
u_{k-1,0} & u_{k-1,1} & & u_{k-1,r-1}
\end{array}
\right)
\end{align*}
Un generico vettore $\mathbf{a}$ di $E$ appartiene al sottospazio $U$ se può essere scritto come
\begin{align*}
\mathbf{a}
=
(\lambda_{0}, \lambda_{1}, \dots , \lambda_{k-1})
\left(
\begin{array} {c}
  \mathbf{u}_{0} \\
  \mathbf{u}_{1} \\
  \vdots \\
  \mathbf{u}_{k-1}
\end{array}
\right)
\end{align*}
infatti ad ogni $k$-upla $(\lambda_{0}, \lambda_{1}, \dots , \lambda_{k-1})$ corrisponde biunivocamente un vettore del sottospazio $U$.
\\
Torniamo ora al nostro ideale $\mathfrak{a} \unlhd \mathcal{R}$ generato dal polinomio $a(x)$; come appena visto ammette $\lbrace a(x)x^{j} \rbrace_{j=0}^{k-1}$ come sua base e dall'interludio sulle matrici generatrici dei sottospazi vettoriali possiamo dedurre immediatamente la dimostrazione del seguente
\begin{corollario} \label{teo:matGenId}
   Una matrice generatrice dell'ideale $\mathfrak{a} = (a(x)) \unlhd \mathcal{R}$ può essere scritta come
   \begin{align*}
  G_{\mathfrak{a}}
  =
  \left(
  \begin{array} {c}
    a(x) \\
    a(x)x^{1} \\
    \vdots \\
    a(x)x^{k-1}
  \end{array}
  \right)
  =
  \left(
  \begin{array} {c c c c c c c c }
  a_{0} & & \cdots & & a_{r-k} & 0 & \cdots & 0   \\
  0 & a_{0} & & \cdots& a_{r-k - 1} & a_{r-k} & 0 \cdots & 0   \\
   & \vdots & &  &  &  & \vdots &    \\
  0 & \cdots & 0 & a_{0} &  \cdots &  & \cdots & a_{r-k}
  \end{array}
  \right)
  \end{align*}
  che è una matrice circolante non quadrata.
\end{corollario}

Da quanto visto i mattoni fondamentali per costruire gli ideali di $\mathcal{R}$ sono i fattori irriducibili di $x^r-1$ il cui prodotto ne determina i divisori; nella prossima definizione daremo loro un nome. Tuttavia anche il prodotto dei mattoni che vengono scartati nella costruzione di $\mathfrak{a}$ assumeranno un significato (e quindi un nome) particolare, la cui scelta sarà chiarita nel corso del capitolo sui codici lineari.
\begin{definizione}
   I divisori monici di $x^r - 1$ in $\mathcal{R}$ sono detti {\bf divisori di} $\mathcal{R}$. Sia $\mathfrak{a}$ ideale generato da $a(x)$ determinato da un prodotto di divisori di $\mathcal{R}$, allora il prodotto dei divisori monici di $x^r - 1$ che non dividono $a(x)$ è un polinomio indicato come
   \begin{align*}
      \hat{a}(x) = (x^r-1)/a(x)
   \end{align*}
   ed è detto {\bf polinomio di controllo}. Convenzionalmente se $\mathfrak{a}$ è generato dal polinomio $a(x)$ allora l'ideale generato dal polinomio $\hat{a}(x)$ è indicato con $\hat{\mathfrak{a}}$.
\end{definizione}
\noindent
Ogni divisore di $\mathcal{R}$ è determinato univocamente dall'insieme delle sue radici $r$-esime dell'unità (sottoinsieme di tutte le radici di $x^r -1$), al quale corrisponde univocamente l'insieme dei loro esponenti.
\begin{definizione}
   Si definiscono {\bf radici principali} dell'ideale $\mathfrak{a}$ le radici principali del divisore che lo genera. L'insieme degli esponenti delle radici principali di $\mathfrak{a} = (a(x))$ si denotano con $Esp(\mathfrak{a})$:
   \begin{align*}
      Esp(\mathfrak{a}) := \lbrace t \mid 0\leq t \leq r-1, a(\xi^{t}) = 0 \rbrace \subseteq \mathbb{Z}_{r}
   \end{align*}
\end{definizione}
\noindent
Chiaramente se $v \in Esp(\mathfrak{a})$ allora ogni altro intero dell'orbita $O(v)$ deve appartenere a $Esp(\mathfrak{a})$.
\\
Presentiamo due proprietà sull'insieme $Esp(\mathfrak{a})$ appena definito.

\begin{prop}
   Sia $a(x)$ divisore in $\mathcal{R}_{r,q}$, generatore dell'ideale $\mathfrak{a}$ allora
   \begin{align*}
      Esp(\mathfrak{a}) = \mathbb{Z}_{r} \setminus Esp(\hat{\mathfrak{a}})
   \end{align*}
\end{prop}
\begin{proof}
   Per definizione abbiamo che
   \begin{align*}
      a(x) = \prod_{ t \in Esp(\mathfrak{a}) } (x - \xi^{t})
      & &
      \hat{a}(x) = \prod_{ t \in Esp(\hat{\mathfrak{a}}) } (x - \xi^{t})
   \end{align*}
   inoltre $a(x) \hat{a}(x) = x^r-1$, da cui $Esp(\mathfrak{a})$ e $Esp(\hat{\mathfrak{a}}) $ formano una partizione di $\mathbb{Z}_{r}$.
\end{proof}
%%
\begin{prop}
   Siano $\mathfrak{a}, \mathfrak{b}$ ideali di $\mathcal{R}_{r,q}$, $\mathfrak{a} \subset \mathfrak{b}$, allora
   $Esp(\mathfrak{a}) \supset Esp(\mathfrak{b})$.
\end{prop}
\begin{proof}
   Se $\mathfrak{a} = (a(x))$ e $\mathfrak{b} = (b(x))$ allora per definizione
   \begin{align*}
      (a(x))
      = \lbrace h(x)\prod_{ t \in Esp(\mathfrak{a}) } (x - \xi^{t}) \mid h(x) \in \mathcal{R} \rbrace
      \\
      (b(x))
      = \lbrace h(x)\prod_{ t \in Esp(\mathfrak{b}) } (x - \xi^{t}) \mid h(x) \in \mathcal{R} \rbrace
   \end{align*}
   Dato che $\mathfrak{a} \subset \mathfrak{b}$ allora esiste almeno un indice $t_{0}$ tale che $(x - \xi^{t_{0}}) $ è un divisore di $a(x)$ ma non divide $b(x)$, quindi
   \begin{align*}
      Esp(\mathfrak{b}) \subset Esp(\mathfrak{a})
   \end{align*}
   dato che $t_{0}$ appartiene ad $Esp(\mathfrak{a})$ ma non ad $Esp(\mathfrak{b})$.
\end{proof}

\begin{osservazione}
$Esp$ può dare lo spunto per definire, in parallelo alla definizione delle mappe $I$ e $V$ della geometria algebrica che agiscono fra insiemi algebrici affini dello spazio affine $n$-dimensionale e ideali dell'anello dei polinomi ad $n$ indeterminate\footnote{Presentate ad esempio in \cite{perrin}.} altre due corrispondenze nel contesto che stiamo esaminando:
\begin{align*}
   I: \mathcal{P}(\mathscr{L})  &\longrightarrow
                       \lbrace \mathfrak{a} \mid \mathfrak{a} \trianglelefteq \mathcal{R}_{r,q} \rbrace  \\
              A &\longmapsto I(A) = (\prod_{v\in A} M^{(v)}(x)) = \mathfrak{a}_{A} \\
   V: \lbrace \mathfrak{a} \mid \mathfrak{a} \trianglelefteq \mathcal{R}_{r,q} \rbrace  &\longrightarrow
	                \mathcal{P}(\mathscr{L})  \\
              (\prod_{v\in A} M^{(v)}(x)) = \mathfrak{a}_{A}  &\longmapsto V(\mathfrak{a}_{A}) = A
\end{align*}
dove con $\mathcal{P}(\mathscr{L})$ intendiamo l'insieme delle parti dell'insieme delle etichette. Con queste notazioni
\begin{align*}
   Esp(\mathfrak{a}) = \bigcup_{v \in \mathfrak{a}_{A}} O(v)
\end{align*}

\end{osservazione}



%%%%%%%%%%%%%%%%%%%%%%%%%%%%%%%%%%%%%%%%%%%%%%%%
%%%%%%%%%%%%%%%%%%%%%% subSEZIONI    %%%%%%%%%%%%%%%%%%%%
\subsection{Ideali massimali e minimali}

Ora lo scopo è quello di definire formalmente la corrispondenza biunivoca fra gli ideali ed i sottoinsiemi dell'insieme delle etichette $\mathscr{L}_{r,q}$.\\
Verificheremo quindi che le corrispondenze del seguente diagramma sono biiezioni e determineremo gli ideali massimali e gli ideali minimali scegliendo sottoinsiemi di  $\mathscr{L}$ opportuni.

\vspace{0.2cm}

\[
\begindc{\commdiag}[3]
%insiemi
\obj(0,12)[A]{$ \lbrace A \mid A \subseteq \mathscr{L} \rbrace $}
%[R]{$ \lbrace \mathfrak{a} \mid \mathfrak{a} \trianglelefteq \mathcal{R}_{r,q} \rbrace $}
\obj(50,24)[O]{$ \lbrace \cup_{t \in A} O(t)  \mid A \subseteq \mathscr{L} \rbrace $}
%[A]{$ \lbrace A \mid A \subseteq \mathscr{L} \rbrace $}
\obj(50,0)[R]{$ \lbrace \mathfrak{a} \mid \mathfrak{a} \trianglelefteq \mathcal{R}_{r,q} \rbrace $}
%[O]{$ \lbrace \cup_{t \in A} O(t)  \mid A \subseteq \mathscr{L} \rbrace $}


%frecce
\mor{A}{R}{}
\mor{A}{O}{}
\mor{O}{R}{}

\enddc
\]

\vspace{0.6cm}

\noindent
La prima parte della corrispondenza che vogliamo determinare è conseguenza del seguente
\begin{teorema}\label{teo:corrispEtiIde1}
   Per ogni sottoinsieme $A$ dell'insieme delle etichette $\mathscr{L}$ esiste un ideale $\mathfrak{a}$ di $\mathcal{R}_{r,q}$ generato dal prodotto dei fattori irriducibili $M^{(v)}(x)$ per $v\in A$.
\end{teorema}
\begin{proof}
   Segue dal che ogni elemento del sottoinsieme $A$ di $\mathscr{L}$ è il rappresentante di un'orbita i cui elementi determinano le radici di un divisore irriducibile di $x^r - 1$ come esponenti della radice primitiva $r$-esima dell'unità.\\
   In altre e più comprensibili parole, per ciascun elemento $v$ di $A$ possiamo determinare in modo unico un insieme e un polinomio fra loro correlati:
   \begin{align*}
      \lbrace \xi^{t} \rbrace_{t \in O(v)} \qquad \qquad \prod_{t \in O(v)} (x - \xi^{t}) = M^{(v)}(x)
   \end{align*}
   Quindi $A$ determina un polinomio come prodotto dei $M^{(v)}(x)$ per $v$ in $A$
   \begin{align*}
      a(x) = \prod_{v\in A} M^{(v)}(x) = \prod_{v\in A} (  \prod_{t \in O(v)} (x - \xi^{t}) )
   \end{align*}
   che è un divisore in $\mathcal{R}_{r,q} $ e, per il teorema \ref{teo:genDiIdeali}, determina univocamente un ideale di $\mathfrak{a}$ di $\mathcal{R}_{r,q}$.
\end{proof}
\noindent
La seconda parte della corrispondenza che vogliamo ottenere è data dal seguente teorema:
\begin{teorema}\label{teo:corrispEtiIde2}
   Per ogni ideale $\mathfrak{a} = (a(x))$ di $\mathcal{R}_{r,q}$ esiste un unico sottoinsieme dell'insieme delle etichette $\mathscr{L}$ determinato dagli indici $v$ dei fattori irriducibili $M^{(v)}(x)$ che costituiscono il generatore $a(x)$.
\end{teorema}
\begin{proof}
   Da quanto visto nel teorema sui generatori degli ideali, $a(x)$ è il prodotto di un insieme di divisori monici irriducibili di $x^r - 1$:
   \begin{align*}
      a(x) =  M^{(v_{1})}(x) M^{(v_{2})}(x) \cdots M^{(v_{m})}(x) = \prod_{j=1}^{m} (  \prod_{t \in O(v_{j})} (x - \xi^{t}) )
   \end{align*}
   Allora l'insieme $ \lbrace v_{1}, \dots , v_{m} \rbrace$ costituisce un sottoinsieme di $\mathscr{L}$ unicamente determinato da $a(x) $, che è proprio l'insieme cercato.
\end{proof}
\noindent
Per entrambe le dimostrazioni siamo passati per l'unione delle orbite determinate dalle etichette $\lbrace \xi^{t} \rbrace_{t \in O(v)}$, come indicato in modo schematico nel diagramma iniziale. Ogni ideale è unione di polinomi ciascuno dei quali corrisponde ad un'orbita rappresentata da un elemento dell'insieme delle etichette. E viceversa.

\begin{esempio}
   Sia $r = 9$ e $q = 2$, allora come visto in \ref{ese:fattor2_9} sappiamo che $ G = \mathbb{Z}_{9}^{\star} $ e che $\mathscr{L} = \lbrace 0,1,3 \rbrace$. I possibili ideali di $\mathcal{R}_{9,2} $ sono generati dai possibili prodotti dei divisori irriducibili di $x^9-1$, cioè da
   \begin{align*}
      M^{(0)}(x)&= x+1\\
      M^{(1)}(x) &= x^6+x^3+1\\
      M^{(3)}(x) &= x^2+x+1
   \end{align*}
   Volendo elencare gli ideali definiti dal sottoinsieme $A$ di $\mathscr{L} $ con la notazione $\mathfrak{a}_{A}$, si ottiene
   \begin{align*}
      &\mathfrak{a}_{\lbrace 0 \rbrace} = (M^{(0)}(x)) = (x-1) \\
      &\mathfrak{a}_{\lbrace 1 \rbrace} = (M^{(1)}(x)) = (x^6+x^3+1) \\
      &\mathfrak{a}_{\lbrace 3 \rbrace} = (M^{(3)}(x)) = (x^2+x+1) \\
      &\mathfrak{a}_{\lbrace 0,1 \rbrace} = (M^{(0)}(x)M^{(1)}(x)) = (x^3 -1) \\
      &\mathfrak{a}_{\lbrace 0,3 \rbrace} = (M^{(0)}(x)M^{(3)}(x)) = (x^7 + x^6 + x^4 + x^3 + x + 1) \\
      &\mathfrak{a}_{\lbrace 1,3 \rbrace}
                    = (M^{(1)}(x)M^{(3)}(x)) = (x^8 +x^7 +x^6 + x^5 + x^4 + x^3 + x^2 + x + 1) \\
      &\mathfrak{a}_{\lbrace 0,1,3 \rbrace} = (M^{(0)}(x)M^{(1)}(x)M^{(3)}(x)) = (x^9 -1) = (0) \\
      &\mathfrak{a}_{\emptyset} = (1)
   \end{align*}
   che sono, come ci aspettavamo $8$ ideali, contando l'ideale improprio e l'ideale banale.
\end{esempio}
Ricordando che un ideale $\mathfrak{a}$ è massimale se fra $\mathfrak{a}$ ed $(1)$ non ci sono ideali intermedi ed è minimale se non ci sono ideali intermedi fra $\mathfrak{a}$ ed $(0)$, vogliamo esaminare quali sono gli ideali massimali e quali gli ideali minimali in $\mathcal{R}$. Come già osservato nel precedente esempio, scegliendo $A = \mathscr{L}$, allora l'ideale definito da $A$ è l'ideale triviale. Se invece scegliamo $A = \emptyset$ l'ideale definito da $A$ è l'ideale improprio. \\
Scegliendo invece $A = \lbrace v \rbrace$ allora l'ideale corrispondente è $(M^{(v)}(x))$. Inoltre $deg(M^{(v)}(x)) = \arrowvert O(v) \arrowvert$ ed $dim(\mathfrak{a}_{\lbrace v \rbrace}) = r - \arrowvert O(v) \arrowvert$. È facile verificare che gli ideali determinati da sottoinsiemi di $\mathscr{L}$ con un solo elemento sono ideali {\bf massimali}, e che sono gli unici ideali massimali in $\mathcal{R}$.\\
Se invece $A = \mathscr{L} \setminus \lbrace v \rbrace$, cioè il complementare del sottoinsieme esaminato nel caso precedente, allora
$deg(\hat{M}^{(v)}(x) ) = r - \arrowvert O(v) \arrowvert$ ed $dim(\mathfrak{a}_{\mathscr{L} \setminus \lbrace v \rbrace}) = \arrowvert O(v) \arrowvert$. Si può verificare in analogia con il caso precedente, che gli ideali determinati da sottoinsiemi di $\mathscr{L}$ di questo tipo sono {\bf minimali}, e che sono gli unici minimali in $\mathcal{R}$.\\
Possiamo allora aggiungere un pezzo al corollario \ref{coroll:cardIdealiDiR} che oltre alla quantità fornisce informazioni sulla qualità degli ideali.

\begin{corollario}\label{cor:cardIdealiDiR}
   In $\mathcal{R}_{r,q}$ ci sono esattamente $2^{l}$ ideali, $l$ dei quali massimali ed $l$ dei quali minimali.
\end{corollario}
\begin{proof}
   Il sottoinsieme $A$ di $\mathscr{L}$ può essere scelto come singleton esattamente in $l$ modi, al quale corrispondono $l$ possibili ideali massimali; può essere scelto come complementare di un singleton in $l$ modi diversi al quale corrispondono $l$ ideali minimali.
\end{proof}


%%%%%%%%%%%%%%%%%%%%%%%%%%%%%%%%%%%%%%%%%%%%%%%%
%%%%%%%%%%%%%%%%%%%%%% subSEZIONI    %%%%%%%%%%%%%%%%%%%%
\subsection{Ideali ortogonali}

Definiamo l'ortogonalità geometrica di due polinomi in $\mathcal{R}_{r,q}$ a partire dal prodotto scalare usuale nell'algebra dei vettori $\mathcal{V}_{r, q}^{c}$ che viene ereditata da $\mathcal{R}$ grazie all'isomorfismo $\psi_{2}$.


\begin{definizione}
   Dati $a(x)$ e $b(x)$ in $\mathcal{R}$ si definisce loro {\bf prodotto scalare} l'elemento del campo $\mathbb{F}_{q}$ determinato da
   \begin{align*}
      \langle a(x), b(x) \rangle = \sum_{k \in \mathbb{Z}_{r} } a_{k}b_{k}
   \end{align*}
   che corrisponde all'usuale prodotto scalare dei vettori circolanti  $\psi_{2}(a(x)), \psi_{2}(b(x))$ corrispondenti.
\end{definizione}
\begin{definizione}
   Si dice che i polinomi $a(x)$ e $b(x)$ di $\mathcal{R}$ sono geometricamente ortogonali, o {\bf g-ortogonali}, se il loro prodotto scalare è nullo.
   Mentre si dice che sono algebricamente ortogonali, o {\bf a-ortogonali}, se il loro prodotto in $\mathcal{R}$ è nullo.
\end{definizione}
\begin{osservazione} \label{oss:aOrtogonali}
   Possiamo osservare che $a(x)$ e $b(x)$ sono g-ortogonali se
   \begin{align*}
      \sum_{k \in \mathbb{Z}_{r} } a_{k}b_{k} = 0
   \end{align*}
   Mentre sono a-ortogonali se per ogni $j$ in $\mathbb{Z}_{r}$
   \begin{align*}
      \sum_{k \in \mathbb{Z}_{r}} a_{k}b_{j-k} = 0
   \end{align*}
\end{osservazione}

Con la definizione precedente possiamo considerare i sottospazi ortogonali e gli ideali ortogonali di $\mathcal{R}$.
\begin{definizione}
   sia $S\subseteq \mathcal{R}$ sottoinsieme, allora si definisce {\bf sottospazio a-ortogonale} generato da $S$ l'insieme
   \begin{align*}
      S_{a}^{\perp} := \lbrace f(x) \in \mathcal{R} \mid f(x)g(x) = 0 \quad \forall g(x) \in S \rbrace
   \end{align*}
   e si definisce {\bf sottospazio g-ortogonale} generato da $S$ l'insieme
   \begin{align*}
      S_{g}^{\perp} := \lbrace f(x) \in \mathcal{R} \mid \langle f(x), g(x)\rangle = 0 \quad \forall g(x) \in S \rbrace
   \end{align*}
   Se $S = \mathfrak{a}$ ideale di $\mathcal{R}$ allora $\mathfrak{a}_{a}^{\perp}$ è ancora un ideale, detto {\bf ideale a-ortogonale}.
\end{definizione}
\begin{esempio} \label{ese:ortogonali1}
   In  $\mathcal{R}_{5,2}$ i polinomi $x^3 + 1$ ed $x^2+x$ sono g-ortogonali ma non sono a-ortogonali.
\end{esempio}

Si può trovare una correlazione fra la a-ortogonalità e la g-ortogonalità\footnote{\cite{cerruti} proprietà 1.14 pag. 17.}:
\begin{osservazione} \label{oss:aOrtogonali2}
   Dati i polinomi $a(x)$ e $b(x)$, segue, dall'osservazione \ref{oss:aOrtogonali} e dalle definizioni, che
   \begin{align*}
      b(x^{-1}) = \sum_{k \in \mathbb{Z}_{r}} b_{-k}x^{k} = 0 \\
      a(x)b(x) = 0 \iff \forall j \in \mathbb{Z}_{r}  \sum_{k \in \mathbb{Z}_{r}} a_{k}b_{j-k} = 0 \\
      \langle a(x), x^{j}b(x^{-1}) \rangle = \sum_{k \in \mathbb{Z}_{r} } a_{k}b_{j-k}
   \end{align*}
   Da cui risulta immediato verificare che $a(x)$ e $b(x)$ sono a-ortogonali se e solo se $a(x)$ è g-ortogonale a $b(x^{-1})$ e ad ogni suo shift.
\end{osservazione}
Sapendo che tutti gli ideali in $\mathcal{R}$ sono generati da un divisore, dato $\mathfrak{a} = (a(x))$, qual è il divisore che genera $\mathfrak{a}_{a}^{\perp}$?
\begin{lemmax} \label{lemma:idOrtogonali}
   Sia $\mathfrak{a} $ ideale di $\mathcal{R}$ generato da $a(x)$, allora
   \begin{enumerate}
      \item $\mathfrak{a}_{a}^{\perp} = (\hat{a}(x))$
      \item $\mathfrak{a}_{g}^{\perp} $ non è in generale un ideale.
   \end{enumerate}
\end{lemmax}
\begin{proof}
   \begin{enumerate}
      \item Tutti i polinomi $a$-ortogonali ad $a(x)$ sono quelli il cui prodotto con $a(x)$ risulta essere $0$ modulo $x^r - 1$. Il più piccolo polinomio con tale proprietà è il polinomio di controllo
      \begin{align*}
         \hat{a}(x) = (x^r-1)/a(x)
      \end{align*}
      e quindi $\mathfrak{a}_{a}^{\perp} = (\hat{a}(x))$.
      \item $\mathfrak{a}_{g}^{\perp} $ è chiuso per la somma grazie alla linearità del prodotto scalare, ma all'esempio \ref{ese:ortogonali1} in $\mathcal{R}_{5,2}$ per $a(x) = x^2+x$ e $b(x) = x^3 + 1$ si ha che $b(x) \in \mathfrak{a}_{g}^{\perp}$, ma $x^{2}b(x) \notin \mathfrak{a}_{g}^{\perp}$:
      \begin{align*}
         \langle x^3 + 1, x^2+x \rangle = 0\\
         \langle x^2 + 1, x^2+x \rangle \neq 0
      \end{align*}
   \end{enumerate}
\end{proof}
L'ideale a-ortogonale può essere semplicemente detto {\bf ideale ortogonale}, senza che sorgano ambiguità. \\
La prossima definizione permette di riformulare il lemma in una proprietà nella quale si esamina una nuova relazione fra ideali ortogonali e spazi g-ortogonali.
\begin{definizione}
   sia $\mathfrak{a} \trianglelefteq \mathcal{R}$ allora l'ideale $\bar{\mathfrak{a}}$ le cui radici principali sono le inverse delle radici principali di $\mathfrak{a}$ è detto {\bf ideale coniugato}.
\end{definizione}
\begin{osservazione}
   Si verifica che
   \begin{align*}
      Esp(\bar{\mathfrak{a}}) = \cup_{v \in Esp(\mathfrak{a}) } O(-v)
   \end{align*}
   Inoltre se $a(x)$ è il divisore che genera l'ideale $\mathfrak{a}$, allora $\bar{\mathfrak{a}} = (a(x^{-1}))$.
\end{osservazione}
%%%%% prop finale ideali ortogonali.
\begin{prop}\label{prop:IdOrtogonali}
   Con le notazioni che abbiamo introdotto valgono le seguenti proprietà:
   \begin{enumerate}
      \item Sia $\mathfrak{a} = (a(x))$ ideale di $\mathcal{R}$, allora
      \begin{align*}
         \mathfrak{a}_{a}^{\perp} = (\hat{a}(x)) \qquad  \mathfrak{a}_{a}^{\perp} = \bar{\mathfrak{a}}_{g}^{\perp}
      \end{align*}
      \item Per $v \in \mathscr{L}$ segue che
      \begin{align*}
           (M^{(v)}(x))_{a}^{\perp} = (\hat{M}^{(v)}(x)) = (M^{(-v)}(x))_{g}^{\perp}
      \end{align*}
   \end{enumerate}
\end{prop}
\begin{proof}
   Il secondo punto è conseguenza del primo, mentre per il primo punto la prima equazione è conseguenza del lemma \ref{lemma:idOrtogonali}.
   Rimane da dimostrare per doppia inclusione che
   \begin{align*}
        \mathfrak{a}_{a}^{\perp} = \bar{\mathfrak{a}}_{g}^{\perp}
    \end{align*}
    Se $f(x) \in \mathfrak{a}_{a}^{\perp}$ allora $f(x)a(x) = 0$ modulo $x^r - 1$, e dall'osservazione \ref{oss:aOrtogonali2} segue che per ogni $j$ in $\mathbb{Z}_{r}$
    \begin{align*}
       \langle f(x), x^{j}a(x^{-1}) \rangle = 0
    \end{align*}
    Ma dato che dal corollario \ref{cor:baseDellIdeale} $\lbrace a(x^{-1})x^{j} \rbrace_{j=0}^{k-1}$ è un sistema di generatori di  $\bar{\mathfrak{a}}$, segue che $f(x) \in \bar{\mathfrak{a}}_{g}^{\perp}$. \\
    Viceversa se $f(x) \in \bar{\mathfrak{a}}_{g}^{\perp}$, allora segue che $f(x)$ è ortogonale ad ogni elemento della base di $\bar{\mathfrak{a}}$, quindi come prima, per ogni $j$ in $\mathbb{Z}_{r}$
    \begin{align*}
       \langle f(x), x^{j}a(x^{-1}) \rangle = 0
    \end{align*}
    da cui dall'osservazione \ref{oss:aOrtogonali2} $f(x) \in \mathfrak{a}_{a}^{\perp}$.
\end{proof}


% \begin{esempio}
%    Primo dubbio: come è possibile $\mathfrak{a}_{a}^{\perp} = \bar{\mathfrak{a}}_{g}^{\perp}$ se $\mathfrak{a}_{g}^{\perp}$ non è in generale un ideale?
% \end{esempio}

In questo paragrafo abbiamo potuto osservare come gli isomorfismi presentati nel capitolo \ref{cap:algebreisomorfe} non siano solo un esercizio di stile. Consentono ad $\mathcal{R}_{r,q}$ di ereditare il prodotto scalare da cui l'ortogonalità geometrica\footnote{L'isomorfismo di $\mathcal{R}_{r,q}$ con l'algebra delle matrici circolanti consente di definire sull'algebra dei polinomi una forma $r$-lineare alternante che ad ogni matrice corrispondente associa il suo determinante. Una ricerca in questa direzione è stata seguita in \cite{montabone} e da \cite{wynjones}. Altre strade possono dirigersi verso lo studio degli ideali e degli idempotenti nella rappresentazione di $\mathcal{R}_{r,q}$ come algebra di matrici circolanti e di vettori circolanti.}. Nel prossimo paragrafo continuiamo ad analizzare i polinomi nella loro rappresentazione vettoriale, passando però da $\mathcal{Q}_{r,q}$.


%%%%%%%%%%%%%%%%%%%%%%%%%%%%%%%%%%%%%%%%%%%%%%%%
%%%%%%%%%%%%%%%%%%%%%% subSEZIONI    %%%%%%%%%%%%%%%%%%%%
\section{Elementi idempotenti}

Gli elementi idempotenti giocano un ruolo particolare sia nella teoria dei codici correttori che nello studio degli ideali dell'algebra $\mathcal{R}_{r,q}$. \\
Ricordiamo che la fattorizzazione di $\mathcal{R}$ come prodotto di campi è indicata con
\begin{align*}
    \mathcal{Q}_{r,q}
    :=
    \prod_{v\in \mathscr{L}} \quotient{\mathbb{F}_{q} \lbrack x \rbrack  }{(M^{(v)}(x))}
    =
    \prod_{v\in \mathscr{L}} \mathcal{Q}_{r,q}^{(v)}
\end{align*}
dove il singolo campo appartenente al prodotto del secondo membro è indicato con
\begin{align*}
    \mathcal{Q}_{r,q}^{(v)}
    :=
    \quotient{\mathbb{F}_{q} \lbrack x \rbrack  }{(M^{(v)}(x))}
\end{align*}
per $v\in \mathscr{L}$.\\
La funzione $\gamma$, isomorfismo fra le due strutture $\mathcal{R}$ e $\mathcal{Q}$ è stata definita nel teorema \ref{teo:teoremaGamma} come
\begin{align*}
\gamma :  \mathcal{R}_{r,q}  & \longrightarrow  \prod_{v\in \mathscr{L}} \mathcal{Q}_{r,q}^{(v)}   \\
                        a(x) &\longmapsto (a(x)\mod{M^{(v)}(x)})_{v\in \mathscr{L}}
\end{align*}
che composta con $\mu_{1}^{-1}$ consente di passare da $\mathcal{R}$ al prodotto di campi $\mathcal{P} = \prod_{v\in \mathscr{L}} \mathcal{P}^{(v)}$, dove $\mathcal{P}^{(v)} := \mathbb{F}_{q}(\xi^{v})$.
\begin{definizione}
   Un polinomio $a(x)$ appartenente all'algebra $\mathcal{R}_{r,q}$ è detto {\bf idempotente} se $a(x)^2 = a(x)$.
\end{definizione}
\noindent
In generale non è un problema semplice trovare gli idempotenti in $\mathcal{R}$, senza sfruttare la sua fattorizzazione in campi. Infatti in $\mathcal{Q}$, i cui elementi sono vettori di lunghezza $m(v)$ costituiti da elementi appartenenti ai campi $\mathcal{Q}^{(v)}$, gli idempotenti sono solo i vettori che contengono $1$ o $0$ in ogni posizione.
\begin{prop}
   Gli idempotenti in $\mathcal{Q}$ sono tutti e soli i vettori costituiti da $1$ e da $0$.
\end{prop}
\begin{proof}
   È conseguenza della definizione di campo: gli unici idempotenti nel campo $\mathcal{Q}^{(v)}$ sono l'unità e lo zero, quindi il prodotto di due vettori uguali di $\mathcal{Q}$ dà se stesso se e solo se è costituito da zeri e da unità
\end{proof}
\noindent
% Segue immediatamente il
% \begin{corollario}
%    In $\mathcal{Q}$ ci sono esattamente $2^l$ idempotenti.
% \end{corollario}
\begin{osservazione}
   Si possono considerare in modo equivalente gli idempotenti in $\mathcal{P}$. Ricordando che
   \begin{align*}
      \eta:  \mathcal{R} &\longrightarrow \mathcal{P} =  \prod_{v\in \mathscr{L}} \mathbb{F}(\xi^{v}) \\
	a(x) &\longmapsto  (a(\xi^{v}))_{v\in \mathscr{L}}
    \end{align*}
    analogamente a prima gli unici idempotenti del campo $\mathbb{F}(\xi^{v})$ sono l'unità e lo zero.
\end{osservazione}

Possiamo notare una stretta correlazione fra idempotenti ed ideali, dato che ogni ideale è generato da un idempotente di $\mathcal{Q}$: in un campo gli unici idempotenti sono $0$ ed $1$ e gli unici ideali sono $(0)$ ed $(1)$, inoltre ci sono $2^{l}$ idempotenti, esattamente quanti sono gli ideali. Quali idempotenti generano ideali massimali e quali generano ideali minimali?
\begin{definizione}
   Il vettore $l$-dimensionale {\bf a} di $\mathcal{Q}$ è detto {\bf idempotente minimale} (o primitivo) se è costituito da un vettore di $l-1$ zeri ed ha un solo $1$.
\end{definizione}
\noindent
Gli idempotenti minimali sono evidentemente idempotenti e la loro somma genera tutti gli altri idempotenti. Il \lq\lq complementare\rq\rq di un idempotente minimale, cioé il vettore di $\mathcal{Q}$ che contiene $1$ in ogni posizione tranne in una che contiene $0$ è detto {\bf idempotente massimale}.\\
Indichiamo con con $\mathbf{e}_{v}$ il $v$-esimo idempotente minimale, dove $v$ non corrisponde propriamente alla posizione dell'unità nel vettore ma è un pedice dell'insieme $\mathscr{L}$. Si indica invece con $\mathbf{e}_{v}(x)$ il polinomio in $\mathcal{R}$ la cui immagine tramite $\gamma$ coincide con $\mathbf{e}_{v}$. \\
Dimostriamo quanto detto fino ad ora:
\begin{teorema}
   Sia $\mathbf{e}_{i}$ l'$i$-esimo idempotente minimale, $\mathfrak{1}$ il vettore nel quale ogni elemento è un $1$ ed $\mathfrak{0}$ il vettore nel quale ogni elemento è uno zero in $\mathcal{Q}_{r,q}$. Allora valgono le seguenti proprietà
   \begin{enumerate}
      \item Il numero dei fattori di $x^r-1$ in $\mathbb{F}_{q}$ coincide con il numero degli idempotenti primitivi e con il numero degli idempotenti massimali.
      \item Gli idempotenti sono ortogonali: $\mathbf{e}_{i}\mathbf{e}_{j}  = \mathfrak{0}$ per $i \neq j$.
      \item Gli idempotenti decompongono l'unità: $\sum_{i \in \mathscr{L}}\mathbf{e}_{i}  = \mathfrak{1}$.
      \item Le combinazioni lineari di idempotenti generano tutti gli ideali.
      \item Ogni elemento di $\mathcal{Q}$ si decompone come combinazione lineare a coefficienti in $\mathbb{F}_{q}$ degli idempotenti primitivi.
   \end{enumerate}
\end{teorema}
\begin{proof}
   \begin{enumerate}
      \item Infatti la lunghezza dei vettori di $\mathcal{Q}$ coincide con il numero dei fattori di $x^r-1$.
      \item L'ortogonalità degli idempotenti è una conseguenza della definizione di prodotto nell'algebra $\mathcal{Q}$.
      \item Per definizione di somma in $\mathcal{Q}$ si ha che: $\sum_{j \in \mathscr{L}}\mathbf{e}_{j}  = \mathfrak{1}$.
      \item Si verifica facilmente che $\mathbf{e}_{i} $ è un ideale in $\mathcal{Q}$, così come lo è la somma $\mathbf{e}_{i} + \mathbf{e}_{j}$. Per induzione si ha la tesi.
      \item Sia $(q_{v}(x))_{v \in \mathscr{L}} \in \mathcal{Q}$, allora per la linearità dell'algebra $\mathcal{Q}$ posso scrivere
      \begin{align*}
         (q_{v}(x))_{v \in \mathscr{L}} = \sum_{j \in \mathscr{L}} q_{v}(x) \mathbf{e}_{j}
      \end{align*}
   \end{enumerate}
\end{proof}

\begin{corollario}\label{coroll:idempotentiMinimali}
   Ogni idempotente minimale genera un ideale minimale, ogni idempotente massimale genera un ideale massimale.
\end{corollario}
\begin{proof}
   È immediato verificare che non ci sono ideali fra $\mathbf{e}_{j}$ e $(0)$ e fra il suo complementare e $(1)$.
\end{proof}
